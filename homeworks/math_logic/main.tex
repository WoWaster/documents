\documentclass[10pt, a4paper, oneside]{memoir}
%%% PDF settings
\pdfvariable minorversion 7 % Set PDF version to 1.7.

%%% Fonts and language setup.
\usepackage{polyglossia}
% Setup fonts.
\usepackage{fontspec}
\setmainfont{CMU Serif}
\setsansfont{CMU Sans Serif}
\setmonofont{CMU Typewriter Text}

\usepackage{microtype} % Add fancy-schmancy font tricks

\usepackage{xcolor} % Add colors support.

%% Math
\usepackage{amsmath, amsfonts, amssymb, amsthm, mathtools} % Advanced math tools.
\usepackage{thmtools}
\usepackage{unicode-math} % Allow TTF and OTF fonts in math and allow direct typing unicode math characters.
\unimathsetup{
    warnings-off={
            mathtools-colon,
            mathtools-overbracket
        }
}
\setmathfont{Latin Modern Math} % default
\setmathfont[range={\setminus,\varnothing,\smashtimes}]{Asana Math}

%%% Images
\usepackage{graphicx}
\graphicspath{{figures/}}
\usepackage{import}

%%% Polyglossia setup after (nearly) everything as described in documentation.
\setdefaultlanguage{russian}
\setotherlanguage{english}

\usepackage{csquotes}

%%% Custom commands
\newcommand{\R}{\mathbb{R}}
\newcommand{\N}{\mathbb{N}}
\newcommand{\Z}{\mathbb{Z}}
\newcommand{\Q}{\mathbb{Q}}
\newcommand{\C}{\mathbb{C}}
\newcommand{\id}{\mathrm{id}}
\AtBeginDocument{\renewcommand{\leq}{\leqslant}}
\AtBeginDocument{\renewcommand{\geq}{\geqslant}}
\AtBeginDocument{\renewcommand{\Re}{\operatorname{Re}}}
\AtBeginDocument{\renewcommand{\Im}{\operatorname{Im}}}
\AtBeginDocument{\renewcommand{\phi}{\varphi}}
\AtBeginDocument{\renewcommand{\epsilon}{\varepsilon}}

%%% theorem-like envs
\theoremstyle{definition}

\declaretheoremstyle[spaceabove=0.5\topsep,
    spacebelow=0.5\topsep,
    headfont=\bfseries\sffamily,
    bodyfont=\normalfont,
    headpunct=.,
    postheadspace=5pt plus 1pt minus 1pt]{myStyle}
\declaretheoremstyle[spacebelow=\topsep,
    headfont=\bfseries\sffamily,
    bodyfont=\normalfont,
    headpunct=.,
    postheadspace=5pt plus 1pt minus 1pt,]{myStyleWithFrame}
\declaretheoremstyle[spacebelow=\topsep,
    headfont=\bfseries\sffamily,
    bodyfont=\normalfont,
    headpunct=.,
    postheadspace=5pt plus 1pt minus 1pt,
    qed=\blacksquare]{myProofStyleWithFrame}

\usepackage[breakable]{tcolorbox}
\tcbset{sharp corners=all, colback=white}
% \tcolorboxenvironment{theorem}{}
% \tcolorboxenvironment{theorem*}{}
% \tcolorboxenvironment{axiom}{}
% \tcolorboxenvironment{assertion}{}
% \tcolorboxenvironment{lemma}{}
% \tcolorboxenvironment{proposition}{}
% \tcolorboxenvironment{corollary}{}
% \tcolorboxenvironment{definition}{}
% \tcolorboxenvironment{proofReplace}{toprule=0mm,bottomrule=0mm,rightrule=0mm, colback=white, breakable }

\declaretheorem[name=Теорема, style=myStyleWithFrame]{theorem}
\declaretheorem[name=Теорема, numbered=no, style=myStyleWithFrame]{theorem*}
\declaretheorem[name=Аксиома, sibling=theorem, style=myStyleWithFrame]{axiom}
\declaretheorem[name=Преположение, sibling=theorem, style=myStyleWithFrame]{assertion}
\declaretheorem[name=Лемма, style=myStyleWithFrame]{lemma}
\declaretheorem[name=Предложение, sibling=theorem, style=myStyleWithFrame]{proposition}
\declaretheorem[name=Следствие, numberwithin=theorem, style=myStyleWithFrame]{corollary}

\declaretheorem[name=Определение, style=myStyleWithFrame]{definition}
\declaretheorem[name=Свойство, style=myStyle]{property}
\declaretheorem[name=Свойства, numbered=no, style=myStyle]{propertylist}

\declaretheorem[name=Пример, style=myStyle]{example}
\declaretheorem[name=Замечание, numbered=no, style=myStyle]{remark}

\declaretheorem[name=Доказательство, numbered=no, style=myProofStyleWithFrame]{proofReplace}
\renewenvironment{proof}[1][\proofname]{\begin{proofReplace}}{\end{proofReplace}}
% \declaretheorem[name=Доказательство, numbered=no, style=myProofStyleWithFrame]{longProof}

%%% Memoir settings
\chapterstyle{ger}
\setlength{\headheight}{2\baselineskip}

%%% HyperRef
\usepackage{hyperref}

%%% Memoir settings
\chapterstyle{ger}
\setlength{\headheight}{2\baselineskip}

\title{Доказательство теоремы \textbf{2, b, \S 2}}
\author{Пономарев Николай, 244 группа}
\date{}

\usepackage{pdflscape}

\semiisopage
\begin{document}

\maketitle
Необходимо доказать следующую теорему с помощью средств аксиоматической теории чисел:
\begin{theorem*}
    Если $a = bq + c$, то совокупность общих делителей чисел $a$ и $b$ совпадает с совокупностью общих делителей чисел $b$ и $c$; в частности, $(a, b) = (b, c)$.
\end{theorem*}

Введем предикатный символ $\mid$:
\begin{definition}[делимость]
    \begin{equation}
        y \mid x \Leftrightarrow \exists z (x = yz)
    \end{equation}
\end{definition}

Тогда на языке аксиоматической теории чисел (\FA) данная теорема записывается так\footnote{здесь и далее, запись $x \cdot y$ равносильна записи $xy$; а так же при переходе к языку \FA{} будем переименовывать большинство переменных и констант, чтобы соответствовать принятым обозначениям}:
\begin{theorem*}
    \begin{equation} \label{thm}
        \forall x \forall y \forall p \forall q (x = py + q \implies \forall z ((z \div x \land z \div y \implies z \div q) \land (z \div y \land z \div q \implies z \div x)))
    \end{equation}
\end{theorem*}

Для доказательства нам потребуется следующая теорема:
\begin{theorem}[\textbf{2, b, \S 1}] \label{auxthm}
    Если в равенстве вида $k + l + \dotsb + n = p + q + \dotsb + s$ относительно всех членов, кроме какого-либо одного, известно, что они кратны $b$, то и этот один член кратен $b$.

    Или на языке \FA:
    \begin{align}
        \forall w \forall x_1 \dots \forall x_n \forall y_1 \dots \forall y_m \forall z ( & z \mid x_1 \land z \mid x_2 \land \dotsb \land z \mid x_n \land \notag \\
                                                                                          & z \mid y_1 \land z \mid y_2 \land \dotsb \land z \mid y_m \land \notag \\
                                                                                          & x_1 + \dotsb + x_n + w = y_1 + \dotsb + y_m \rightarrow \notag         \\
                                                                                          & z \mid w)
    \end{align}
\end{theorem}
\begin{proof}
    Примем без доказательства.
\end{proof}

Приступим к доказательству теоремы:
\begin{proof}
    Запишем теорему \ref{auxthm} в удобном для нас виде:
    \begin{equation}
        \forall w \forall x \forall y \forall z (z \div x \land z \div y \land x = y + w \implies z \mid w)
    \end{equation}
    Дерево вывода см. далее.
    Для правил с кванторами условие на то, что переменная или терм свободна для подстановки, считаем выполненным.

    Примечания к дереву:
    \begin{description}
        \item[(*)] $u, v, w, r$ не входят свободно в заключение правила;
        \item[(**)] $s$ не входит свободно в заключение правила, т.е. $s \not\eqcirc u, v, w, r$.
    \end{description}

    Аксиомы:
    \begin{enumerate}
        \item при $t_4 \eqcirc s$ и $t_2 \eqcirc u$ или $t_4 \eqcirc s$ и $t_2 \eqcirc v$;
        \item при $t_4 \eqcirc s$ и $t_3 \eqcirc u$ или $t_4 \eqcirc s$ и $t_3 \eqcirc v$;
        \item при $t_2 \eqcirc u$, $t_3 \eqcirc wv$ и $t_1 \eqcirc r$;
        \item при $t_4 \eqcirc s$ и $t_1 \eqcirc r$;
        \item при $t_4 \eqcirc s$ и $t_2 \eqcirc v$ или $t_4 \eqcirc s$ и $t_2 \eqcirc r$;
        \item при $t_4 \eqcirc s$ и $t_3 \eqcirc v$ или $t_4 \eqcirc s$ и $t_3 \eqcirc r$;
        \item при $t_2 \eqcirc u$, $t_3 \eqcirc wv$ и $t_1 \eqcirc r$;
        \item при $t_4 \eqcirc s$ и $t_1 \eqcirc u$;
    \end{enumerate}
\end{proof}

\end{document}
