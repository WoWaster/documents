\documentclass[10pt, a4paper, oneside]{memoir}
%%% Обязательные пакеты
%% Beamer
\usepackage{beamerthemesplit}
\usetheme{SPbGU}
\beamertemplatenavigationsymbolsempty
\usepackage{appendixnumberbeamer}

%% Локализация
\usepackage{fontspec}
\setmainfont{CMU Serif}
\setsansfont{CMU Sans Serif}
\setmonofont{CMU Typewriter Text}
%\setmonofont{Fira Code}[Contextuals=Alternate,Scale=0.9]
%\setmonofont{Inconsolata}
% \newfontfamily\cyrillicfont{CMU Serif}

\usepackage{polyglossia}
\setdefaultlanguage{russian}
\setotherlanguage{english}
\usepackage[autostyle]{csquotes} % Правильные кавычки в зависимости от языка

%% Графика
\usepackage{wrapfig} % Позволяет вставлять графику, обтекаемую текстом
\usepackage{pdfpages} % Позволяет вставлять многостраничные pdf документы в текст

%% Математика
\usepackage{amsmath, amsfonts, amssymb, amsthm, mathtools} % "Адекватная" работа с математикой в LaTeX

% Математические окружения с русским названием
\newtheorem{rutheorem}{Теорема}
\newtheorem{ruproof}{Доказательство}
\newtheorem{rudefinition}{Определение}
\newtheorem{rulemma}{Лемма}

%%% Дополнительные пакеты. Используются в презентации, но могут быть отключены при необходимости
\usepackage{tikz} % Мощный пакет для создание рисунков, однако может очень сильно замедлять компиляцию
\usetikzlibrary{decorations.pathreplacing,calc,shapes,positioning,tikzmark}

\usepackage{multirow} % Ячейка занимающая несколько строк в таблице

%% Пакеты для оформления алгоритмов на псевдокоде
\usepackage[noend]{algpseudocode}
\usepackage{algorithm}
\usepackage{algorithmicx}

\usepackage{fancyvrb}

\NewDocumentCommand{\xxHash}{}{\textsc{xxHash}}
\NewDocumentCommand{\riscv}{}{\textsc{RISC-V}}
\NewDocumentCommand{\xxh}{m}{\textsc{XXH{#1}}}
\NewDocumentCommand{\sew}{}{\textsc{SEW}}
\NewDocumentCommand{\vl}{}{\textsc{VL}}
\NewDocumentCommand{\rvv}{}{\textsc{RVV}}
\usepackage{booktabs}
\usepackage{tabularx}
\usepackage{siunitx} % для таблиц с единицами измерений

%%% Memoir settings
\chapterstyle{ger}
\setlength{\headheight}{2\baselineskip}

\title{Доказательство теоремы \textbf{2, b, \S 2}}
\author{Пономарев Николай, 244 группа}
\date{}

\usepackage{pdflscape}

\semiisopage
\begin{document}

\maketitle
Необходимо доказать следующую теорему с помощью средств аксиоматической теории чисел:
\begin{theorem*}
    Если $a = bq + c$, то совокупность общих делителей чисел $a$ и $b$ совпадает с совокупностью общих делителей чисел $b$ и $c$; в частности, $(a, b) = (b, c)$.
\end{theorem*}

Введем предикатный символ $\mid$:
\begin{definition}[делимость]
    \begin{equation}
        y \mid x \Leftrightarrow \exists z (x = yz)
    \end{equation}
\end{definition}

Тогда на языке аксиоматической теории чисел (\FA) данная теорема записывается так\footnote{здесь и далее, запись $x \cdot y$ равносильна записи $xy$; а так же при переходе к языку \FA{} будем переименовывать большинство переменных и констант, чтобы соответствовать принятым обозначениям}:
\begin{theorem*}
    \begin{equation} \label{thm}
        \forall x \forall y \forall p \forall q (x = py + q \implies \forall z ((z \div x \land z \div y \implies z \div q) \land (z \div y \land z \div q \implies z \div x)))
    \end{equation}
\end{theorem*}

Для доказательства нам потребуется следующая теорема:
\begin{theorem}[\textbf{2, b, \S 1}] \label{auxthm}
    Если в равенстве вида $k + l + \dotsb + n = p + q + \dotsb + s$ относительно всех членов, кроме какого-либо одного, известно, что они кратны $b$, то и этот один член кратен $b$.

    Или на языке \FA:
    \begin{align}
        \forall w \forall x_1 \dots \forall x_n \forall y_1 \dots \forall y_m \forall z ( & z \mid x_1 \land z \mid x_2 \land \dotsb \land z \mid x_n \land \notag \\
                                                                                          & z \mid y_1 \land z \mid y_2 \land \dotsb \land z \mid y_m \land \notag \\
                                                                                          & x_1 + \dotsb + x_n + w = y_1 + \dotsb + y_m \rightarrow \notag         \\
                                                                                          & z \mid w)
    \end{align}
\end{theorem}
\begin{proof}
    Примем без доказательства.
\end{proof}

Приступим к доказательству теоремы:
\begin{proof}
    Запишем теорему \ref{auxthm} в удобном для нас виде:
    \begin{equation}
        \forall w \forall x \forall y \forall z (z \div x \land z \div y \land x = y + w \implies z \mid w)
    \end{equation}
    Дерево вывода см. далее.
    Для правил с кванторами условие на то, что переменная или терм свободна для подстановки, считаем выполненным.

    Примечания к дереву:
    \begin{description}
        \item[(*)] $u, v, w, r$ не входят свободно в заключение правила;
        \item[(**)] $s$ не входит свободно в заключение правила, т.е. $s \not\eqcirc u, v, w, r$.
    \end{description}

    Аксиомы:
    \begin{enumerate}
        \item при $t_4 \eqcirc s$ и $t_2 \eqcirc u$ или $t_4 \eqcirc s$ и $t_2 \eqcirc v$;
        \item при $t_4 \eqcirc s$ и $t_3 \eqcirc u$ или $t_4 \eqcirc s$ и $t_3 \eqcirc v$;
        \item при $t_2 \eqcirc u$, $t_3 \eqcirc wv$ и $t_1 \eqcirc r$;
        \item при $t_4 \eqcirc s$ и $t_1 \eqcirc r$;
        \item при $t_4 \eqcirc s$ и $t_2 \eqcirc v$ или $t_4 \eqcirc s$ и $t_2 \eqcirc r$;
        \item при $t_4 \eqcirc s$ и $t_3 \eqcirc v$ или $t_4 \eqcirc s$ и $t_3 \eqcirc r$;
        \item при $t_2 \eqcirc u$, $t_3 \eqcirc wv$ и $t_1 \eqcirc r$;
        \item при $t_4 \eqcirc s$ и $t_1 \eqcirc u$;
    \end{enumerate}
\end{proof}

\end{document}
