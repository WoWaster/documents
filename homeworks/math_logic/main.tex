\documentclass[12pt, a4paper, oneside]{memoir}
%%% PDF settings
\pdfvariable minorversion 7 % Set PDF version to 1.7.

%%% Fonts and language setup.
\usepackage{polyglossia}
% Setup fonts.
\usepackage{fontspec}
\setmainfont{CMU Serif}
\setsansfont{CMU Sans Serif}
\setmonofont{CMU Typewriter Text}

\usepackage{microtype} % Add fancy-schmancy font tricks

\usepackage{xcolor} % Add colors support.

%% Math
\usepackage{amsmath, amsfonts, amssymb, amsthm, mathtools} % Advanced math tools.
\usepackage{thmtools}
\usepackage{unicode-math} % Allow TTF and OTF fonts in math and allow direct typing unicode math characters.
\unimathsetup{
    warnings-off={
            mathtools-colon,
            mathtools-overbracket
        }
}
\setmathfont{Latin Modern Math} % default
\setmathfont[range={\setminus,\varnothing,\smashtimes}]{Asana Math}

%%% Images
\usepackage{graphicx}
\graphicspath{{figures/}}
\usepackage{import}

%%% Polyglossia setup after (nearly) everything as described in documentation.
\setdefaultlanguage{russian}
\setotherlanguage{english}

\usepackage{csquotes}

%%% Custom commands
\newcommand{\R}{\mathbb{R}}
\newcommand{\N}{\mathbb{N}}
\newcommand{\Z}{\mathbb{Z}}
\newcommand{\Q}{\mathbb{Q}}
\newcommand{\C}{\mathbb{C}}
\newcommand{\id}{\mathrm{id}}
\AtBeginDocument{\renewcommand{\leq}{\leqslant}}
\AtBeginDocument{\renewcommand{\geq}{\geqslant}}
\AtBeginDocument{\renewcommand{\Re}{\operatorname{Re}}}
\AtBeginDocument{\renewcommand{\Im}{\operatorname{Im}}}
\AtBeginDocument{\renewcommand{\phi}{\varphi}}
\AtBeginDocument{\renewcommand{\epsilon}{\varepsilon}}

%%% theorem-like envs
\theoremstyle{definition}

\declaretheoremstyle[spaceabove=0.5\topsep,
    spacebelow=0.5\topsep,
    headfont=\bfseries\sffamily,
    bodyfont=\normalfont,
    headpunct=.,
    postheadspace=5pt plus 1pt minus 1pt]{myStyle}
\declaretheoremstyle[spacebelow=\topsep,
    headfont=\bfseries\sffamily,
    bodyfont=\normalfont,
    headpunct=.,
    postheadspace=5pt plus 1pt minus 1pt,]{myStyleWithFrame}
\declaretheoremstyle[spacebelow=\topsep,
    headfont=\bfseries\sffamily,
    bodyfont=\normalfont,
    headpunct=.,
    postheadspace=5pt plus 1pt minus 1pt,
    qed=\blacksquare]{myProofStyleWithFrame}

\usepackage[breakable]{tcolorbox}
\tcbset{sharp corners=all, colback=white}
% \tcolorboxenvironment{theorem}{}
% \tcolorboxenvironment{theorem*}{}
% \tcolorboxenvironment{axiom}{}
% \tcolorboxenvironment{assertion}{}
% \tcolorboxenvironment{lemma}{}
% \tcolorboxenvironment{proposition}{}
% \tcolorboxenvironment{corollary}{}
% \tcolorboxenvironment{definition}{}
% \tcolorboxenvironment{proofReplace}{toprule=0mm,bottomrule=0mm,rightrule=0mm, colback=white, breakable }

\declaretheorem[name=Теорема, style=myStyleWithFrame]{theorem}
\declaretheorem[name=Теорема, numbered=no, style=myStyleWithFrame]{theorem*}
\declaretheorem[name=Аксиома, sibling=theorem, style=myStyleWithFrame]{axiom}
\declaretheorem[name=Преположение, sibling=theorem, style=myStyleWithFrame]{assertion}
\declaretheorem[name=Лемма, style=myStyleWithFrame]{lemma}
\declaretheorem[name=Предложение, sibling=theorem, style=myStyleWithFrame]{proposition}
\declaretheorem[name=Следствие, numberwithin=theorem, style=myStyleWithFrame]{corollary}

\declaretheorem[name=Определение, style=myStyleWithFrame]{definition}
\declaretheorem[name=Свойство, style=myStyle]{property}
\declaretheorem[name=Свойства, numbered=no, style=myStyle]{propertylist}

\declaretheorem[name=Пример, style=myStyle]{example}
\declaretheorem[name=Замечание, numbered=no, style=myStyle]{remark}

\declaretheorem[name=Доказательство, numbered=no, style=myProofStyleWithFrame]{proofReplace}
\renewenvironment{proof}[1][\proofname]{\begin{proofReplace}}{\end{proofReplace}}
% \declaretheorem[name=Доказательство, numbered=no, style=myProofStyleWithFrame]{longProof}

%%% Memoir settings
\chapterstyle{ger}
\setlength{\headheight}{2\baselineskip}

%%% HyperRef
\usepackage{hyperref}


\title{Доказательство теоремы \textbf{2, b, \S 2}}
\author{Пономарев Николай, 244 группа}
\date{}

\NewDocumentCommand{\FA}{}{\textbf{FA}}
\RenewDocumentCommand{\land}{}{\mathbin{\&}}

\usepackage{tikz}
\usetikzlibrary{trees}
\usepackage{pdflscape}
\usepackage{forest}

\semiisopage
\begin{document}

\maketitle
Необходимо доказать следующую теорему с помощью средств аксиоматической теории чисел:
\begin{theorem*}
    Если $a = bq + c$, то совокупность общих делителей чисел $a$ и $b$ совпадает с совокупностью общих делителей чисел $b$ и $c$; в частности, $(a, b) = (b, c)$.
\end{theorem*}

На языке аксиоматической теории чисел (\FA) данная теорема записывается так\footnote{здесь и далее, запись $x \cdot y$ равносильна записи $xy$; запись $(a, b)$ равносильна $\gcd(a, b)$.}:
\begin{theorem*}
    \begin{equation}
        \forall x \forall y \exists w \exists z ((x = yw + z) \rightarrow \gcd(x, y) = \gcd(y, z)) \label{thm}
    \end{equation}
\end{theorem*}

Введем функциональный символ $\mid$:
\begin{definition}[делимость]
    \begin{equation}
        y \mid x \Leftrightarrow \exists z (x = yz)
    \end{equation}
\end{definition}
а так же $\neq$:
\begin{definition}[неравенство]
    \[x \neq y \Leftrightarrow \lnot (x = y)\]
\end{definition}

Определим НОД:
\begin{definition}[НОД]
    \begin{equation} \label{gcd}
        \gcd(x, y) \Leftrightarrow \exists z (z \mid x \land z \mid y \land \forall w (w \mid x \land w \mid y \rightarrow w \mid z))
    \end{equation}
\end{definition}
Очевидным следствием из определения являются факты:
\begin{gather}
    \gcd(x, y) \mid x \label{gcd:cor:1} \\
    \gcd(x, y) \mid y \label{gcd:cor:2}
\end{gather}

Для доказательства нам потребуется следующая теорема:
\begin{theorem}[\textbf{2, b, \S 1}] \label{auxthm}
    Если в равенстве вида $k + l + \dotsb + n = p + q + \dotsb + s$ относительно всех членов, кроме какого-либо одного, известно, что они кратны $b$, то и этот один член кратен $b$.

    Или на языке \FA\footnote{здесь и далее, при переходе к языку \FA{} мы будем переименовывать большинство переменных и констант, чтобы соответствовать принятым обозначениям в методе резолюций.}:
    \begin{align}
        \forall z ( & z \mid x_1 \land z \mid x_2 \land \dotsb \land z \mid x_n \land \notag \\
                    & z \mid y_1 \land z \mid y_2 \land \dotsb \land z \mid y_m \land \notag \\
                    & x_1 + \dotsb + x_n + w = y_1 + \dotsb + y_m \rightarrow \notag         \\
                    & z \mid w)
    \end{align}
\end{theorem}
\begin{proof}
    Примем без доказательства.
\end{proof}

Теперь докажем вспомогательные леммы.
\begin{lemma}
    Если $a = bq+c$, то $\gcd(a, b) \mid c$.
    Или на языке \FA:
    \begin{equation}
        \forall x \forall y \exists w \exists z ((x = yw + z) \rightarrow \gcd(x, y) \mid z) \label{lemma1}
    \end{equation}
\end{lemma}
\begin{proof}
    Для начала приведем отрицание \eqref{lemma1} к множеству предложений:
    \begin{align*}
                          & \lnot \forall x \forall y \exists w \exists z ((x = yw + z) \rightarrow \gcd(x, y) \mid z) \\
        \Leftrightarrow{} & \lnot \forall x \forall y \exists w \exists z (\lnot (x = yw + z) \lor \gcd(x, y) \mid z)  \\
        \Leftrightarrow{} & \exists x \exists y \forall w \forall z ((x = yw + z) \land \lnot (\gcd(x, y) \mid z))     \\
        \Rightarrow {}    & \forall w \forall z ((a = bw + z) \land \lnot (\gcd(a, b) \mid z))                         \\
        \Rightarrow{}     & (a = bw + z) \land \lnot (\gcd(a, b) \mid z),
    \end{align*}
    здесь $a$ и $b$~--- сколемовские константы

    Теперь применим теорему \ref{auxthm} к $x = yq + c$ и $\gcd(x, y)$:
    \[(\gcd(x, y) \mid x \land \gcd(x, y) \mid yq \land x = yq + c) \rightarrow \gcd(x, y) \mid c\]
    Приведем это высказывание к множеству предложений:
    \begin{align*}
                          & (\gcd(x, y) \mid x \land \gcd(x, y) \mid yq \land x = yq + c) \rightarrow \gcd(x, y) \mid c             \\
        \Leftrightarrow{} & \lnot (\gcd(x, y) \mid x \land \gcd(x, y) \mid yq \land x = yq + c) \lor \gcd(x, y) \mid c              \\
        \Leftrightarrow{} & \lnot (\gcd(x, y) \mid x) \lor \lnot (\gcd(x, y) \mid yq) \lor (x \neq yq + c) \lor (\gcd(x, y) \mid c)
    \end{align*}

    И добавим к множеству предложений следствия из определения НОД:
    \begin{gather*}
        \gcd(u, v) \mid u\\
        \gcd(u, v) \mid v
    \end{gather*}

    Применение метода резолюций доказываем лемму (см. \nameref{res1}).
\end{proof}

\begin{lemma}
    Если $a = bq+c$, то $\gcd(b, c) \mid a$.
    Или на языке \FA:
    \begin{equation}
        \forall x \forall y \exists w \exists z ((x = yw + z) \rightarrow \gcd(y, z) \mid x) \label{lemma2}
    \end{equation}
\end{lemma}
\begin{proof}
    Доказательство практически полностью аналогично предыдущей лемме и поэтому опущено.
\end{proof}

\begin{lemma}
    Если $\gcd(x, y) \mid y$ и $\gcd(x, y) \mid z$, то $\gcd(x, y) \mid \gcd(y, z)$.
    Или на языке \FA:
    \begin{equation} \label{lemma3}
        \forall x \forall y \forall z((\gcd(x, y) \mid y) \land (\gcd(x, y) \mid z) \rightarrow (\gcd(x, y) \mid \gcd(y, z)))
    \end{equation}
\end{lemma}
\begin{proof}
    Приведем отрицание \eqref{lemma3} к множеству предложений:
    \begin{align*}
                          & \lnot \forall x \forall y \forall z((\gcd(x, y) \mid y) \land (\gcd(x, y) \mid z) \rightarrow (\gcd(x, y) \mid \gcd(y, z)))  \\
        \Leftrightarrow{} & \lnot \forall x \forall y \forall z(\lnot ((\gcd(x, y) \mid y) \land (\gcd(x, y) \mid z)) \lor (\gcd(x, y) \mid \gcd(y, z))) \\
        \Leftrightarrow{} & \exists x \exists y \exists z ((\gcd(x, y) \mid y) \land (\gcd(x, y) \mid z) \land \lnot (\gcd(x, y) \mid \gcd(y, z)))       \\
        \Rightarrow{}     & (\gcd(a, b) \mid b) \land (\gcd(a, b) \mid c) \land \lnot (\gcd(a, b) \mid \gcd(b, c)),
    \end{align*}
    где $a$, $b$ и $c$~--- сколемовские константы.

    Замкнем квантором всеобщности \eqref{gcd} и приведем его к множеству предложений:
    \begin{align*}
                          & \forall x \forall y \exists z (z \mid x \land z \mid y \land \forall w (w \mid x \land w \mid y \rightarrow w \mid z))                   \\
        \Leftrightarrow{} & \forall x \forall y \exists z (z \mid x \land z \mid y \land \forall w (\lnot (w \mid x) \lor \lnot (w \mid y) \lor (w \mid z)))         \\
        \Rightarrow{}     & \forall x \forall y \forall w (f(x, y) \mid x \land f(x, y) \mid y \land (\lnot (w \mid x) \lor \lnot (w \mid y) \lor (w \mid f(x, y)))) \\
        \Rightarrow{}     & f(x, y) \mid x \land f(x, y) \mid y \land (\lnot (w \mid x) \lor \lnot (w \mid y) \lor (w \mid f(x, y))),
    \end{align*}
    где $f(x,y)$~-- сколемовская функция.
    Применение метода резолюций доказываем лемму (см. \nameref{res2}\footnote{Если какие-то из предложений не используются в выводе, то они не выписываются.}).
\end{proof}

\begin{lemma}
    $\neq$ обладает транзитивностью.
    На языке \FA:
    \begin{equation}
        \forall x \forall y \forall z (x \neq y \land y \neq z \rightarrow x \neq z) \label{lemma4}
    \end{equation}
\end{lemma}
\begin{proof}
    Очевидно из определения.
\end{proof}

\begin{lemma}
    Если $a \mid b$ и $b \mid a$, то $a = b$. Или на языке \FA:
    \begin{equation}
        \forall x \forall y (x \mid y \land y \mid x \rightarrow x = y) \label{lemma5}
    \end{equation}
\end{lemma}

\begin{proof}
    Приведем отрицание \eqref{lemma5} к множеству предложений, раскрыв делимость по определению:
    \begin{align*}
                          & \lnot \forall x \forall y (x \mid y \land y \mid x \rightarrow x = y)                      \\
        \Leftrightarrow{} & \lnot \forall x \forall y (\exists w (y = xw) \land \exists z (x = yz) \rightarrow x = y)  \\
        \Leftrightarrow{} & \lnot \forall x \forall y (\lnot (\exists w (y = xw) \land \exists z (x = yz)) \lor x = y) \\
        \Leftrightarrow{} & \exists x \exists y (\exists w (y = xw) \land \exists z (x = yz) \land x \neq y)           \\
        \Rightarrow{}     & (b = aq) \land (a = bs) \land (a \neq b),
    \end{align*}
    где $a$, $b$, $q$, $s$~--- сколемовские константы.

    Запишем симметричность равенства
    \[\forall x \forall y (x = y \rightarrow y = x)\]
    теперь приведем ее к множеству предложений:
    \begin{align*}
                          & \forall x \forall y (x = y \rightarrow y = x) \\
        \Leftrightarrow{} & \forall x \forall y (x \neq y \lor y = x)     \\
        \Rightarrow{}     & x \neq y \lor y = x
    \end{align*}
    Приведем \eqref{lemma4} к множеству предложений:
    \begin{align*}
                          & \forall x \forall y \forall z (x \neq y \land y \neq z \rightarrow x \neq z) \\
        \Leftrightarrow{} & \forall x \forall y \forall z (x = y \lor y = z \lor x \neq z)               \\
        \Rightarrow{}     & x = y \lor y = z \lor x \neq z
    \end{align*}
    Так же запишем транзитивность равенства
    \[\forall x \forall y \forall z (x = y \land y = z \rightarrow x = z)\]
    и приведем к множеству предложений
    \begin{align*}
                          & \forall x \forall y \forall z (x = y \land y = z \rightarrow x = z) \\
        \Leftrightarrow{} & \forall x \forall y \forall z (x \neq y \lor y \neq z \lor x = z)   \\
        \Rightarrow{}     & x \neq y \lor y \neq z \lor x = z
    \end{align*}
    И с помощью метода резолюций докажем лемму (см. \nameref{res3}).
\end{proof}

\begin{lemma}
    \[\forall x \forall y (\gcd(x, y) = \gcd(y, x))\]
\end{lemma}
\begin{proof}
    Доказательство очевидно, т.к. определение $\gcd$ симметрично относительно переменных $x$ и $y$.
\end{proof}
Далее позволим себе отождествлять $\gcd(x, y)$ и $\gcd(y, x)$.

Теперь можем доказать теорему:
\begin{proof}
    Во всем доказательстве константы $a$, $b$, $q$, $c$, для удобства, подразумеваются константами из условия теоремы.

    Для начала приведем условие теоремы \eqref{thm} к множеству предложений:
    \begin{align*}
                          & \lnot \forall x \forall y \exists w \exists z ((x = yw + z) \rightarrow \gcd(x, y) = \gcd(y, z)) \\
        \Leftrightarrow{} & \lnot \forall x \forall y \exists w \exists z (\lnot (x = yw + z) \lor \gcd(x, y) = \gcd(y, z))  \\
        \Leftrightarrow{} & \exists x \exists y \forall w \forall z ((x = yw + z) \land \gcd(x, y) \neq \gcd(y, z))          \\
        \Rightarrow{}     & \forall w \forall z ((a = bw + z) \land \gcd(a, b) \neq \gcd(b, z))                              \\
        \Rightarrow{}     & (a = bw + z) \land \gcd(a, b) \neq \gcd(b, z),
    \end{align*}
    где $a$, $b$~--- сколемовские константы.
    Однако здесь мы можем сразу подставить $w$ и $z$ и получить
    \[(a = bq + c) \land \gcd(a, b) \neq \gcd(b, c)\]

    Приведем условия лемм 1, 2, 3, 5 к множествам предложений:
    для леммы 1:
    \[(x \neq yq + c) \lor \gcd(x, y) \mid c\]
    для леммы 2:
    \[(x \neq yq + c) \lor \gcd(y, c) \mid x\]
    для леммы 3:
    \[\lnot (\gcd(x, y) \mid y) \lor \lnot (\gcd(x, y) \mid z) \lor (\gcd(x, y) \mid \gcd(y, z))\]
    для леммы 5:
    \[\lnot (x \mid y) \lor \lnot (y \mid x) \lor x = y\]
    Кроме того, будем использовать \eqref{gcd:cor:1} и \eqref{gcd:cor:2}.
    Применим метод резолюций, чтобы доказать теорему (см. \nameref{res4}).
\end{proof}

\begin{landscape}
    \subsection{Приложение 1}\label{res1}
    \begin{forest}
        for tree={
        math content,
        grow=north,
        l=2cm
        }
        [nill
        [{\lnot (\gcd(a, b) \mid z)}, edge label={node[midway,right] {$|^{z}_{c}$}}]
        [{(\gcd(a, b) \mid c)}
        [{a = bw + z}]
        [{(x \neq yq + c) \lor (\gcd(x, y) \mid c)}, edge label={node[midway,right] {$|^{x\ y}_{a\ b}$}}
        [{\gcd(u, v) \mid v}, edge label={node[midway,right] {$|^{u\ v}_{x\ yq}$}}]
        [{\lnot (\gcd(x, y) \mid yq) \lor (x \neq yq + c) \lor (\gcd(x, y) \mid c)}
        [{\gcd(u, v) \mid u}, , edge label={node[midway,right] {$|^{u\ v}_{x\ y}$}}]
        [{\lnot (\gcd(x, y) \mid x) \lor \lnot (\gcd(x, y) \mid yq) \lor (x \neq yq + c) \lor (\gcd(x, y) \mid c)}]
        ]]
        ]
        ]
    \end{forest}

    \subsection{Приложение 2}\label{res2}
    \begin{forest}
        for tree={
        math content,
        grow=north,
        l=2cm
        }
        [nill
        [{\lnot (\gcd(a, b) \mid \gcd(b, c))}]
        [{\gcd(a, b) \mid f(b, c)}, edge label={node[midway,right] {$|^{f}_{\gcd}$}}
        [{\gcd(a, b) \mid c}]
        [{\lnot (\gcd(a, b) \mid y) \lor (\gcd(a, b) \mid f(b, y))}, edge label={node[midway,right] {$|^{y}_{c}$}}
        [{\gcd(a, b) \mid b}]
        [{\lnot (w \mid x) \lor \lnot (w \mid y) \lor (w \mid f(x, y))}, edge label={node[midway,right] {$|^{x\ w}_{b\ \gcd(a, b)}$}}]
        ]
        ]
        ]
    \end{forest}
\end{landscape}

\subsection{Приложение 3}\label{res3}
\begin{forest}
    for tree={
    math content,
    grow=north,
    l=2cm
    }
    [nill
    [{a = bs}]
    [{a \neq bs}
    [{x \neq y \lor y = x}, edge label={node[midway,right] {$|^{x\ y}_{a\ bs}$}}]
    [{bs \neq a}
    [{a \neq b}]
    [{bs \neq z \lor z = b}, , edge label={node[midway,right] {$|^{z}_{a}$}}
    [{x \neq y \lor y = x}, edge label={node[midway,right] {$|^{x\ y}_{b\ z}$}}]
    [{b = z \lor bs \neq z}
    [{x = y \lor y = z \lor x \neq z}, edge label={node[midway,right] {$|^{x\ \ y}_{bs\ b}$}}]
    [{bs \neq b}
    [{a \neq b}]
    [{bs \neq z \lor a = z}, edge label={node[midway,right] {$|^{z}_{b}$}}
    [{a = bs}]
    [{x \neq y \lor y \neq z \lor x = z}, edge label={node[midway,right] {$|^{x\ y}_{a\ bs}$}}]
    ]
    ]
    ]
    ]
    ]
    ]
    ]
\end{forest}

\begin{landscape}
    \subsection{Приложение 4}\label{res4}
    \begin{forest}
        for tree={
        math content,
        grow=north,
        l=2cm,
        scale=0.7
        }
        [nill
        [{\gcd(c, b) \mid \gcd(b, a)}
        [{\gcd(b, c) \mid b}]
        [{\lnot (\gcd(c, b) \mid b) \lor (\gcd(c, b) \mid \gcd(b, a))}
        [{\gcd(b, c) \mid a}
        [{(a = bq + c)}]
        [{(x \neq yq + c) \lor \gcd(y, c) \mid x}, edge label={node[midway,right] {$|^{x\ y}_{a\ b}$}}]
        ]
        [{\lnot (\gcd(x, y) \mid y) \lor \lnot (\gcd(x, y) \mid z) \lor (\gcd(x, y) \mid \gcd(y, z))}, edge label={node[midway,right] {$|^{x\ y\ z}_{c\ b\ a}$}}]
        ]
        ]
        [{\lnot (\gcd(b, c) \mid \gcd(a, b))}
        [{\gcd(a, b) \mid \gcd(b, c)}
        [{(\gcd(a, b) \mid b)}]
        [{\lnot (\gcd(a, b) \mid b) \lor (\gcd(a, b) \mid \gcd(b, c))}
        [{\lnot (\gcd(x, y) \mid y) \lor \lnot (\gcd(x, y) \mid z) \lor (\gcd(x, y) \mid \gcd(y, z))}, edge label={node[midway,right] {$|^{x\ y\ z}_{a\ b\ c}$}}]
        [{\gcd(a, b) \mid c}
        [{(a = bq + c)}]
        [{(x \neq yq + c) \lor \gcd(x, y) \mid c}, edge label={node[midway,right] {$|^{x\ y}_{a\ b}$}}]
        ]
        ]
        ]
        [{\lnot (\gcd(a, b) \mid \gcd(b, c)) \lor \lnot (\gcd(b, c) \mid \gcd(a, b))}
        [{\lnot (x \mid y) \lor \lnot (y \mid x) \lor x = y}, edge label={node[midway,right] {$|^{x\ y}_{\gcd(a, b)\ \gcd(b, c)}$}}]
        [{\gcd(a, b) \neq \gcd(b, c)}]
        ]
        ]
        ]
    \end{forest}
\end{landscape}



\end{document}
