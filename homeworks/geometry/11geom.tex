\documentclass[12pt, oneside]{memoir}
%%% PDF settings
\pdfvariable minorversion 7 % Set PDF version to 1.7.

%%% Fonts and language setup.
\usepackage{polyglossia}
% Setup fonts.
\usepackage{fontspec}
\setmainfont{CMU Serif}
\setsansfont{CMU Sans Serif}
\setmonofont{CMU Typewriter Text}

\usepackage{microtype} % Add fancy-schmancy font tricks

\usepackage{xcolor} % Add colors support.

%% Math
\usepackage{amsmath, amsfonts, amssymb, amsthm, mathtools} % Advanced math tools.
\usepackage{thmtools}
\usepackage{unicode-math} % Allow TTF and OTF fonts in math and allow direct typing unicode math characters.
\unimathsetup{
    warnings-off={
            mathtools-colon,
            mathtools-overbracket
        }
}
\setmathfont{Latin Modern Math} % default
\setmathfont[range={\setminus,\varnothing,\smashtimes}]{Asana Math}

%%% Images
\usepackage{graphicx}
\graphicspath{{figures/}}
\usepackage{import}

%%% Polyglossia setup after (nearly) everything as described in documentation.
\setdefaultlanguage{russian}
\setotherlanguage{english}

\usepackage{csquotes}

%%% Custom commands
\newcommand{\R}{\mathbb{R}}
\newcommand{\N}{\mathbb{N}}
\newcommand{\Z}{\mathbb{Z}}
\newcommand{\Q}{\mathbb{Q}}
\newcommand{\C}{\mathbb{C}}
\newcommand{\id}{\mathrm{id}}
\AtBeginDocument{\renewcommand{\leq}{\leqslant}}
\AtBeginDocument{\renewcommand{\geq}{\geqslant}}
\AtBeginDocument{\renewcommand{\Re}{\operatorname{Re}}}
\AtBeginDocument{\renewcommand{\Im}{\operatorname{Im}}}
\AtBeginDocument{\renewcommand{\phi}{\varphi}}
\AtBeginDocument{\renewcommand{\epsilon}{\varepsilon}}

%%% theorem-like envs
\theoremstyle{definition}

\declaretheoremstyle[spaceabove=0.5\topsep,
    spacebelow=0.5\topsep,
    headfont=\bfseries\sffamily,
    bodyfont=\normalfont,
    headpunct=.,
    postheadspace=5pt plus 1pt minus 1pt]{myStyle}
\declaretheoremstyle[spacebelow=\topsep,
    headfont=\bfseries\sffamily,
    bodyfont=\normalfont,
    headpunct=.,
    postheadspace=5pt plus 1pt minus 1pt,]{myStyleWithFrame}
\declaretheoremstyle[spacebelow=\topsep,
    headfont=\bfseries\sffamily,
    bodyfont=\normalfont,
    headpunct=.,
    postheadspace=5pt plus 1pt minus 1pt,
    qed=\blacksquare]{myProofStyleWithFrame}

\usepackage[breakable]{tcolorbox}
\tcbset{sharp corners=all, colback=white}
% \tcolorboxenvironment{theorem}{}
% \tcolorboxenvironment{theorem*}{}
% \tcolorboxenvironment{axiom}{}
% \tcolorboxenvironment{assertion}{}
% \tcolorboxenvironment{lemma}{}
% \tcolorboxenvironment{proposition}{}
% \tcolorboxenvironment{corollary}{}
% \tcolorboxenvironment{definition}{}
% \tcolorboxenvironment{proofReplace}{toprule=0mm,bottomrule=0mm,rightrule=0mm, colback=white, breakable }

\declaretheorem[name=Теорема, style=myStyleWithFrame]{theorem}
\declaretheorem[name=Теорема, numbered=no, style=myStyleWithFrame]{theorem*}
\declaretheorem[name=Аксиома, sibling=theorem, style=myStyleWithFrame]{axiom}
\declaretheorem[name=Преположение, sibling=theorem, style=myStyleWithFrame]{assertion}
\declaretheorem[name=Лемма, style=myStyleWithFrame]{lemma}
\declaretheorem[name=Предложение, sibling=theorem, style=myStyleWithFrame]{proposition}
\declaretheorem[name=Следствие, numberwithin=theorem, style=myStyleWithFrame]{corollary}

\declaretheorem[name=Определение, style=myStyleWithFrame]{definition}
\declaretheorem[name=Свойство, style=myStyle]{property}
\declaretheorem[name=Свойства, numbered=no, style=myStyle]{propertylist}

\declaretheorem[name=Пример, style=myStyle]{example}
\declaretheorem[name=Замечание, numbered=no, style=myStyle]{remark}

\declaretheorem[name=Доказательство, numbered=no, style=myProofStyleWithFrame]{proofReplace}
\renewenvironment{proof}[1][\proofname]{\begin{proofReplace}}{\end{proofReplace}}
% \declaretheorem[name=Доказательство, numbered=no, style=myProofStyleWithFrame]{longProof}

%%% Memoir settings
\chapterstyle{ger}
\setlength{\headheight}{2\baselineskip}

%%% HyperRef
\usepackage{hyperref}

\begin{document}
Пономарев, 144
\section*{Задача 1}
Дано:
\[\alpha: 2x-y-3z+14=0 \qquad A(1,2,3) \qquad B(1,-3,0)\]
Выяснить лежат ли точки с одной стороны плоскости.

Для этого необходимо подставить координаты точек в левую часть уравнение плоскости,
если полученное число 0, то точка принадлежит плоскости,
если число $>0$, то точка лежит по одну стороны плоскости,
если $<0$, то точка лежит по другую сторону плоскости.

Подставим $A$:
\[2 \cdot 1 - 1 \cdot 2 - 3 \cdot 3 + 14 = 2 - 2 - 9 + 14 = 5\]

Подставим $B$:
\[2 \cdot 1 - 1 \cdot (-3) - 3 \cdot 0 + 14 = 2 + 3 + 14 = 19\]

В обоих случаях числа одного знака и $\neq 0$, значит точки лежат по одну сторону плоскости.

\section*{Задача 2}
Даны точки $A, B, C$, и векторы $\va = \overrightarrow{AB}$ и $\vb = \overrightarrow{CB}$.
Доказать, что \[\dist(AB; C) = \left| \frac{(\va \times \vb) \times \va}{|\va|^2} \right| = d\]

\noindent\begin{minipage}{0.45\textwidth}
    \begin{tikzpicture}
        \node[dot, label=$A$] at (3,0) {};
        \node[dot, label=$B$] at (0,0) {};
        \node[dot, label=$C$] at (2,2) {};
        \node[dot, label=$D$] at (5,2) {};
        \draw[-Latex] (3, 0) --node[above] {$\va$} (0,0);
        \draw[-Latex] (2,2) --node[above, sloped] {$\vb$} (0,0);
        \draw (2,2) -- (5,2) -- (3,0);
        \draw (2,2) --node[right] {$d$} (2,0);
    \end{tikzpicture}
\end{minipage}
\begin{minipage}{0.45\textwidth}
    \begin{equation}\label{one}
        \begin{gathered}
            S = |\va \times \vb| = |\va| d\\
            d =  \left| \frac{(\va \times \vb)}{|\va|} \right|
        \end{gathered}
    \end{equation}
\end{minipage}

\begin{equation}
    \begin{gathered} \label{two}
        |\va\times\vb| = |a||b|\sin \alpha \qquad
        d = \left| \frac{(\va \times \vb) | \va| \sin\alpha}{|\va|^2} \right|\\
        \text{Но $(\va \times \vb) \perp \va$}\qquad
        d = \left| \frac{(\va \times \vb)}{|\va|} \right|
    \end{gathered}
\end{equation}
\eqref{one} = \eqref{two} \blacksquare

\section*{Задача 3}
Дано:
\[A(1,4,-2) \qquad l: \frac{x-4}{2} = \frac{y+1}{7} = \frac{z+2}{-3} \qquad \gamma: 2x+y+z=0\]
Найти $\alpha$, т.ч. $A \in \alpha, \alpha \parallel l, \alpha \perp \gamma$.

Пусть $\alpha: Ax + By + Cz + D = 0$.

\begin{gather*}
    \begin{cases}
        A \in \alpha \implies A + 4B - 2C + D = 0   \\
        \alpha \parallel l \implies 2A + 7B - 3C =0 \\
        \alpha \perp \gamma \implies 2A + B + C = 0
    \end{cases}
    \intertext{Решим эту систему:}
    \begin{cases}
        C = -2A -B               \\
        A + 4B + 4A + 2B + D = 0 \\
        2A + 7B + 6A + 3B = 0
    \end{cases} \Leftrightarrow
    \begin{cases}
        C = -2A -B      \\
        5A + 6B + D = 0 \\
        8A + 10B = 0    \\
    \end{cases}\\
    \begin{cases}
        A = - \frac{5}{4}B           \\
        C = 2 \cdot \frac{5}{4}B - B \\
        D =  5 \cdot \frac{5}{4}B - 6B
    \end{cases}\Leftrightarrow
    \begin{cases}
        A = - \frac{5}{4}B                \\
        C = \frac{5}{2} B - \frac{2}{2} B \\
        D = \frac{25}{4}B - \frac{24}{4}B
    \end{cases}
    \begin{cases}
        A = - \frac{5}{4}B \\
        C = \frac{3}{2}B   \\
        D = \frac{1}{4}B
    \end{cases}
    \intertext{Для удобства возьмем $B = 4$}
    \alpha: -5x + 4y + 6z + 1 = 0
\end{gather*}

\section*{Задача 4}
Дано:
\[\beta: x - 3y + z + 5 = 0 \qquad \gamma: -2x + z - 2 = 0 \qquad \delta: x - y + 2 = 0\]
Надо: построить $\alpha$, через прямую пересечения $\beta$ и $\gamma$, т.ч. $\alpha \perp \delta$

Найдем прямую пересечения плоскостей:
\begin{gather*}
    \begin{cases}
        x - 3y + z + 5 = 0 \\
        -2x + z - 2 = 0
    \end{cases} \Leftrightarrow
    \begin{cases}
        z = 2x + 2             \\
        x - 3y +2x + 2 + 5 = 0 \\
    \end{cases}\\
    \begin{cases}
        z = 2x + 2 \\
        3x - 3y + 7 = 0
    \end{cases} \Leftrightarrow
    \begin{cases}
        z = 2x + 2 \\
        y = x + \frac{7}{3}
    \end{cases} \Leftrightarrow
    \begin{cases}
        x = \frac{z-2}{2} \\
        x = y - \frac{7}{3}
    \end{cases}\\
    \Leftrightarrow \frac{x}{1} = \frac{y - \frac{7}{3}}{1} = \frac{z-2}{2} : l
\end{gather*}

Найдем направляющий вектор $l$ и вектор нормали $\delta$:
\[\vv_l = \{1,1,2\} \qquad \vn_\delta = \{1, -1, 0\}\]

Их векторное произведение будет вектором нормали $\alpha$, найдем его:
\[
    \vv_l \times \vn_\delta = \begin{vmatrix}
        \vi & \vj & \vk \\
        1   & 1   & 2   \\
        1   & - 1 & 0
    \end{vmatrix} = 2 \vi + 2\vj - 2 \vk = \{2, 2, -2\} \parallel \{1,1,-1\} = \vn_\alpha
\]

Тогда $\alpha: x + y - z + D = 0$, найдем $D$, для этого подставим в уравнение точку принадлежащую плоскости $\alpha$
и прямой $l$, например $(1;\frac{10}{3}; 4)$
\begin{gather*}
    1 + \frac{10}{3} - 4 + D = 0 \Leftrightarrow
    D = - \frac{1}{3}
    \intertext{Вернемся к $\alpha$}
    x + y - z - \frac{1}{3} = 0 \quad |\cdot 3\\
    \alpha: 3x+3y-3z-1=0
\end{gather*}

\section*{Задача 5}
Дано: \[A(-2,-1,3) \qquad l: \frac{x-4}{2} = \frac{y+1}{7} = \frac{z+2}{-3}\]
Найти $\dist(A, l)$

Найдем направляющий вектор и точку лежащую на прямой
\[\vv_l = \{2, 7, -3\} \qquad B(6,6,-5)\]
Тогда
\begin{gather*}
    \overrightarrow{AB} = \{8, 7, -8\}\\
    \overrightarrow{AB} \times \vv_l = \begin{vmatrix}
        \vi & \vj & \vk \\
        8   & 7   & -8  \\
        2   & 7   & -3
    \end{vmatrix} = 35 \vi + 8 \vj + 42 \vk = \{35, 8, 42\}\\
    \dist(A, l) = \frac{\left|\overrightarrow{AB} \times \vv_l \right|}{|\vv_l|} =
    \frac{\sqrt{35^2 + 8^2 + 42^2}}{\sqrt{2^2 + 7^2 + (-3)^2}} = \sqrt{\frac{3053}{62}}
\end{gather*}

\end{document}