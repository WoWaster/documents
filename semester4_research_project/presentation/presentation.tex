% \documentclass{beamer}
%Для защит онлайн лучше использовать разрешение 16x9
\documentclass[aspectratio=169]{beamer}

%%% Обязательные пакеты
%% Beamer
\usepackage{beamerthemesplit}
\usetheme{SPbGU}
\beamertemplatenavigationsymbolsempty
\usepackage{appendixnumberbeamer}

%% Локализация
\usepackage{fontspec}
\setmainfont{CMU Serif}
\setsansfont{CMU Sans Serif}
\setmonofont{CMU Typewriter Text}
%\setmonofont{Fira Code}[Contextuals=Alternate,Scale=0.9]
%\setmonofont{Inconsolata}
% \newfontfamily\cyrillicfont{CMU Serif}

\usepackage{polyglossia}
\setdefaultlanguage{russian}
\setotherlanguage{english}
\usepackage[autostyle]{csquotes} % Правильные кавычки в зависимости от языка

%% Графика
\usepackage{wrapfig} % Позволяет вставлять графику, обтекаемую текстом
\usepackage{pdfpages} % Позволяет вставлять многостраничные pdf документы в текст

%% Математика
\usepackage{amsmath, amsfonts, amssymb, amsthm, mathtools} % "Адекватная" работа с математикой в LaTeX

% Математические окружения с русским названием
\newtheorem{rutheorem}{Теорема}
\newtheorem{ruproof}{Доказательство}
\newtheorem{rudefinition}{Определение}
\newtheorem{rulemma}{Лемма}

%%% Дополнительные пакеты. Используются в презентации, но могут быть отключены при необходимости
\usepackage{tikz} % Мощный пакет для создание рисунков, однако может очень сильно замедлять компиляцию
\usetikzlibrary{decorations.pathreplacing,calc,shapes,positioning,tikzmark}

\usepackage{multirow} % Ячейка занимающая несколько строк в таблице

%% Пакеты для оформления алгоритмов на псевдокоде
\usepackage[noend]{algpseudocode}
\usepackage{algorithm}
\usepackage{algorithmicx}

\usepackage{fancyvrb}

\NewDocumentCommand{\xxHash}{}{\textsc{xxHash}}
\NewDocumentCommand{\riscv}{}{\textsc{RISC-V}}
\NewDocumentCommand{\xxh}{m}{\textsc{XXH{#1}}}
\NewDocumentCommand{\sew}{}{\textsc{SEW}}
\NewDocumentCommand{\vl}{}{\textsc{VL}}
\NewDocumentCommand{\rvv}{}{\textsc{RVV}}
\usepackage{booktabs}
\usepackage{tabularx}
\usepackage{siunitx} % для таблиц с единицами измерений


% То, что в квадратных скобках, отображается внизу по центру каждого слайда.
\title[Оптимизация \xxHash{}]{Оптимизация библиотеки \xxHash{} для архитектуры \riscv{}}

% То, что в квадратных скобках, отображается в левом нижнем углу.
\institute[СПбГУ]{}

% То, что в квадратных скобках, отображается в левом нижнем углу.
\author[Николай Пономарев]{Николай Алексеевич Пономарев, группа 21.Б10-мм}

\begin{document}
{
\setbeamertemplate{footline}{}
% Лого университета или организации, отображается в шапке титульного листа
\begin{frame}
	\includegraphics[width=1.4cm]{pictures/SPbGU_Logo.png}
	\vspace{-35pt}
	\hspace{-10pt}
	\begin{center}
		\begin{tabular}{c}
			\scriptsize{Санкт-Петербургский государственный университет} \\
			\scriptsize{Кафедра системного программирования}
		\end{tabular}
		\titlepage
	\end{center}

	\btVFill

	{\scriptsize
		% У научного руководителя должна быть указана научная степень
		\textbf{Научный руководитель:} К.~К.~Смирнов, ст. преподаватель кафедры ИАС \\
	}
	\begin{center}
		\vspace{5pt}
		\scriptsize{Санкт-Петербург\\
			2023}
	\end{center}

\end{frame}
}

\begin{frame}
	\frametitle{\xxHash{}}
	\xxHash{}~--- современная библиотека для высокопроизводительного хеширования

	Особенности:
	\begin{itemize}
		\item Поддерживает вычисление 32-, 64- и 128-битных хешей
		\item Скорость работы превышает пропускную способность оперативной памяти\footnote{\url{https://github.com/Cyan4973/xxHash/blob/v0.8.1/README.md}}
		\item Имеет поддержку векторных расширений процессора
		      \begin{itemize}
			      \item SSE2, AVX2, AVX512 (x86\_64)
			      \item NEON, SVE (ARM)
			      \item VSX (PowerPC)
			      \item Но нет поддержки \rvv{} (\riscv{})
		      \end{itemize}
	\end{itemize}
\end{frame}

\begin{frame}
	\frametitle{Постановка задачи}
	\textbf{Целью} работы является реализация хеш-функций из библиотеки \xxHash{} при помощи векторных расширений архитектуры \riscv{}.

	\textbf{Задачи}:
	\begin{itemize}
		\item Сравнить возможности разных версий векторного расширения \riscv{}
		\item Выбрать целевую платформу для адаптации кода и проведения измерений
		\item Адаптировать одну из существующих реализаций под выбранную платформу
		\item Выполнить замеры производительности адаптированного кода
	\end{itemize}
\end{frame}

%Идеально, если есть по одному слайду на каждую поставленную задачу
\begin{frame}
	\frametitle{Векторные возможности \riscv{}}

	Векторное расширение \riscv{}, сокращенно \rvv{}, в данный момент представлено двумя несовместимыми версиями:
	\begin{center}
		\begin{tabularx}{0.8\linewidth}{XX}
			\toprule
			\rvv{} 1.0 & \rvv{} 0.7.1 \\ \midrule
			Стабильная версия & Нестабильная версия \\ \midrule
			Поддерживается современными компиляторами & Требуется специализированный компилятор \\ \midrule
			Есть перегрузка intrinsic функций & Нет перегрузки intrinsic функций \\ \midrule
			Нет устройств в продаже\footnote{На момент написания} & Есть устройства в свободной продаже \\
			\bottomrule
		\end{tabularx}
	\end{center}
\end{frame}

\begin{frame}
	\frametitle{Целевая платформа}

	В качестве целевой платформы был выбран одноплатный ПК \textsc{Sipeed Lichee RV} на чипе \textsc{Allwinner D1}
	\begin{itemize}
		\item Единственная доступная на момент написания платформа с поддержкой \rvv{}
		\item Поддерживает \rvv{}~0.7.1
		\item Не поддерживает 64-битные элементы вектора $\implies$ используем 32-битные
	\end{itemize}

\end{frame}

\begin{frame}[t]
	\frametitle{Проблемы при адаптации}
	В качестве базовой была выбрана реализация для SSE2, она использует векторы с 64-битными элементами.
	Адаптированная реализация использует 32-битные элементы вектора.

	При реализации были встречены следующие проблемы:
	\begin{itemize}
		\item Сложности обработки 64-битных чисел, представленных как пары 32-битных
		      \begin{itemize}
			      \item Необходимость реализации умножения с ручным расширением
			      \item Необходимость ручной реализации сложения с переносом
		      \end{itemize}
		\item Сложности с работой с масками в \rvv{} 0.7.1
		      \begin{itemize}
			      \item Отсутствие операции загрузки маски из памяти, необходимость создания маски по уже загруженному вектору
		      \end{itemize}
		\item Работа с неопределенным поведением при загрузке по невыровненному адресу
	\end{itemize}
\end{frame}

\begin{frame}
	\frametitle{Экспериментальное исследование}
	Измерения проводились на одноплатном компьютере \textsc{Sipeed Lichee RV 86} со следующими характеристиками:
	\begin{itemize}
		\item Процессор \textsc{Allwinner D1} с частотой 1~ГГц;
		\item Оперативная память \textsc{DDR3} объемом 512~Мб с частотой 800~МГц;
		\item Операционная система \textsc{Debian Sid} с последними обновлениями на момент тестирования.
	\end{itemize}

	Для компиляции использовался компилятор от компании \textsc{Alibaba} с флагами \texttt{-O3} и \texttt{-march=rv64gcv0p7}.
	Для выбора набора функций использовались флаги \texttt{-DXXH\_VECTOR=XXH\_SCALAR} и \texttt{-DXXH\_VECTOR=XXH\_RVV} соответственно.
	Данные измерений были получены с помощью поставляемой вместе с библиотекой утилиты \texttt{xxhsum}.
\end{frame}

\begin{frame}
	\frametitle{Результаты экспериментального исследования}
	\begin{table}[h]
		\def\arraystretch{1.1}  % Растяжение строк в таблицах
		\setlength\tabcolsep{0.2em}
		\centering
		\caption{Сравнение производительности хеш-функции \xxh{3} на входных данных размером в 1000~Кб; числа приведены с относительной погрешностью 0.5\%}
		\begin{tabular}[C]{
				c
				*1{S
							[table-figures-uncertainty=2, separate-uncertainty=true, table-align-uncertainty=true,
								table-figures-integer=3, table-figures-decimal=2, round-precision=2,
								table-number-alignment=center]
					}
			}
			\toprule
			\multicolumn{1}{r}{Набор функций} & \multicolumn{1}{r}{Скорость работы, Мб/с} \\ \midrule
			Скалярный & 169.3 \\ \midrule
			Векторный & 116.0 \\
			\bottomrule
		\end{tabular}
		\label{tab:speed}
	\end{table}

	Вероятные причины замедления:
	\begin{itemize}
		\item Большое количество операций загрузки при вызове функций
		\item Использование инструкций объединения векторов и инструкций перестановки элементов вектора
	\end{itemize}
\end{frame}

\begin{frame}
	\frametitle{Результаты}
	\begin{itemize}
		\item Изучены возможности разных версий векторного расширения \riscv{}
		\item Выбрана целевая платформа для адаптации кода и проведения измерений
		\item Проведена адаптация одной из существующих реализаций под выбранную платформу
		\item Выполнены замеры производительности адаптированного кода
	\end{itemize}

	Исходный код расположен по адресу: \url{https://github.com/WoWaster/xxHash}.
\end{frame}

\end{document}
