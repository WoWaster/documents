% !TeX spellcheck = ru_RU
% !TEX root = vkr.tex

\section{Метод}
Реализация в широком смысле: что таки было сделано. Скорее всего самый большой раздел.

\emph{Крайне желательно} к Новому году иметь что-то, что сюда можно написать.

Для понимания того как курсовая записка (отзыв по учебной практи\-ке/ВКР) должна писаться, можно посмотреть видео ниже. Они про научные доклады и написание научных статей, учебные практики и ВКР отличаются тем, что тут есть требования отдельных глав (слайдов) со списком задач и списком результатов. Но даже если вы забьёте не требования специфичные для ВКР, и соблюдете все рекомендации ниже, получившиеся скорее всего будет лучше чем первая попытка 99\% ваших однокурсников.

\begin{enumerate}
	\item \enquote{Как сделать великолепный научный доклад} от Саймона Пейтона Джонса~\cite{SPJGreatTalk} (на английском).
	\item \enquote{Как написать великолепную научную статью} от Саймона Пейтона Джонса~\cite{SPJGreatPaper} (на английском).
	\item \enquote{Как писать статьи так, чтобы люди их смогли прочитать} от Дэрэка Драйера~\cite{DreyerYoutube2020} (на английском). По предыдующей ссылке разбираются, что должно быть в статье, т.е. как она должна выгля\-деть на высоком уровне. Тут более низкоуровнево изложено как должны быть устроены параграфы и т.п.
	\item Ещё можно посмотреть How to Design Talks~\cite{JhalaYoutube2020} от Ranjit Jhala, но я думаю, что первых трех ссылок всем хватит.
\end{enumerate}

\subsection{Некоторые типичные ошибки}
Здесь мы будем собирать основные ошибки, которые случаются при написании текстов.
В интернетах тоже можно найти коллекции типич\-ных косяков\footnote{\href{https://www.read.seas.harvard.edu/~kohler/latex.html}{https://www.read.seas.harvard.edu/\textasciitilde kohler/latex.html} (дата доступа:   \DTMdate{2022-12-16}).}.

Рекомендуется по-умол\-ча\-нию использовать красивые греческие бук\-вы $\sigma$  и $\phi$, а именно $\phi$ вместо $\varphi$. В данном шаблоне команды для этих букв переставлены местами по сравнению с ванильным \TeX'ом.

Также, если работа пишется на русском языке, необходимо, чтобы работа была написана на \textit{грамотном} русском языке даже, если автор, из ближнего зарубежья\footnote{
	Теоретически, возможен вариант написания текстов на английском языке, но это необходимо обсудить в первую очередь с научным руководителем.}.
Это включает в себя:
\begin{itemize}
	\item разделы должны оформляться с помощью \verb=\section{...}=, а также \verb=\subsection= и т.~п.; не нужно пытаться нумеровать вручную;
	\item точки после окончания предложений должны присутствовать;
	\item пробелы после запятых  и точек, в конце слова перед скобкой;
	\item неразрывные пробелы, там, где нужны пробелы, но переносить на другую строку нельзя, например \verb=т.~е.= или \verb=А.~Н.~Терехов=;
	\item дефис, там где нужен дефис (обозначается с помощью одиночного \enquote{минуса} (англ. dash) на клавиатуре);
	\item двойной дефис, там где он нужен; а именно  при указании проме\-жутка в цифрах: в 1900--1910 г.~г., стр. 150--154;
	\item тире (т.~е. \verb=---= --- тройной минус) на месте тире, а не что-то другое; в русском языке тире не может \enquote{съезжать} на новую строку, поэтому стоит использовать такой синтаксис: \verb=До~--- после=;
	\item правильные двойные кавычки должны набираться с помощью пакета \texttt{csquotes}: для основного языка в \texttt{polyglossia} стоит использовать команду \verb=\enquote{текст}=, для второго языка стоит использовать \verb=\foreignquote{язык}{текст}=;
	\item все перечисления должны оформляться с помощью \verb=\enumerate= или \verb=\itemize=; пункты перечислений должны либо начинаться с заглавной буквой и заканчиваться точкой, либо начинаться со строчной и закачиваться точкой с запятой; последний пункт пере\-числений всегда заканчивается точкой.
	\item Перед выкладкой финальной версии необходимо просмотреть лог \LaTeX'a, и обратить внимание на сообщения вида \emph{Overfull \textbackslash hbox (1.29312pt too wide) in paragraph}. Обычно, это означает, что текст выползает за поля, и надо подсказать, как правильно слова на слоги, чтобы перенос произощел автоматически. Это делается, например, так: \verb=соломо\-волокуша=.
\end{itemize}

\subsection{Листинги, картинка и прочий \enquote{не текст}}

Различный \enquote{не текст} имеет свойство отображаться не там, где он написан в текстовом в \LaTeX{}, поэтому у него должна быть самодостаточ\-ная понятная подпись \verb=\caption{...}=, уникальная метка \verb=\label{...}=, чтобы на неё можно было бы ссылаться в тексте с помощью \verb=\ref{...}=. Ниже вы сможете увидеть таблицу \ref{time_cmp_obj_func}, на которую мы сослались буквально только что.

\enquote{Не текста} может быть довольно много, чтобы не засорять корневую папку, хорошим решением будет складывать весь \enquote{не текст} в папку \texttt{figures}.
Заклинание \verb=\includegraphics{}= уже знает этот путь и будет искать там файлы без указания папки.
Команда \verb=\input{}= умеет ходит по путям, например \verb=\input{figures/my_awesome_table.tex}=.
Кроме того, листинги кода можно подтягивать из файла с помощью команды \verb=\lstinputlisting{file}=.
%% TODO: Проверить, что на Windows \input{folder/file} работает, если нет, использовать пакет import

%% Вставка кода с помощью listings
\begin{lstlisting}[caption={Название для листинга кода. Достаточно длинное, чтобы люди, которые смотрят картинку сразу после названия статьи (т.~е. все люди), смогли разобраться и понять к чему в статье листинги, картинки и прочий \enquote{не текст}.}, language=Caml, frame=single]
  let x = 5 in x+1
\end{lstlisting}
%% Вставка кода с помощью minted
% \begin{listing}
%   \caption{Название для листинга кода. Достаточно длинное, чтобы люди, которые смотрят картинку сразу после названия статьи (т.~е. все люди), смогли разобраться и понять к чему в статье листинги, картинки и прочий \enquote{не текст}.}
%   \begin{minted}[frame=single]{ocaml}
%     let x = 5 in x+1
%   \end{minted}
% \end{listing}

\subsubsection{Выделения куска листинга с помощью tikz}
Это обывает полезно в текста, а ещё чаще~--- в презентациях. Пример сделан на основе вопроса на \textsc{StackOveflow}\footnote{\url{https://tex.stackexchange.com/questions/284311} (дата доступа:   \DTMdate{2022-12-16}).}.

\begin{figure}
	\begin{lstlisting}[escapechar=!,basicstyle=\ttfamily, language=c]
#include <stdio.h>
#include <math.h>

int main(void)
{
  double c = -1;
  double z = 0;

  printf ("For c = %lf:\n", c);
  for (int i=0; i<10; i++ ) {
    printf ( !\tikzmark{a}!"z %d = %lf\n"!\tikzmark{b}!, i, z);
    z = pow(z, 2) + c;
  }
}
\end{lstlisting}

	\begin{tikzpicture}[use tikzmark]
		\draw[fill=gray,opacity=0.1]
		([shift={(-3pt,2ex)}]pic cs:a)
		rectangle
		([shift={(3pt,-0.65ex)}]pic cs:b);
	\end{tikzpicture}
	\caption{Пример листинга и \textsc{TIKZ} декорации к нему, оформленные в окружении \texttt{figure}. Обратите внимание, что рисунок отображается не там, где он в документе, а может \enquote{плавать}.}
\end{figure}
