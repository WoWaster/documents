% !TeX spellcheck = ru_RU
% !TEX root = vkr.tex

\section*{Введение}
\thispagestyle{withCompileDate}

\xxHash{}\footnote{\url{https://github.com/Cyan4973/xxHash}}~--- современная библиотека для хеширования, целью которой является генерация хеша со скоростью, сравнимой со скоростью оперативной памяти.
\riscv{}, в свою очередь, это молодая и активно развивающаяся архитектура, которая получила поддержку от таких компаний как \textsc{Google}~\cite{GoogleAnnouncesOfficial}, \textsc{Imagination Technologies}~\cite{WhyWeVe} и \textsc{Alibaba}~\cite{robinsonAlibabaLaunchesRISCV}.

Высокую скорость работы библиотеки, в частности, хешей \xxh{3} и \xxh{128}, обеспечивает реализация алгоритмов хеширования с помощью векторных расширений процессора.
Некоторые процессоры архитектуры \riscv{} имеют поддержку векторных инструкций.
К сожалению, долгое время спецификация векторного расширения находилась в разработке, что привело к наличию в продаже процессоров с разными версиями расширения~\cite{ImplicationsWidelyDistributed}.

Библиотека \xxHash{} ещё не содержит в себе оптимизаций с помощью векторного расширения для платформы \riscv{}~(сокращенно \rvv{}).
На момент написания данной работы в продаже можно было найти только процессоры с \enquote{урезанной} поддержкой \rvv{}.
Изучение вопроса о том, возможно ли ускорение алгоритмов при переносе на процессоры с \enquote{ограниченными} возможностями на примере библиотеки \xxHash{}, является целью данной работы.

В работе приводится изучение различий в версиях \rvv{}, разбор трудностей, встреченных при переносе на процессор с \enquote{ограниченными} возможностями, и исследование результатов измерения производительности.
