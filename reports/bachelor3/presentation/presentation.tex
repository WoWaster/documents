% \documentclass{beamer}
%Для защит онлайн лучше использовать разрешение 16x9
\documentclass[aspectratio=169]{beamer}

\usepackage{beamerthemesplit}
\usepackage{wrapfig}
\usetheme{SPbGU}
\usepackage{pdfpages}
\usepackage{amsmath}
\usepackage{cmap} 
\usepackage[T2A]{fontenc} 
\usepackage[utf8]{inputenc}
\usepackage[english,russian]{babel}
\usepackage{indentfirst}
\usepackage{amsmath}
\usepackage{tikz}
\usepackage{multirow}
\usepackage[noend]{algpseudocode}
\usepackage{algorithm}
\usepackage{algorithmicx}
\usetikzlibrary{shapes,arrows}
\usepackage{fancyvrb}
\usepackage{appendixnumberbeamer}
\usepackage{tabularx}
\usepackage{array}
\usepackage{booktabs}

\newtheorem{rutheorem}{Теорема}
\newtheorem{ruproof}{Доказательство}
\newtheorem{rudefinition}{Определение}
\newtheorem{rulemma}{Лемма}

\beamertemplatenavigationsymbolsempty

% То, что в квадратных скобках, отображается внизу по центру каждого слайда. 
\title[Оптимизация CRC--32]{Оптимизация алгоритма CRC--32 под RISC-V}

% То, что в квадратных скобках, отображается в левом нижнем углу. 
\institute[СПбГУ]{}

% То, что в квадратных скобках, отображается в левом нижнем углу.
\author[Николай Пономарев]{Николай Алексеевич Пономарев, группа 21.Б10-мм}
 
\begin{document}
{
\setbeamertemplate{footline}{}
% Лого университета или организации, отображается в шапке титульного листа
\begin{frame}
  \includegraphics[width=1.4cm]{pictures/SPbGU_Logo.png}
  \vspace{-35pt}
  \hspace{-10pt}
  \begin{center}
    \begin{tabular}{c}
      \scriptsize{Санкт-Петербургский государственный университет} \\
      \scriptsize{Кафедра системного программирования}
    \end{tabular}
    \titlepage
  \end{center}

  \btVFill

  {\scriptsize
    {\bfseries Научный руководитель:} ст. преподаватель кафедры ИАС К. К. Смирнов \\
  }
  \begin{center}
    \vspace{5pt}
    \scriptsize{Санкт-Петербург\\
      2022}
  \end{center}

\end{frame}
}

\begin{frame}
  \frametitle{Постановка задачи}
  \textbf{Целью} работы является реализация оптимизации CRC--32 с использованием аппаратных инструкций умножения многочленов на архитектуре RISC-V

  \textbf{Задачи}:
  \begin{itemize}
    \item Изучить варианты оптимизации алгоритма CRC--32
    \item Выбрать целевую платформу для проведения измерений с учетом необходимых расширений процессора
    \item Адаптировать одну из существующих реализаций под RISC-V
    \item Выполнить замеры производительности оптимизированного кода
  \end{itemize}
\end{frame}


\begin{frame}
  \frametitle{Алгоритм CRC--32 и существующие оптимизации}
  Алгоритм CRC~(Cyclic Redundancy Check) используется для проверки целостности сообщения при передаче данных.
  В данной работе речь идет о его варианте --- CRC-32.

  Известные способы оптимизации:
  \begin{itemize}
    \item Заранее вычисленные таблицы значений
    \item Использование аппаратных инструкций вычисления CRC-32
    \item Использование инструкций умножения многочленов над $\mathbb{F}_2$
  \end{itemize}
\end{frame}


\begin{frame}
  \frametitle{Необходимые возможности процессора}
  \begin{itemize}
    \item RISC-V обладает модульной архитектурой
    \item Инструкции для умножения многочленов над $\mathbb{F}_2$ содержатся в расширении B\footnote{Standard Extension for Bit Manipulation}
    \item Можем использовать только 64-битную базу
  \end{itemize}
\end{frame}


\begin{frame}
  \frametitle{Платформы для измерений}
  \begin{itemize}
    \item Нет ни одного доступного процессора с расширением B
    \item Необходим симулятор, были найдены:
          \begin{itemize}
            \item Spike
            \item gem5
          \end{itemize}
    \item Для измерений подходит только gem5
  \end{itemize}
\end{frame}

\begin{frame}
  \frametitle{Реализация}
  \begin{itemize}
    \item В качестве базовой реализации была выбрана реализация из ядра Linux
    \item Адаптированная реализация обрабатывает 128 бит за раз
  \end{itemize}
\end{frame}

\begin{frame}[t]
  \frametitle{Эксперимент}
  Эксперименты проводились на симуляторе gem5 со следующими характеристиками:
  \begin{itemize}
    \item MinorCPU с частотой 1~ГГц и размером кэша L1 в 64~Кб
    \item 512~Мб ОЗУ DDR4 с частотой 2400 МГц
  \end{itemize}

  Исходные файлы компилировались с флагами \texttt{-O3 -static}, а затем запускались на симуляторе.

  \begin{table}[h]
    \begin{center}
      \begin{tabularx}{0.95\linewidth}{*3{>{\raggedleft\arraybackslash}X}}
        \toprule
        Объем данных, байт & Стандартный алгоритм, тиков & Оптимизированный алгоритм, тиков \\ \midrule
        128                & $308.5 \cdot 10^3$          & $83\cdot 10^3$                   \\ \midrule
        1024               & $2277.5 \cdot 10^3$         & $424 \cdot 10^3$                 \\ \midrule
        8192               & $18 \cdot 10^6$             & $3.6 \cdot 10^6$                 \\ \midrule
        65536              & $191 \cdot 10^6$            & $30.5 \cdot 10^6$                \\
        \bottomrule
      \end{tabularx}
    \end{center}
    \caption{Результаты измерений}
  \end{table}
\end{frame}

\begin{frame}
  \frametitle{Результаты}
  \begin{itemize}
    \item Изучены существующие способы оптимизации алгоритма CRC--32
    \item Выбрана целевая платформа для проведения измерений с учетом необходимых расширений процессора
    \item Проведена адаптация реализации для x86 под RISC-V с использованием инструкции clmul
    \item Произведены измерения быстродействия оптимизированного кода на симуляторе gem5.
  \end{itemize}
\end{frame}

\end{document}