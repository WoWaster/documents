% !TeX spellcheck = ru_RU
% !TEX root = vkr.tex

\section{Постановка задачи}
\label{sec:task}

Дословно \enquote{Целью работы является... Для её выполнения были постав\-лены следующие задачи:}
\begin{enumerate}
    \item реализовать это (раздел~\ref{subsec:task1});
    \item спроектировать то-то (раздел~\ref{subsec:task2}) наилучшим образом;
    \item протестировать на том-то (раздел~\ref{subsec:task3}) и обогнать тех-то;
    \item \sout{изучить язык \OCaml{}} писать тут не надо, так как тут должны быть задачи, выполнение которых можно проверить/оценить прочитав текст или выслушав доклад;
          (т.е. Ваши достижения должны быть опровержимы)
          \begin{itemize}
              \item это может вызвать сомнения по поводу обзора~--- \emph{выполнить обзор} писать можно и нужно, но защищаемым результатом будут не ваши знания, а текст обзора (то есть он должен иметь ценность сам по себе);
          \end{itemize}
    \item обязательна задача на валидацию результата, будь то эксперимент, апробация, внедрение~--- то есть доказательство того, что Вы сделали что-то, нужное пользователю.
          Не путайте с валидацией~--- доказательством того, что Вы сделали то, что хотели Вы (например, тесты~--- валидация результата, хорошо, но недостаточно).
\end{enumerate}
