% \documentclass{beamer}
%Для защит онлайн лучше использовать разрешение 16x9
\documentclass[aspectratio=169]{beamer}

%%% Обязательные пакеты
%% Beamer
\usepackage{beamerthemesplit}
\usetheme{SPbGU}
\beamertemplatenavigationsymbolsempty
\usepackage{appendixnumberbeamer}

%% Локализация
\usepackage{fontspec}
\setmainfont{CMU Serif}
\setsansfont{CMU Sans Serif}
\setmonofont{CMU Typewriter Text}
%\setmonofont{Fira Code}[Contextuals=Alternate,Scale=0.9]
%\setmonofont{Inconsolata}
% \newfontfamily\cyrillicfont{CMU Serif}

\usepackage{polyglossia}
\setdefaultlanguage{russian}
\setotherlanguage{english}
\usepackage[autostyle]{csquotes} % Правильные кавычки в зависимости от языка

%% Графика
\usepackage{wrapfig} % Позволяет вставлять графику, обтекаемую текстом
\usepackage{pdfpages} % Позволяет вставлять многостраничные pdf документы в текст

%% Математика
\usepackage{amsmath, amsfonts, amssymb, amsthm, mathtools} % "Адекватная" работа с математикой в LaTeX

% Математические окружения с русским названием
\newtheorem{rutheorem}{Теорема}
\newtheorem{ruproof}{Доказательство}
\newtheorem{rudefinition}{Определение}
\newtheorem{rulemma}{Лемма}

%%% Дополнительные пакеты. Используются в презентации, но могут быть отключены при необходимости
\usepackage{tikz} % Мощный пакет для создание рисунков, однако может очень сильно замедлять компиляцию
\usetikzlibrary{decorations.pathreplacing,calc,shapes,positioning,tikzmark}

\usepackage{multirow} % Ячейка занимающая несколько строк в таблице

%% Пакеты для оформления алгоритмов на псевдокоде
\usepackage[noend]{algpseudocode}
\usepackage{algorithm}
\usepackage{algorithmicx}

\usepackage{fancyvrb}

\NewDocumentCommand{\xxHash}{}{\textsc{xxHash}}
\NewDocumentCommand{\riscv}{}{\textsc{RISC-V}}
\NewDocumentCommand{\xxh}{m}{\textsc{XXH{#1}}}
\NewDocumentCommand{\sew}{}{\textsc{SEW}}
\NewDocumentCommand{\vl}{}{\textsc{VL}}
\NewDocumentCommand{\rvv}{}{\textsc{RVV}}
\usepackage{booktabs}
\usepackage{tabularx}
\usepackage{siunitx} % для таблиц с единицами измерений


% То, что в квадратных скобках, отображается внизу по центру каждого слайда.
\title[Инфраструктура тестирования]{Разработка инфраструктуры для автоматизации проверки задач по курсу \enquote{Теория формальных языков}}

% То, что в квадратных скобках, отображается в левом нижнем углу.
\institute[СПбГУ]{}

% То, что в квадратных скобках, отображается в левом нижнем углу.
\author[Николай Пономарев]{Николай Алексеевич Пономарев, группа 21.Б10-мм}

\begin{document}
{
\setbeamertemplate{footline}{}
% Лого университета или организации, отображается в шапке титульного листа
\begin{frame}
	\includegraphics[width=1.4cm]{pictures/SPbGU_Logo.png}
	\vspace{-35pt}
	\hspace{-10pt}
	\begin{center}
		\begin{tabular}{c}
			\scriptsize{Санкт-Петербургский государственный университет} \\
			\scriptsize{Кафедра системного программирования}
		\end{tabular}
		\titlepage
	\end{center}

	\btVFill

	{\scriptsize
		% У научного руководителя должна быть указана научная степень
		\textbf{Научный руководитель:} доцент кафедры системного программирования, к. ф.-м. н., Григорьев С. В. \\
	}
	\begin{center}
		\vspace{5pt}
		\scriptsize{Санкт-Петербург\\
			2024}
	\end{center}

\end{frame}
}

\begin{frame}
	\frametitle{Введение}
	\begin{itemize}
		\item На мат-мехе читается курс по формальным языкам
		\item Домашние задание требуют реализации нетривиальных алгоритмов
		\item На проверку домашних заданий преподаватели тратят большое количество времени
		\item Алгоритмические задачи хорошо проверяются с помощью property-based тестирования
	\end{itemize}
\end{frame}

% Обязательный слайд: четкая формулировка цели данной работы и постановка задачи
% Описание выносимых на защиту результатов, процесса или особенностей их достижения и т.д.
\begin{frame}
	\frametitle{Постановка задачи}
	\textbf{Цель}: разработать инфраструктуру для автоматизации проверки задач по курсу \enquote{Теория формальных языков}.

	\textbf{Задачи}:
	\begin{enumerate}
		\item Изучить способы автоматического тестирования учебных задач.
		\item Создать базовую инфраструктуру для автоматического тестирования.
		\item Произвести проверку автоматических тестов, путём реализации решений
		\item Исправить накопленный технический долг
		      \begin{enumerate}
			      \item Сменить систему управления зависимостями
			      \item Обновить зависимости
		      \end{enumerate}
	\end{enumerate}
\end{frame}

\begin{frame}
	\frametitle{Виды автоматических тестов}

	Автоматические тесты можно характеризовать несколькими способами:
	\begin{itemize}
		\item По видимости
		      \begin{itemize}
			      \item открытые
			      \item закрытые
		      \end{itemize}
		\item По способу реализации
		      \begin{itemize}
			      \item ручные
			      \item property-based
			      \item смешанные
		      \end{itemize}
		\item По способу сдачи решений
		      \begin{itemize}
			      \item Learning Management Systems
			      \item Pull Request
			      \item \GitHub{} \textsc{Classroom}
		      \end{itemize}
	\end{itemize}
\end{frame}

\begin{frame}
	\frametitle{Используемые технологии}

	Используемые технологии во многом были зафиксированы.
	\begin{itemize}
		\item \python{}~--- основной язык программирования
		\item \textsc{Pytest}~--- библиотека для тестирования
		\item \cfpqdata{}~--- библиотека-датасет графов и грамматик
		\item \pyformlang{}~--- библиотека для работы с грамматиками, регулярными выражениями и различными конечными автоматами
		\item \networkx{}~--- библиотека для работы с графами
		\item \scipy{}~--- библиотека для работы с разреженными матрицами
		\item \antlr{}~--- инструмент для генерации парсеров
		\item \textsc{Grammarinator}~--- инструмент генерации слов по грамматике \antlr{}
	\end{itemize}

\end{frame}

\begin{frame}
	\frametitle{Требования к реализации}

	Научным руководителем были зафиксированы следующие требования к тестам:
	\begin{itemize}
		\item Тесты должны быть открытыми
		\item Тесты должны быть property-based
		\item Задачи должны сдаваться с помощью PR
		\item Имена и сигнатуры требуемых функций и классов известны заранее
		\item Архитектуру проекта и связь между модулями студенты определяют сами
	\end{itemize}

\end{frame}

\begin{frame}
	\frametitle{Инфраструктура решения}

	Тесты к нереализованным студентом задачам необходимо пропускать.
	Это можно сделать с помощью:
	\begin{itemize}
		\item создания отдельных репозиториев для каждой задачи
		\item конфигурации системы тестирования
		\item проверки наличия требуемых функций
	\end{itemize}

\end{frame}

\begin{frame}[fragile=singleslide]
	\frametitle{Условный запуск тестов}

	\begin{center}
		\inputminted[]{python3}{pictures/minimal_example.py}
	\end{center}

\end{frame}

\begin{frame}
	\frametitle{Структура решения}

	Здесь будет диаграмма с тем, что есть про задания и слова про реализацию (или лучше словами?)

\end{frame}

\begin{frame}
	\frametitle{Обновление используемых инструментов}

	\begin{itemize}
		\item При реализации потребовалось обновить пакеты
		\item Получили проблемы с \pip{}
		\item В семестре обновили зависимости библиотеки \cfpqdata
		\item На будущее решили перейти на \poetry
	\end{itemize}

\end{frame}

\begin{frame}
	\frametitle{Апробация}

	Весной курс читался первокурсникам магистратуры ИТМО.
	В течение семестра были выявлены следующие проблемы:
	\begin{itemize}
		\item Отсутствие ручных тестов
		\item Необходимость тестирования всех требуемых функции
		\item Необходимость предупреждать студентов о том, что решения нужно оптимизировать
	\end{itemize}

	\begin{center}
		\begin{tikzpicture}[shorten >=1pt, node distance=2cm, on grid]
			\node [circle, draw] (3) {3};
			\node [circle, draw, right = of 3] (4) {4};
			\node [circle, draw, right = of 4] (6) {6};
			\node [circle, draw, right = of 6] (7) {7};
			\node [circle, draw, right = of 7] (8) {8};
			\node [circle, draw, right = of 8] (9) {9};

			\path[->] (9) edge (8);
			\path[->] (8) edge (7);
			\path[->] (7) edge (6);
			\path[->] (6) edge (4);
			\path[->] (4) edge (3);

			\path[->] (7) edge [bend right] (4);
			\path[->] (8) edge [bend right] (4);
			\path[->] (9) edge [bend right] (4);

			\path[->] (8) edge [bend left] (6);
			\path[->] (9) edge [bend left] (6);

			\path[->] (9) edge [bend right] (7);
		\end{tikzpicture}
	\end{center}

\end{frame}

\begin{frame}
	\frametitle{Результаты}

	\begin{enumerate}
		\item Изучены способы автоматического тестирования учебных задач
		\item Создана базовая инфраструктура для автоматического тестирования
		\item Произведена проверка автоматических тестов, путём реализации решений
		\item Исправлен накопленный технический долг
	\end{enumerate}
	\blfootnote{Исходный код тестов и инфраструктуры: \url{https://github.com/FormalLanguageConstrainedPathQuerying/formal-lang-course}}
	\blfootnote{Имя пользователя: \texttt{WoWaster}}
	\blfootnote{Исходный код решений находится в приватном репозитории}
	\blfootnote{Pull Request в проект \cfpqdata{}: \url{https://github.com/FormalLanguageConstrainedPathQuerying/CFPQ_Data/pull/92}}

\end{frame}

\appendix

\begin{frame}
	\frametitle{Ход работы студента}

	Сюда чё и как сдавать

\end{frame}

\end{document}
