% !TeX spellcheck = ru_RU
% !TEX root = vkr.tex

\section*{Введение}
\thispagestyle{withCompileDate}

Сложно представить курс по информатике без практических заданий: они позволяют лучше усвоить и закрепить теоретический материал.
К сожалению, на проверку заданий у преподавателя может уходить довольно много времени, а контрольные цифры приема на специальности, связанные с информационными технологиями, увеличиваются каждый год~\cite{2024GoduKolichestvo}.
Некоторые преподаватели решают данную проблему, привлекая других проверяющих, например старшекурсников, но данный подход не является идеальным: по наблюдениям кафедры системного программирования на каждые 4--5 человек нужен один проверяющий.
Поиск такого количества людей является непростой задачей.

Курс по теории формальных языков является важной частью обучения IT-специалиста.
Формальные языки находят своё применение как в теоретических областях, таких как математическая логика~\cite{guhaParikhAutomataInfinite2022,draghiciSemEnovArithmetic2023a}, так и в прикладных: при написании парсеров~\cite{gruneParsingTechniques2008, scottGLLParsing2010}, в графовых базах данных~\cite{hellingsQueryingPathsGraphs2016,noleRegularPathQueries2016}, в биоинформатике~\cite{dyrkaEstimatingProbabilisticContextfree2019,wjandersonEvolvingStochasticContext2012}.

На математико-механическом факультете СПбГУ так же читается данный курс.
В нём имеется 12 практических заданий.
В большинстве задач требуется реализовать нетривиальный алгоритм: это вызывает сложности как у студентов, так как требуется аккуратность при выполнении, так и у проверяющих, ведь для проверки необходимо разобраться в коде студента.
Ранее при сдаче задания студенты обязаны были предоставлять рукописные тесты к каждой задаче.
Тем не менее в данной системе со временем нашелся серьёзный недостаток: тестовое покрытие не всегда описывало все крайние случаи, поэтому иногда студентам удавалось сдать не корректные решения.

При более тщательном изучении задач, появилась идея о том, что алгоритмические задачи в курсе хорошо поддадутся property-based тестированию.
Таким образом было принято решение автоматизировать проверку задач, используя property-based тесты.
Кроме того, в курсе использовались старые версии библиотек, а также стандартный менеджер пакетов \python{}.
Данные проблемы тоже было решено исправить.
