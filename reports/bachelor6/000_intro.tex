% !TeX spellcheck = ru_RU
% !TEX root = vkr.tex

\section*{Введение}
\thispagestyle{withCompileDate}

% Формат из 4х частей рекомендуется в курсе Д.~Кознова~\cite{koznov} по написанию текстов.

% \begin{enumerate}
%     \item Известная информация (background/обзор).
%     \item Неизвестная информация (пробел в знаниях, \enquote{Gap}).
%     \item Гипотезы, вопросы, цели~--- \enquote{что болит}, что будет решать Ваша работа.
%     \item Подход, план решения задачи, предлагаемое решение.
% \end{enumerate}

% Последний абзац должен читаться и быть понятен в отрыве от других трёх.
% Никакие абзацы нумеровать нельзя.

% Части (абзацы) должны занять максимум две страницы, идеально уложиться в одну.

% С.-П. Джонс~\cite{SPJGreatPaper} предлагает несколько другой формат написания введения.
% Вполне возможно, что если Ваша работа про языки программирования, то его формат будет удачнее.

% Введение и заключение читают чаще всего, поэтому они должны быть \enquote{вылизаны} до блеска.

% TL; DR. Надо автоматизировать проверку домашек, потому что время преподавателя ценно, кроме того, легко упустить тонкости в реализации сложных алгоритмов

Трудно представить курс по информатике без практических заданий: они позволяют лучше усвоить и закрепить теоретический материал.
К сожалению, на проверку заданий у преподавателя может уходить довольно много времени, а контрольные цифры приема на специальности, связанные с информационными технологиями, увеличиваются каждый год \note{Здесь ссылка на какую-нибудь новость об увеличении КЦП}.
Некоторые преподаватели решают данную проблему, привлекая других проверяющих, например старшекурсников, но данный подход не является идеальным: по наблюдениям кафедры СП \note{Да?} на каждые 4--5 человек нужен проверяющий. \note{что-то ещё надо}

Теория формальный языков является одной из важных составляющих \enquote{программистской} \note{пока не понимаю как заменить} образовательной программы.
Она находит своё применение как в теоретических областях, таких как теория сложности и теория вычислимости, так и в прикладных: при написании парсеров, в графовых базах данных, в биоинформатике \note{А вот здесь уже ссылки на статьи из README на GH}.

На математико-механическом факультете СПбГУ на направлении \enquote{Программная инженерия} данный курс читает доцент С.~В.~Григорьев.
В курсе имеется 12 домашних заданий, из которых 10 являются обязательными (обычной сложности?).
В большинстве задач требуется реализовать нетривиальный алгоритм: это вызывает сложности как у студентов, так как требуется аккуратность при выполнении задания, так и у проверяющих, ведь для проверки необходимо вникнуть в код студента.

Поэтому С.~В.~Григорьевым была предложена идея по автоматизации проверки задач на основе property-based тестов. (и тут тоже надо расширить)
