% !TeX spellcheck = ru_RU
% !TEX root = vkr.tex

\section*{Введение}
\thispagestyle{withCompileDate}

Сложно представить курс по информатике без практических заданий: они позволяют лучше усвоить и закрепить теоретический материал.
К сожалению, на проверку заданий у преподавателя может уходить довольно много времени, а контрольные цифры приема на специальности, связанные с информационными технологиями, увеличиваются каждый год \note{Здесь ссылка на какую-нибудь новость об увеличении КЦП}.
Некоторые преподаватели решают данную проблему, привлекая других проверяющих, например старшекурсников, но данный подход не является идеальным: по наблюдениям кафедры системного программирования \note{Да?} на каждые 4--5 человек нужен один проверяющий.
Поиск такого количества людей является непростой задачей.

Курс по теории формальных языков является важной частью обучения IT-специалиста.
Ведь формальные языки находит своё применение как в теоретических областях, таких как теория сложности и теория вычислимости, так и в прикладных: при написании парсеров, в графовых базах данных, в биоинформатике \note{Здесь ссылки на статьи из README на GH}.

На математико-механическом факультете СПбГУ на направлении \enquote{Программная инженерия} данный курс читает доцент С.~В.~Григорьев.
В курсе имеется 12 практических заданий.
В большинстве задач требуется реализовать нетривиальный алгоритм: это вызывает сложности как у студентов, так как требуется аккуратность при выполнении, так и у проверяющих, ведь для проверки необходимо разобраться в коде студента.
Поэтому иногда студентам удается сдать не корректное решение, потому что проверяющий \enquote{на глаз} не нашел ошибку.

Тем не менее алгоритмические задачи довольно часто поддаются property-based тестированию, так как известны свойства входных и выходных данных.
Поэтому С.~В.~Григорьевым была предложена идея по автоматизации проверки задач на основе property-based тестов.

\note{Ссылки будут потом, когда я получу одобрение на текст введения или хотя бы общие идеи}
