% !TeX spellcheck = ru_RU
% !TEX root = vkr.tex

\section{Обзор}
\label{sec:relatedworks}

\subsection{Обзор существующих решений}
\label{subsec:types}

Автоматические тесты можно разделить по нескольким критериям: по видимости, по способу реализации, по способу сдачи решений.

\subsubsection{Классификация по видимости}

По видимости тесты можно разделить на открытые и закрытые.

Открытые тесты явно доступны студентам.
Инфраструктура для такого способа проста, так как не требуются сложные системы для выдачи тестов.
Тем не менее в такой системе есть шанс появления частичных решений~--- таких решений, которые полностью проходят имеющиеся тесты, но не решают задачу полностью.

Закрытые тесты каким-либо образом скрыты от студента.
Такие тесты сложнее с точки зрения инфраструктуры, но в такой системе сложнее \enquote{подгонять} решение к тестам.

\subsubsection{Классификация по способу реализации}

По способу реализации тесты бывают ручные и property-based.

Ручные тесты проще для реализации.
Хотя имеют и минусы: их нужно написать в достаточном объеме, чтобы покрыть различные случаи.
Кроме того, они так же подвержены частичному решению.

Property-based тесты проверяют известные свойства программы.
Тестовые данные генерируются при каждом запуске, тем самым обеспечивая широкое покрытие случаев.
Но здесь кроется проблема: если одна тестируемая функция тестируется другой тестируемой (не эталонной) функцией, то можно получить работающие тесты при некорректном поведении.

В качестве универсального решения можно совмещать оба похода.
Тогда за базовую корректность отвечают ручные тесты, а за полноту реализации~--- property-based тесты.

\subsubsection{Классификация по способу сдачи решений}
\note{Раздел с одной стороны логичный, с другой тут сильно проще продолбаться с перечислением опций, пока пусто}

PR \note{Понимаю что написать}, github classroom \note{Спросить у Вовы}, LMS \note{Почитать работы Литвиноских ребят?}

\note{Воркфлоу -- студент делает форк, а там на каждую задачку делает ветку. Для проверки открывает PR. В среднем задача проверяющего удостоверится в том, что тесты проходят. (Хотя иногда стоит смотреть в код) \\
    Почему не Github Classroom? Так научник решил}

\subsection{Обзор используемых технологий}

Данный курс читается не первый год, поэтому технологии во многом были зафиксированы.
\begin{itemize}
    \item \python{}\footnote{\url{https://www.python.org/}~--- официальный сайт \python{}~(дата доступа: \DTMdate{2024-05-27}).}~--- основной язык программирования.
    \item \textsc{Pytest}\footnote{\url{https://pytest.org/}~--- официальный сайт библиотеки \textsc{Pytest}~(дата доступа: \DTMdate{2024-05-27}).}~--- библиотека для тестирования.
    \item \cfpqdata{}\footnote{\url{https://formallanguageconstrainedpathquerying.github.io/CFPQ_Data/}~--- официальный сайт библиотеки \cfpqdata{}~(дата доступа: \DTMdate{2024-05-27}).}~--- библиотека-датасет графов и грамматик.
    \item \textsc{Pyformlang}~\cite{romeroPyformlangEducationalLibrary2021}~--- библиотека для работы с грамматиками, регулярными выражениями и различными конечными автоматами.
    \item \networkx{}~\cite{SciPyProceedings_11}~--- библиотека для работы с графами.
    \item \textsc{SciPy}~\cite{virtanenSciPyFundamentalAlgorithms2020}~--- библиотека для работы с разреженными матрицами.
    \item \textsc{ANTLR4}~\cite{parrDefinitiveANTLRReference2013}~--- инструмент для генерации парсеров.
\end{itemize}

В процессе работы появилась необходимость генерации слов из языка, описываемого по грамматикой для ANTLR.
Для этого использовалась библиотека \textsc{Grammarinator}.
Задача генерации слов из языка встречается не часто, поэтому \textsc{Grammarinator} является единственным существующим решением~\cite{GeneratingGrammarconformantTexts}.
