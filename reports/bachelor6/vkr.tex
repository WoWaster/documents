% !TEX TS-program = xelatex
% !BIB program = bibtex
% !TeX spellcheck = ru_RU

% About magic macros see also
% https://tex.stackexchange.com/questions/78101/

% По умолчанию используется шрифт 14 размера.
% Если Вы не влезаете в лимит страниц и нужен 12-й шрифт,
% то уберите опцию [14pt]
\documentclass[14pt, russian]{matmex-diploma-custom}

%%% PDF settings
\pdfvariable minorversion 7 % Set PDF version to 1.7.

%%% Fonts and language setup.
\usepackage{polyglossia}
% Setup fonts.
\usepackage{fontspec}
\setmainfont{CMU Serif}
\setsansfont{CMU Sans Serif}
\setmonofont{CMU Typewriter Text}

\usepackage{microtype} % Add fancy-schmancy font tricks

\usepackage{xcolor} % Add colors support.

%% Math
\usepackage{amsmath, amsfonts, amssymb, amsthm, mathtools} % Advanced math tools.
\usepackage{thmtools}
\usepackage{unicode-math} % Allow TTF and OTF fonts in math and allow direct typing unicode math characters.
\unimathsetup{
    warnings-off={
            mathtools-colon,
            mathtools-overbracket
        }
}
\setmathfont{Latin Modern Math} % default
\setmathfont[range={\setminus,\varnothing,\smashtimes}]{Asana Math}

%%% Images
\usepackage{graphicx}
\graphicspath{{figures/}}
\usepackage{import}

%%% Polyglossia setup after (nearly) everything as described in documentation.
\setdefaultlanguage{russian}
\setotherlanguage{english}

\usepackage{csquotes}

%%% Custom commands
\newcommand{\R}{\mathbb{R}}
\newcommand{\N}{\mathbb{N}}
\newcommand{\Z}{\mathbb{Z}}
\newcommand{\Q}{\mathbb{Q}}
\newcommand{\C}{\mathbb{C}}
\newcommand{\id}{\mathrm{id}}
\AtBeginDocument{\renewcommand{\leq}{\leqslant}}
\AtBeginDocument{\renewcommand{\geq}{\geqslant}}
\AtBeginDocument{\renewcommand{\Re}{\operatorname{Re}}}
\AtBeginDocument{\renewcommand{\Im}{\operatorname{Im}}}
\AtBeginDocument{\renewcommand{\phi}{\varphi}}
\AtBeginDocument{\renewcommand{\epsilon}{\varepsilon}}

%%% theorem-like envs
\theoremstyle{definition}

\declaretheoremstyle[spaceabove=0.5\topsep,
    spacebelow=0.5\topsep,
    headfont=\bfseries\sffamily,
    bodyfont=\normalfont,
    headpunct=.,
    postheadspace=5pt plus 1pt minus 1pt]{myStyle}
\declaretheoremstyle[spacebelow=\topsep,
    headfont=\bfseries\sffamily,
    bodyfont=\normalfont,
    headpunct=.,
    postheadspace=5pt plus 1pt minus 1pt,]{myStyleWithFrame}
\declaretheoremstyle[spacebelow=\topsep,
    headfont=\bfseries\sffamily,
    bodyfont=\normalfont,
    headpunct=.,
    postheadspace=5pt plus 1pt minus 1pt,
    qed=\blacksquare]{myProofStyleWithFrame}

\usepackage[breakable]{tcolorbox}
\tcbset{sharp corners=all, colback=white}
% \tcolorboxenvironment{theorem}{}
% \tcolorboxenvironment{theorem*}{}
% \tcolorboxenvironment{axiom}{}
% \tcolorboxenvironment{assertion}{}
% \tcolorboxenvironment{lemma}{}
% \tcolorboxenvironment{proposition}{}
% \tcolorboxenvironment{corollary}{}
% \tcolorboxenvironment{definition}{}
% \tcolorboxenvironment{proofReplace}{toprule=0mm,bottomrule=0mm,rightrule=0mm, colback=white, breakable }

\declaretheorem[name=Теорема, style=myStyleWithFrame]{theorem}
\declaretheorem[name=Теорема, numbered=no, style=myStyleWithFrame]{theorem*}
\declaretheorem[name=Аксиома, sibling=theorem, style=myStyleWithFrame]{axiom}
\declaretheorem[name=Преположение, sibling=theorem, style=myStyleWithFrame]{assertion}
\declaretheorem[name=Лемма, style=myStyleWithFrame]{lemma}
\declaretheorem[name=Предложение, sibling=theorem, style=myStyleWithFrame]{proposition}
\declaretheorem[name=Следствие, numberwithin=theorem, style=myStyleWithFrame]{corollary}

\declaretheorem[name=Определение, style=myStyleWithFrame]{definition}
\declaretheorem[name=Свойство, style=myStyle]{property}
\declaretheorem[name=Свойства, numbered=no, style=myStyle]{propertylist}

\declaretheorem[name=Пример, style=myStyle]{example}
\declaretheorem[name=Замечание, numbered=no, style=myStyle]{remark}

\declaretheorem[name=Доказательство, numbered=no, style=myProofStyleWithFrame]{proofReplace}
\renewenvironment{proof}[1][\proofname]{\begin{proofReplace}}{\end{proofReplace}}
% \declaretheorem[name=Доказательство, numbered=no, style=myProofStyleWithFrame]{longProof}

%%% Memoir settings
\chapterstyle{ger}
\setlength{\headheight}{2\baselineskip}

%%% HyperRef
\usepackage{hyperref}


\begin{document}
% TODO: Formatting
% !TeX spellcheck = ru_RU
% !TEX root = vkr.tex

%% Если что-то забыли, при компиляции будут ошибки Undefined control sequence \my@title@<что забыли>@ru
%% Если англоязычная титульная страница не нужна, то ее можно просто удалить.
\filltitle{ru}{
	%% Актуально только для курсовых/практик. ВКР защищаются не на кафедре а в ГЭК по направлению,
	%%   и к моменту защиты вы будете уже не в группе.
	chair              = {Кафедра системного программирования},
	group              = {21.Б10-мм},
	%
	%% Макрос filltitle ненавидит пустые строки, поэтому обязателен хотя бы символ комментария на строке
	%% Актуально всем.
	title              = {Разработка инфраструктуры для сборки проектов с открытым исходным кодом для архитектуры \riscv{}},
	%
	%% Здесь указывается тип работы. Возможные значения:
	%%   production - производственная практика;
	%%   coursework - отчёт по курсовой работе;
	%%   practice - отчёт по учебной практике;
	%%   prediploma - отчёт по преддипломной практике;
	%%   master - ВКР магистра;
	%%   bachelor - ВКР бакалавра.
	type               = {practice},
	%
	%% Здесь указывается вид работы. От вида работы зависят критерии оценивания.
	%%   solution - <<Решение>>. Обучающемуся поручили найти способ решения проблемы в области разработки программного обеспечения или теоретической информатики с учётом набора ограничений.
	%%   experiment - <<Эксперимент>>. Обучающемуся поручили изучить возможности, достоинства и недостатки новой технологии, платформы, языка и т. д. на примере какой-то задачи.
	%%   production - <<Производственное задание>>. Автору поручили реализовать потенциально полезное программное обеспечение.
	%%   comparison - <<Сравнение>>. Обучающемуся поручили сравнить несколько существующих продуктов и/или подходов.
	%%   theoretical - <<Теоретическое исследование>>. Автору поручили доказать какое-то утверждение, исследовать свойства алгоритма и т.п., при этом не требуя написания кода.
	kind               = {experiment},
	%
	author             = {Пономарев Николай Алексеевич},
	%
	%% Актуально только для ВКР. Указывается код и название направления подготовки. Типичные примеры:
	%%   02.03.03 <<Математическое обеспечение и администрирование информационных систем>>
	%%   02.04.03 <<Математическое обеспечение и администрирование информационных систем>>
	%%   09.03.04 <<Программная инженерия>>
	%%   09.04.04 <<Программная инженерия>>
	%% Те, что с 03 в середине --- бакалавриат, с 04 --- магистратура.
	specialty          = {02.03.03 <<Математическое обеспечение и администрирование информационных систем>>},
	%
	%% Актуально только для ВКР. Указывается шифр и название образовательной программы. Типичные примеры:
	%%   СВ.5006.2017 <<Математическое обеспечение и администрирование информационных систем>>
	%%   СВ.5162.2020 <<Технологии программирования>>
	%%   СВ.5080.2017 <<Программная инженерия>>
	%%   ВМ.5665.2019 <<Математическое обеспечение и администрирование информационных систем>>
	%%   ВМ.5666.2019 <<Программная инженерия>>
	%% Шифр и название программы можно посмотреть в учебном плане, по которому вы учитесь.
	%% СВ.* --- бакалавриат, ВМ.* --- магистратура. В конце --- год поступления (не обязательно ваш, если вы были в академе/вылетали).
	programme          = {СВ.5006.2019 <<Математическое обеспечение и администрирование информационных систем>>},
	%
	%% Актуально только для ВКР, только для матобеса и только 2017-2018 годов поступления. Указывается профиль подготовки, на котором вы учитесь.
	%% Названия профилей можно найти в учебном плане в списке дисциплин по выбору. На каком именно вы, вам должны были сказать после второго курса (можно уточнить в студотделе).
	%% Вот возможные вариканты:
	%%   Математические основы информатики
	%%   Информационные системы и базы данных
	%%   Параллельное программирование
	%%   Системное программирование
	%%   Технология программирования
	%%   Администрирование информационных систем
	%%   Реинжиниринг программного обеспечения
	% profile            = {Системное программирование},
	%
	%% Актуально всем.
	%supervisorPosition = {проф. каф. СП, д.ф.-м.н., проф.}, % Терехов А.Н.
	supervisorPosition = {старший преподаватель Кафедры\\системного программирования}, % Григорьев С.В.
	supervisor         = {Я.~А.~Кириленко},
	%
	%% Актуально только для практик и курсовых. Если консультанта нет, закомментировать или удалить вовсе.
	% consultantPosition = {должность ООО <<Место работы>>, степень,},
	% consultant         = {К.~К.~Консультант},
	%
	%% Актуально только для ВКР.
	% reviewerPosition   = {должность ООО <<Место работы>> степень},
	% reviewer           = {Р.~Р.~Рецензент},
}

% \filltitle{en}{
%     chair              = {Advisor's chair},
%     group              = {ХХ.BХХ-mm},
%     title              = {Template for SPbU qualification works},
%     type               = {practice},
%     author             = {FirstName Surname},
%     %
%     %% Possible choices:
%     %%   02.03.03 <<Software and Administration of Information Systems>>
%     %%   02.04.03 <<Software and Administration of Information Systems>>
%     %%   09.03.04 <<Software Engineering>>
%     %%   09.04.04 <<Software Engineering>>
%     %% Те, что с 03 в середине --- бакалавриат, с 04 --- магистратура.
%     specialty          = {02.03.03 ``Software and Administration of Information Systems''},
%     %
%     %% Possible choices:
%     %%   СВ.5006.2017 <<Software and Administration of Information Systems>>
%     %%   СВ.5162.2020 <<Programming Technologies>>
%     %%   СВ.5080.2017 <<Software Engineering>>
%     %%   ВМ.5665.2019 <<Software and Administration of Information Systems>>
%     %%   ВМ.5666.2019 <<Software Engineering>>
%     programme          = {СВ.5006.2019 ``Software and Administration of Information Systems''},
%     %
%     %% Possible choices:
%     %%   Mathematical Foundations of Informatics
%     %%   Information Systems and Databases
%     %%   Parallel Programming
%     %%   System Programming
%     %%   Programming Technology
%     %%   Information Systems Administration
%     %%   Software Reengineering
%     % profile            = {Software Engineering},
%     %
%     %% Note that common title translations are:
%     %%   кандидат наук --- C.Sc. (NOT Ph.D.)
%     %%   доктор ... наук --- Sc.D.
%     %%   доцент --- docent (NOT assistant/associate prof.)
%     %%   профессор --- prof.
%     supervisorPosition = {Sc.D, prof.},
%     supervisor         = {S.S. Supervisor},
%     %
%     consultantPosition = {position at ``Company'', degree if present},
%     consultant         = {C.C. Consultant},
%     %
%     reviewerPosition   = {position at ``Company'', degree if present},
%     reviewer           = {R.R. Reviewer},
% }

\maketitle
\setcounter{tocdepth}{2}
\tableofcontents

\pagebreak
\newfontfamily\myfont{CMU Sans Serif}
\begin{center}
    \hspace{0pt}
    \vfill
    {\Huge\myfont
        Текст ВКР или учебной практики пишется не ради зачета, а чтобы люди его прочитали, поняли как круто Вы все сделали, и могли продолжить с того места, где Вы остановились.
        \vspace{2em}

        Повторять эту страницу в тексте вашей работы \emph{нельзя}.}
    \vfill
    \hspace{0pt}
\end{center}
\pagebreak

% !TeX spellcheck = ru_RU
% !TEX root = vkr.tex


\section*{Введение}
\thispagestyle{withCompileDate}

Искусственный интеллект и анализ графов~--- одни из наиболее привлекательных областей науки в данный момент~\cite{chuiEconomicPotentialGenerative, garciaPathQueryingGraph2024}.
Многие алгоритмы, используемые в этих областях, основаны на линейной алгебре или могут быть переформулированы в её терминах, позволяя использовать развитую экосистему для работы с линейной алгеброй.
Поскольку вычисления в линейной алгебре часто независимы друг от друга, разумно использовать возможности параллельного программирования для ускорения работы алгоритмов.
Кроме того, реальные данные часто являются разреженными~\cite{davisUniversityFloridaSparse2011}, что позволяет использовать алгоритмы разреженной линейной алгебры.

К сожалению, распараллеливание разреженной линейной алгебры~--- сложная задача для традиционных архитектур вычислителей таких, как CPU или GPU, из-за нелокальных обращений к памяти и непредсказуемого количества независимых подзадач~\cite{duHighPerformanceSparseLinear2022,isaac-chassandeDedicatedHardwareAccelerators2024,mohammedPerformanceEnhancementStrategies2022}.
В настоящее время для решения этих проблем всё чаще применяют ускорители на специализированных архитектурах~\cite{dakkakAcceleratingReductionScan2019, zhuMobileMachineLearning2018, jouppiInDatacenterPerformanceAnalysis2017, akkadEmbeddedDeepLearning2024, silvanoSurveyDeepLearning2024}.

\INs{}~--- модель вычислений, которая была описана Yves Lafont в 1990 году.
В этой модели программа представляется в виде графа, и, в силу свойств модели, вычисления происходят только локально и между конечным множеством вершин за шаг, поэтому в данной модели легко достигается параллельность.

Для \INs{} был написан не один интерпретатор, например~\cite{mackieParallelEvaluationInteraction2016, salikhmetovTokenpassingOptimalReduction2016}, однако попыток реализовать ускоритель на его основе не предпринималось, кроме того существующие интерпретаторы используют собственные языки программирования далёкие от распространённых.
Поэтому в рамках проекта \Lamagraph{}\footnote{Репозитории проекта доступны по ссылке: \url{https://github.com/Lamagraph/} (дата обращения: \DTMdate{2025-05-18})} исследуются возможности по разработке параметризуемого многоядерного сопроцессора для разреженной линейной алгебры на основе \INs{} и ML-подобного функционального языка для программирования сопроцессора.

% !TeX spellcheck = ru_RU
% !TEX root = vkr.tex

\section{Постановка задачи}
\label{sec:task}

Целью работы является разработка инфраструктуры для автоматизации проверки задач по курсу \enquote{Теория формальных языков}.
Для её выполнения были поставлены следующие задачи:
\begin{enumerate}
    \item Изучить способы автоматического тестирования учебных задач~(раздел~\ref{subsec:types}).
    \item Создать базовую инфраструктуру для автоматического тестирования~(раздел~\ref{subsec:infra}).
    \item Реализовать решения задач курса~(раздел~\ref{subsec:slns}).
    \item Обновить используемые инструменты~(раздел~\ref{subsec:housekeeping}).
\end{enumerate}

% !TeX spellcheck = ru_RU
% !TEX root = vkr.tex

\section{Обзор}
\label{sec:relatedworks}

В данном разделе нужно описать всё, что необходимо для понимания Вашей работы и что придумали не Вы.
В дальнейших разделах нельзя прерывать повествование, например, для рассказа о деталях используемой технологии или архитектуре старой системы, потому что читателю будет трудно отличить Ваш вклад от не Вашего.

Любой обзор пишется с какой-то целью (обосновать актуальность, найти и описать интересные решения, сравнить и выбрать технологии) и по какой-то методике поиска материала (например, поиск N релевантных статей на таких-то сервисах).
Не будет лишним это всё явно описать.

\subsection{Обзор существующих решений}

\emph{Обзор существующих решений должен быть.}
Здесь нужно писать, что индустрия и наука уже сделали по вашей теме.
Он нужен, чтобы Вы случайно не изобрели какой-нибудь велосипед.

По-английски называется related works или previous works.

Если Ваша работа является развитием предыдущей и плохо понима\-ема без неё, то обзор должен идти в начале.
Если же Вы решаете некоторую задачу новым интересным способом, то если поставить обзор в начале, то читатель может устать, пока доберется до вашего решения.
В этом случае уместней поставить обзор после описания Вашего подхода к проблеме%
\footnote{Такой подход рекомендуется в работе~\cite{SPJGreatPaper}.
    Вполне возможно, что Ваш реальный научный руководитель будет не согласен, и потребует, чтобы обзор был в начале.}.

В обзоре вам нужно рассказать про \emph{преимущества и недостатки} того, что было сделано до Вас.
Неправильным будет перечислять только недостатки, так как если Ваша работа хоть где-то хуже предыдущей, то рецензент будет радостно потирать руки и заваливать Вашу работу.
Гораздо лучше, если Вы честно признаетесь в этом сами.

\subsection{Обзор используемых технологий}

Для технических работ обзор может обозревать продукт, в рамках которого Вы выполняете задачу, другие продукты, где решалась схожая задача, а также используемые технологии с обоснованием выбора тех, которые Вы дальше используете.
Это всё, скорее всего, будет отдельными подразделами обзора.
\enquote{Выбор} подразумевает наличие вариантов, поэтому опишите, из чего выбирали и почему выбрали то, что выбрали.
Очень желательны чёткие критерии сравнения и сводная таблица в конце, где стоят плюсы и минусы рядом с каждым рассматриваемым вариантом.

В обзоре необходимо ссылаться на работы других людей.
В данном шаблоне задумано, что литература будет указываться в файле \verb=vkr.bib=.
В нём указываются пункты литературы в формате \BibTeX{}, а затем на них можно ссылаться с помощью \verb=\cite{...}=.
Та литература, на которую Вы сошлетесь, попадет в список литературы в конце документа.
Если не сошлетесь~---  не попадёт.
Спецификацию в формате \BibTeX{} почти никогда (для второго курса~--- никогда), не нужно придумывать руками.
Правильно: находить в интернете описание цитируемой статьи%
\footnote{Например, \url{https://dl.acm.org/doi/10.1145/3408995} (дата обращения: \DTMdate{2022-12-17}).},
копировать цитату с помощью кнопки \foreignquote{english}{Export Citation} и вставлять в \BibTeX{} файл.
Так же умеет генерировать \BibTeX{}-описания и Google Scholar%
\footnote{Поисковая система для научных текстов Google Scholar, \url{https://scholar.google.com} (дата обращения: \DTMdate{2024-01-13})}.
Если так не делать, то оформление литературы будет обрастать ошибками.
Например, \BibTeX{} по особенному обрабатывает точ\-ки, запятые и \verb=and= в списке авторов, что позволяет ему самому понимать, сколько авторов у статьи, и что там фамилия, что~--- имя, а что~--- отчество.
Google Scholar пытается генерировать описания автоматически, так что, возможно, потребуется ручная правка~--- обязательно проверьте свой список литературы.

В обзоре и в остальном тексте вы наверняка будете использовать названия продуктов или языков программирования (например, \csharp{}).
Для них рекоменду\-ется (в файле \verb=preamble2.tex=) за\-дать специальные команды, чтобы писать сложные названия правильно и одинаково по всему доку\-менту.
Написать с ошибкой  название любимого языка программирова\-ния науч\-ного руко\-водителя~--- идеальный вариант его разозлить.

\subsection{Выводы}

Опишите явно, что читатель должен был вынести из обзора в отдельном подразделе.

% !TeX spellcheck = ru_RU
% !TEX root = vkr.tex

\section{Реализация}

После обсуждения результатов обзора были зафиксированы следующие требования к тестам:
\begin{itemize}
    \item Тесты должны быть открытыми.
    \item Тесты должны быть property-based.
    \item Задачи должны сдаваться с помощью PR.
    \item Имена и сигнатуры требуемых функций и классов известны заранее.
    \item Архитектуру проекта и связь между модулями студенты определяют сами.
\end{itemize}

\subsection{Инфраструктура тестирования}
\label{subsec:infra}

\begin{listing}[b]
    \caption{Код отвечающий за запуск или пропуск теста, в зависимости от наличия или отсутствия решения задачи}
    \inputminted[linenos, breaklines, frame=single, fontsize = \small]{python3}{figures/minimal_example.py}
    \label{listing:skip}
\end{listing}

Основная проблема при написании тестов к курсу~--- необходимость запуска тестов только для реализованных студентом заданий.

Одним из самых простых решений является создание отдельного репозитория для каждой задачи.
Такой подход крайне неудобен для нашего курса, так как создавать 12 репозиториев для заданий, которые зависят друг от друга, а затем добавлять в каждый проверяющего, будет слишком неудобно.

В качестве альтернативного решения можно запускать только необходимые тесты, например указывая их как параметр для фреймворка тестирования.
Здесь не требуется создавать много репозиториев, и после выполнения каждой задачи студенту достаточно будет включить необходимые тесты.
Но кроме этого потребуется исправить пути импорта тестируемых модулей.

Наш подход основывается на том, что пути импорта исправлять нужно всегда и это может служить минимальным требованием для запуска тестов.
\python{}~--- интерпретируемый язык, поэтому поиск необходимых модулей происходит во время исполнения.
Это позволяет прямо во время исполнения узнать существует ли решение нужной задачи, и в зависимости от этого запустить или пропустить тесты.
Достаточно попробовать импортировать нужные объекты, и, в случае получения \texttt{ImportError}, выставить флаг пропуска всех тестов в модуле~(см. листинг \ref{listing:skip}).

Вместе с требованиями и способом запуска текстов, процесс сдачи заданий иллюстрирует рисунок~\ref{fig:workflow}.

\begin{figure}[h]
    \caption{Процесс сдачи заданий студентом}
    \label{fig:workflow}
    \includegraphics[width=\textwidth, trim=0.8cm 0.8cm 0.8cm 0.8cm, clip]{student_workflow.pdf}
\end{figure}

\subsection{Проверка автоматических тестов}
\label{subsec:slns}

Написанием самих тестов занимался мой товарищ, Ефим Кубышкин.
Моей задачей была проверка тестов, путём реализации задач из курса.

Задачи в курсе разделены на три крупных блока.
\begin{enumerate}
    \item Регулярные языки.
    \item Контекстно-свободные запросы.
    \item Синтаксис и семантика.
\end{enumerate}
И обычно предполагают либо создание обертки над библиотечными функциями, либо самостоятельную реализацию алгоритма.

В первом блоке предлагается реализовать два алгоритма регулярных запросов к графу: тензорный алгоритм и алгоритм на основе поиска в ширину.
Во втором блоке необходимо реализовать четыре алгоритма исполнения контекстно-свободных запросов: алгоритм Хеллингса, матричный алгоритм Рустама Азимова, тензорный алгоритм, алгоритм на основе GLL.
Третий блок предполагает работу с модельным языком запросов к графам: написание парсера с использованием \antlr{}, реализацию вывода типов, а затем интерпретатора.

Задача №1 вводная, в ней необходимо создать fork, написать две функции и тесты к ним для знакомства с библиотеками, а также настроить CI для запуска тестов.
Автотесты к ней не были реализованы.

В задаче №2 необходимо научиться преобразовывать регулярное выражение в ДКА и граф в формате \networkx{} в НКА, используя библиотеку \pyformlang{}.
Обе функции по сути являются обертками на цепочками функций из библиотеки.

В задаче №3 требовалось реализовать класс, описывающий конечный автомат в формате разреженной матрицы из \scipy{}, а затем тензорный алгоритм выполнения регулярных запросов к графам~\cite{shemetovaOneAlgorithmEvaluate2021}.

В задаче №4 необходимо было реализовать алгоритм выполнения регулярных запросов к графу на основе обхода в ширину~\cite{elekesGraphBLASSolutionSIGMOD2020}.

Задача №5 подразумевает постановку экспериментов, поэтому не тестировалась.

В задаче №6 необходимо было реализовать функцию преобразования контекстно-свободной грамматики в ослабленную нормальную форму Хомского (ОНФХ) и алгоритм исполнения контекстно-свободных запросов Хеллингса~\cite{hellingsConjunctiveContextFreePath2014}.

В задаче №7 требовалось реализовать матричный алгоритм исполнения контекстно-свободных запросов Рустама Азимова~\cite{azimovContextfreePathQuerying2018}.

В задаче №8 необходимо было сначала научиться работать с Recursive State Machine~(RSM) из \pyformlang{} путём реализации функций конвертации различных представлений КС грамматик в RSM.
А затем реализовать тензорный алгоритм исполнения контекстно-свободных запросов~\cite{orachevContextFreePathQuerying2020, shemetovaOneAlgorithmEvaluate2021}.

В задаче №9 требовалось реализовать алгоритм исполнения контекстно-свободных запросов на основе Generalized LL~\cite{abzalovGLLbasedContextFreePath2023}.

Задача №10 подразумевает постановку экспериментов, поэтому не тестировалась.

В задаче №11 необходимо было написать парсер модельного языка запросов к графам с использованием \antlr{}, а так же функцию подсчёта количества узлов и функцию преобразования дерева разбора в текст программы.

В задаче №12 требовалось реализовать механизм вывода типов и интерпретатор для языка из задания №11.
Эта задача достаточно сильно отличается от предыдущих, так как алгоритм вывода типов требуется придумать самому.
В модельном ЯП не присутствует полиморфизм, что облегчает задачу.
Однако присутствует возможность ссылаться на ещё не объявленные переменные, так как большая часть объявлений~--- правила КС грамматики в регулярной форме (см. листинг \ref{listing:example}).
В таком случае необходимо отдельно строить граф зависимостей объявлений друг от друга.

\begin{listing}
    \caption{Пример объявления грамматики, задающей язык $a^n b^n$, в модельном языке}
    \begin{minted}[linenos, breaklines, frame=single, fontsize = \small]{rust}
        let a = ("a" . b) | "a" ^ [0..0]
        let b = a . "b"
        \end{minted}
    \label{listing:example}
\end{listing}

Реализованные задачи выкладывались в приватный репозиторий.
В нём был настроен CI, который запускал автотесты из \texttt{main} ветки основного репозитория на эталонных решениях.
Общая структура решения продемонстрирована на рисунке~\ref{fig:structure}.

\begin{figure}[h]
    \caption{Структура решения}
    \label{fig:structure}
    \includegraphics[width=\textwidth, trim=0.8cm 0.8cm 0.8cm 0.8cm, clip]{project_structure.pdf}
\end{figure}

\subsection{Обновление используемых инструментов}
\label{subsec:housekeeping}

При реализации одного из заданий появилась необходимость в использовании транзитивного замыкания графа.
Однако при запуске выяснилось, что установленная версия \networkx{} не поддерживает транзитивное замыкание для класса \texttt{MultiDiGraph}.
Дальнейшее исследование показало, что не смотря на указание более свежей версии \networkx{}, \pip{}~--- стандартный пакетный менеджер для \python{}~--- выбирал более старую версию, так как её требовал пакет \cfpqdata{}.

В течение семестра для решения этой проблемы был создан Pull Request в библиотеку \cfpqdata{} с обновлением всех её зависимостей.
Поддержкой библиотеки занимается старший преподаватель кафедры ИАС Азимов Рустам Шухратуллович, поэтому Pull Request был принят быстро, после чего была опубликована новая версия библиотеки.

Дабы избежать подобной проблемы в будущем было решено перейти на использование \poetry{} в качестве менеджера пакетов.
\poetry{} использует собственный решатель зависимостей, который сообщит об ошибке в подобной ситуации.
С данным изменением был оформлен Pull Request, который будет влит в основную ветку после окончания семестра.

% !TeX spellcheck = ru_RU
% !TEX root = vkr.tex

\section{Апробация}

\epigraph{Случайные тесты как коробка шоколадных конфет~--- никогда не знаешь, какой упадёт.}{Барсуков Илья, магистр ИТМО}

В весеннем семестре 2024 года данный курс читался первокурсникам магистратуры ИТМО.
В течение семестра были выявлены следующие проблемы:
\begin{itemize}
    \item Отсутствие ручных тестов.
    \item Необходимость тестирования всех требуемых функций.
    \item Необходимость предупреждать студентов о том, что решения нужно оптимизировать.
\end{itemize}

\subsection{Отсутствие ручных тестов}

\begin{figure}[b]
    \caption{Зависимость задач друг от друга}
    \label{fig:dependencies}
    \centering
    \begin{tikzpicture}[shorten >=1pt, node distance=2cm, on grid]
        \node [circle, draw] (3) {3};
        \node [circle, draw, right = of 3] (4) {4};
        \node [circle, draw, right = of 4] (6) {6};
        \node [circle, draw, right = of 6] (7) {7};
        \node [circle, draw, right = of 7] (8) {8};
        \node [circle, draw, right = of 8] (9) {9};

        \path[->] (9) edge (8);
        \path[->] (8) edge (7);
        \path[->] (7) edge (6);
        \path[->] (6) edge (4);
        \path[->] (4) edge (3);

        \path[->] (7) edge [bend right] (4);
        \path[->] (8) edge [bend right] (4);
        \path[->] (9) edge [bend right] (4);

        \path[->] (8) edge [bend left] (6);
        \path[->] (9) edge [bend left] (6);

        \path[->] (9) edge [bend right] (7);
    \end{tikzpicture}
\end{figure}

При постановке задачи планировалось использовать property-based тесты.
Тем не менее данный подход в одиночку оказался недостаточен.

В большинстве задач тесты основываются на сравнении ответов различных алгоритмов (см. рисунок \ref{fig:dependencies}).
Так алгоритм из задачи №3 тестируется только в тестах задачи №4, когда появляется возможность сравнить результаты.
А затем алгоритмом из задачи №4 тестируются задачи №№6--9.

К сожалению, в таком случае есть вероятность, что задачи №№3 и 4 решены одинаково неправильно, из-за чего тесты проходят, и проверяющий засчитывает некорректное решение.
А при выполнении задания №6 не проходят тесты по непонятной причине.

В такую ловушку попало как минимум два студента: их решения не обрабатывали граничный случай~--- пустую строку.
Из этого был сделан вывод о том, что для каждой задачи стоит иметь небольшой набор ручных тестов, проверяющих в том числе крайние случаи.

\subsection{Тестирование всех функций}

Для некоторых заданий требуются вспомогательные функции, которые затем могут использоваться в тестах.
Тем не менее так получается не всегда, и в таком случае стоит иметь хотя бы тесты проверяющие самое наличие функции, иначе про необходимость реализации забывают как студенты, так и проверяющие.

Основная алгоритмическая функция из задания №3 тестировалась только в задаче №4 и многие студенты забыли её реализовать, и нескольким студентам задача была зачтена без неё.
Эти студенты реализовали её при выполнении задачи №4.

\subsection{Оптимизация решений}

При реализации эталонных решений применялись различные способы оптимизации: ранний выход из цикла, использование оптимальных форматов разреженных матриц.
Студенты же выполняли задания \enquote{в лоб}, что привело к интересным последствиям.

В мае был выполнен рефакторинг кода автоматических тестов, в следствие чего количество тестов резко увеличилось.
Время исполнения тестов наших решений увеличилось немного, а вот тесты у некоторых студентов стали проходить за 4 часа.

В результате, был сделан вывод о том, что необходимо каким-либо образом мотивировать студентов на оптимизацию собственных решений.

% !TeX spellcheck = ru_RU
% !TEX root = vkr.tex

\section*{Заключение}
\textbf{Обязательно.}
Список результатов, который будет либо один к одному соответствовать задачам из раздела~\ref{sec:task}, либо их уточнять (например, если было \enquote{выбрать}, то тут \enquote{выбрано то-то}).

\begin{itemize}
    \item Результат к задаче №1.
    \item Результат к задаче №2.
    \item и т.д.
\end{itemize}
\noindent Если работа на несколько семестров, отчитывайтесь только за текущий.
Можно в свободной форме обрисовать планы продолжения работы, но не увлекайтесь~--- если работа будет продолжена, по ней будет ещё один отчёт.

В заключении \emph{обязательна} ссылка на исходный код, если он выносится на защиту, либо явно напишите тут, что код закрыт.
Если работа чисто теоретическая и это понятно из решённых задач, про код можно не писать.
Обратите внимание, что ссылка на код должна быть именно в заключении, а не посреди раздела с реализацией, где её никто не найдёт.

Старайтесь оформить программные результаты работы так, чтобы это был один репозиторий или один пуллреквест, правильно оформленный~--- комиссии тяжело будет собирать Ваши коммиты по всей истории.
А если над проектом работало несколько человек и всё успело изрядно перемешаться, неизбежны вопросы о Вашем вкладе.

Заключение люди реально читают (ещё до \enquote{основных} разделов работы, чтобы понять, что же получилось и стоит ли вообще работу читать), так что оно должно быть вылизано до блеска.


\setmonofont{CMU Typewriter Text}
\bibliographystyle{ugost2008ls}
\bibliography{vkr}

\end{document}
