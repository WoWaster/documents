% !TeX spellcheck = ru_RU
% !TEX root = vkr.tex


\section*{Введение}
\thispagestyle{withCompileDate}

Искусственный интеллект и анализ графов~--- одни из наиболее привлекательных областей науки в данный момент~\cite{chuiEconomicPotentialGenerative, garciaPathQueryingGraph2024a}.
Многие алгоритмы, используемые в этих областях, основаны на линейной алгебре или могут быть переформулированы в её терминах, позволяя использовать развитую экосистему для работы с линейной алгеброй.
Поскольку вычисления в линейной алгебре часто независимы друг от друга, разумно использовать возможности параллельного программирования для ускорения работы алгоритмов.
А для больших объемов данных разумно использовать разреженную линейную алгебру.

К сожалению, распараллеливание разреженной линейной алгебры~--- сложная задача для традиционных архитектур вычислителей таких, как CPU или GPU, из-за нелокальных обращений к памяти и непредсказуемого количества \enquote{агентов}~\cite{duHighPerformanceSparseLinear2022,isaac-chassandeDedicatedHardwareAccelerators2024,mohammedPerformanceEnhancementStrategies2022}.
В настоящее время для решения этих проблем всё чаще применяют ускорители на специализированных архитектурах~\cite{dakkakAcceleratingReductionScan2019, zhuMobileMachineLearning2018, jouppiInDatacenterPerformanceAnalysis2017, akkadEmbeddedDeepLearning2024, silvanoSurveyDeepLearning2024}.

\INs{}~--- модель вычислений, которая была описана Yves Lafont в 1990 году.
В этой модели программа представляется в виде графа, и, в силу свойств модели, вычисления происходят только локально и между конечным множеством вершин за шаг, поэтому в данной модели легко достигается параллельность.

Для \INs{} был написан не один интерпретатор, например~\cite{mackieParallelEvaluationInteraction2016, salikhmetovTokenpassingOptimalReduction2016a}, однако попыток реализовать ускоритель на его основе не предпринималось, кроме того существующие интерпретаторы используют собственные языки программирования далёкие от распространённых.
Поэтому в рамках проекта \Lamagraph{} исследуются возможности по разработке параметризуемого многоядерного сопроцессора для разреженной линейной алгебры на архитектуре \INs{} и ML-подобного функционального языка программирования для программирования сопроцессора.
