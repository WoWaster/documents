\documentclass{vkr-slides-style}

\filltitle{
    % Ваше ФИО.
    author = {Николай Алексеевич Пономарев},
    % Нельзя оставлять пустые строки.
    % Ваше сокращённое именование, будет показываться слева внизу слайда.
    authorShort = {Николай Пономарев},
    %
    % Официальное название ВКР.
    title = {Разработка транслятора модельного функционального языка в Interaction Nets},
    %
    % Короткое название ВКР, будет снизу по центру слайда.
    titleShort = {Транслятор в Interaction Nets},
    %
    % Научный руководитель.
    advisor = {к.ф.-м.н. С.В. Григорьев, доцент кафедры системного программирования},
    %
    % Консультант (если есть). Если нет, оставьте пустые фигурные скобки.
    consultant = {},
    %
    % Дата доклада.
    date = {23.11.2024}
}

\setbeamertemplate{itemize items}[circle]
\setbeamertemplate{enumerate items}[circle]

\usepackage{fontawesome}

\begin{document}

\makeslidestitle

\begin{frame}
    \frametitle{Суть работы}
    \begin{itemize}
        \item<+-> Interaction Nets~--- модель вычислений, открытая Yves Lafont в 1990 году, которой свойственен массовый параллелизм
        \item<+-> Interaction Nets, как и любую модель вычислений, можно рассматривать как основу архитектуры процессоров
        \item<+-> Разреженная линейная алгебра является базисом для многих задач
              \begin{itemize}
                  \item[\faFrownO] Плохо поддается распараллеливанию на традиционных архитектурах
                  \item[\faLightbulbO] Ускоритель на Interaction Nets
              \end{itemize}
        \item<+-> Цель проекта~--- разработка параметризуемого многоядерного сопроцессора для разреженной линейной алгебры на архитектуре Interaction Nets
        \item<+-> Экспериментальный проект на нестандартной архитектуре $\Rightarrow$ нужен свой транслятор
    \end{itemize}
\end{frame}

\begin{frame}
    \frametitle{Постановка задачи}
    \textbf{Цель:} разработать транслятор модельного функционального языка в Interaction Nets

    \vspace{5mm}
    \textbf{Задачи:}
    \begin{enumerate}
        \item Реализовать интерпретатор модельного ML-подобного языка
        \item Реализовать транслятор из обогащенного $\lambda$-исчисления в Interaction Nets
        \item Реализовать интерпретатор Interaction Nets
        \item Провести эксперименты с наборами инструкций
    \end{enumerate}
\end{frame}

\begin{frame}
    \frametitle{План работы}

    \textbf{Что уже сделано:}
    \begin{enumerate}
        \item Реализация интерпретатора модельного ML-подобного языка
              \begin{enumerate}
                  \item Конкретный синтаксис языка
                  \item AST и парсер языка
              \end{enumerate}
    \end{enumerate}

    \textbf{Планируется к зимней защите:}
    \begin{enumerate}
        \setcounter{enumi}{1}
        \item[]
              \begin{enumerate}
                  \setcounter{enumii}{2}
                  \item Алгоритм вывода типов
                  \item Рассахаривание в обогащенное $\lambda$-исчисление
                  \item Интерпретатор обогащенного $\lambda$-исчисления
              \end{enumerate}
    \end{enumerate}

    \textbf{Планируется к защите ВКР:}
    \begin{enumerate}
        \setcounter{enumi}{1}
        \item Транслятор из обогащенного $\lambda$-исчисления в Interaction Nets
        \item Интерпретатор Interaction Nets
        \item Эксперименты с наборами инструкций
    \end{enumerate}
\end{frame}

\end{document}
