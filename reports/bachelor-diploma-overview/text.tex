\documentclass[12pt]{scrartcl}

%%% Fonts and language setup.
\usepackage{polyglossia}
% Setup fonts.
\usepackage{fontspec}
\setmainfont{CMU Serif}
\setsansfont{CMU Sans Serif}
\setmonofont{CMU Typewriter Text}

\usepackage{microtype} % Add fancy-schmancy font tricks

\usepackage{xcolor} % Add colors support.

%%% Polyglossia setup after (nearly) everything as described in documentation.
\setdefaultlanguage{russian}
\setotherlanguage{english}

\usepackage[autostyle]{csquotes}



\title{Презентация темы кафедре СП}
\author{Николай Пономарев}

\begin{document}

\maketitle

% \section*{Титульный слайд}

Добрый день, меня зовут Пономарев Николай, и тема моей ВКР \enquote{Разработка транслятора модельного функционального языка в Interaction Nets}.

% \section*{Суть работы}

В 1990 году Ив Лафон изобрел Interaction Nets~--- модель вычислений, поддерживающую массовый параллелизм.
Как и любую модель вычислений, Interaction Nets можно рассматривать как основу архитектуры процессора.
Сейчас интерес научного сообщества обращен к таким областям, как анализ графов и искусственный интеллект, которые очень сильно опираются на алгоритмы разреженной линейной алгебры.
Тем не менее, разреженная линейная алгебра плохо поддается распараллеливанию на традиционных архитектурах, поэтому часто применяются ускорители на нестандартных архитектурах.
Interaction Nets как раз является такой нестандартной архитектурой, с помощью которой можно повысить параллелизм.
Поэтому целью нашего проекта является разработка параметризуемого многоядерного сопроцессора для разреженной линейной алгебры на архитектуре Interaction Nets.

% \section*{Постановка задачи}

Поскольку архитектура на Interaction Nets~--- полностью отличается от уже существующих, а сам проект экспериментальный, то использование существующих трансляторов может только усложнить разработку, поэтому целью моей работы стала разработка транслятора модельного функционального языка в Interaction Nets.
Причина выбора функциональной парадигмы заключается в том, что Interaction Nets близка к $\lambda$-исчислению, а структуры разреженной линейной алгебры хорошо выражаются через алгебраические типы данных.

Передо мной поставлены следующие задачи: реализовать интерпретатор модельного ML-подобного языка, затем реализовать транслятор из обогащенного $\lambda$-исчисления в Interaction Nets, после чего реализовать интерпретатор Interaction Nets, и, в конце, провести эксперименты с наборами инструкций.

% \section*{План работы}

На данный момент я в процессе выполнения первой задачи, если точнее, то уже описан конкретный синтаксис языка, реализованы AST и парсер.
К зимней защите планируется доделать алгоритм вывода типов, реализовать рассахаривание в обогащенное $\lambda$-исчисление, а затем его интерпретатор.
Все остальные задачи планируется выполнить к защите ВКР.


\end{document}
