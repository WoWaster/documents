\documentclass[aspectratio=169]{beamer}

% Setup fonts.
\usepackage{fontspec}
\setmainfont{Yanone Kaffeesatz}
\setsansfont{Noto Sans}
\setmonofont{Fira Code}

%%% Fonts and language setup.
\usepackage[russian]{babel}

%% Math
\usepackage{amsmath, amsfonts, amssymb, amsthm, mathtools} % Advanced math tools.
\usepackage{unicode-math} % Allow TTF and OTF fonts in math and allow direct typing unicode math characters.
\unimathsetup{
    warnings-off={
            mathtools-colon,
            mathtools-overbracket
        }
}
\setmathfont{Fira Math}
% \setmathfont{Asana Math}

\usetheme{Rochester}
\usecolortheme{Nord}
\usefonttheme{Nord}
\setbeamertemplate{navigation symbols}{}%remove navigation symbols


\usepackage{minted}
\usemintedstyle{nord}

\title{Алгоритм Рабина-Карпа}
\author{Николай Пономарев}
\date{}

\usepackage{hyperref}

\begin{document}
\begin{frame}[plain,noframenumbering]
    \maketitle
\end{frame}

\setbeamertemplate{navigation symbols}{%
    \usebeamerfont{footline}%
    \usebeamercolor[fg]{footline}%
    \hspace{1em}%
    \insertframenumber/\inserttotalframenumber
}
\begin{frame}
    \frametitle{Проблема}
    \begin{itemize}
        \item Поиск шаблона в строке достаточно частая проблема в информатике
        \item Самым простым решением является последовательное <<прикладывание>> шаблона к строке
        \item Сложность такого решения $O(nm)$
        \item Можно оптимизировать выходом из цикла при первом несовпадении, но гарантий на значительное ускорение нет
    \end{itemize}
\end{frame}

\begin{frame}
    \frametitle{Чутоку теории}
    В информатике существуют так называемые хеш-функции, которые позволяют
    преобразовать произвольные данные в конечное множество значений
    \begin{itemize}
        \item У одинаковых входных данных одинаковый хеш
        \item Однако обратное неверно, т.к. происходят коллизии
    \end{itemize}
    Коллизия -- совпадение результата хеш-функции для разных входных данных
    \begin{itemize}
        \item У <<хороших>> хеш-функций вероятность коллизий достаточно мала
    \end{itemize}
\end{frame}

\begin{frame}
    \frametitle{Задумка}
    Алгоритм Рабина-Карпа был придуман Ричардом Карпом и Майклом Рабином в 1987 году.

    Идея алгоритма заключается в подсчете хеша для подстроки и сравнении его
    с хешем образца, и только в случае равенства посимвольно сравнивать подстроку и образец
\end{frame}

\begin{frame}[fragile]
    \frametitle{Реализация}
    \begin{minted}[linenos]{python}
def rabin_karp(s: str, pattern: str) -> int:
    n = len(s)
    m = len(pattern)
    pattern_hash = hash(pattern)
    for i in range(n - m):
        substring = s[i : i + m]
        substring_hash = hash(substring)
        if pattern_hash == substring_hash:
            if pattern == substring:
                return i
    return -1
    \end{minted}
\end{frame}

\begin{frame}
    \frametitle{Сложность}
    \begin{itemize}
        \item Хеширование и посимвольное сравнение имеют сложность $O(m)$
        \item Хеширование производится на каждом шаге цикла, итоговая сложность $O(nm)$
    \end{itemize}
    Но такая же сложность и у <<наивного>> алгоритма!

    Для решения этой проблемы необходимо использовать скользящий (rolling) хеш.
\end{frame}

\begin{frame}
    \frametitle{Скользящий хеш}
    Скользящий хеш --- это такая хеш-функция, для пересчета которой достаточно знать
    ее предыдущее значение и что изменилось в данных.

    \begin{exampleblock}{Пример}
        Самый простой хеш, который складывает все коды символов в строке
        $s = a_1 a_2 ... a_n$: \[h(s) = \sum_{i=0}^n a_i,\] является скользящим.
        После сдвига на 1 вправо его можно пересчитать за константное время:
        \[h(s[2..n+1]) = h(s[1..n]) - a_1 + a_{n+1}\]
    \end{exampleblock}
\end{frame}

\begin{frame}[fragile]
    \frametitle{Полиномиальный хеш}
    Полиномиальный хеш является достаточно простым скользящим хешем,
    однако он обеспечивает достаточно низкую вероятность коллизий.
    \[h(s) = \left(\sum_{i=1}^n s[i]x^{n-i} \right) \bmod{q}\]

    На практике взятие остатка обычно выполняется после каждой операции:
    \begin{minted}[linenos]{python}
def polynomial_hash(s: str, x: int, q: int) -> int:
    hash = 0
    for a in s:
        hash = (hash * x) % q
        hash = (hash + a) % q
    return hash
    \end{minted}
\end{frame}

\begin{frame}
    \frametitle{Полиномиальный хеш. Сдвиг}
    За основу возьмем строку $s=a_1a_2a_3a_4...a_n$

    Посчитаем хеш для подстроки $s_1 = a_1a_2a_3$:
    \[h(s_1) = \left[\left(\left[\left(\left[\left(a_1 x\right) \bmod{q} + a_2\right] \bmod{q}\right) x \right] \bmod{q}\right) + a_3\right] \bmod{q}\]
    Теперь сдвинемся на 1 символ вправо и подсчитаем хеш для подстроки $s_2 = a_2a_3a_4$:
    \[h(s_2) = \left[\left(h(s_1) + q - \left[\left(a_1 x^2\right)\bmod{q} \right] \right)x + a_4\right] \bmod{q}\]
\end{frame}

\begin{frame}
    \frametitle{Полиномиальный хеш. Выбор $q$ и $x$}
    Вариант Рабина и Карпа:
    \begin{itemize}
        \item фиксированный $x=2$
        \item случайное простое число $q \in \left[2, n^3\right]$, где $n$ -- длина строки
    \end{itemize}

    Вариант Дитзфелбингера и др.:
    \begin{itemize}
        \item случайное $x \in \left(0, q-1\right]$
        \item фиксированное простое $q$ (например простое число Мерсенна: $2^{31}-1$ или $2^{61} -1$)
    \end{itemize}
\end{frame}

%\begin{frame}
%    \frametitle{Хеш Рабина}
%    Кроме полиномиального хеша достаточно часто упоминают хеш Рабина. Для применения
%    этого хеша необходимо представить исходную строку в виде 0 и 1. Идея хеша заключается
%    в том, чтобы рассматривать входную строку $a_1 a_2 ... a_n$ как многочлен
%    \[A(x) = a_1 x^{n-1} \oplus a_2x^{n-2} \oplus ... \oplus a_{n-1} x \oplus a_n x^0\]
%    над конечным полем $\mathbb{F}_2$, где $\oplus$ обозначает побитовое исключающие <<или>>,
%    а сам хеш является остатком от деления $A(x)$
%    на случайно выбранный неприводимый многочлен $P(x)$ степени $L$ над $\mathbb{F}_2$.
%\end{frame}

\begin{frame}
    \frametitle{Сложность с <<хорошим>> хешем}
    \begin{itemize}
        \item Вычисление скользящего хеша занимает $O(n)$
        \item Сравнение строк при совпадении хеша -- $O(m)$
    \end{itemize}
    Тогда сложность алгоритма $O(n+m)$
\end{frame}

\begin{frame}
    \frametitle{Применение}
    \begin{itemize}
        \item Алгоритм легко модифицируется для поиска набора образцов в тексте сохраняя
              хорошее время работы
        \item Из-за этого часто применяется для проверки на плагиат
    \end{itemize}
\end{frame}

\begin{frame}[fragile]
    \frametitle{Материалы}
    \begin{itemize}
        \item \href{https://ru.wikipedia.org/wiki/%D0%90%D0%BB%D0%B3%D0%BE%D1%80%D0%B8%D1%82%D0%BC_%D0%A0%D0%B0%D0%B1%D0%B8%D0%BD%D0%B0_%E2%80%94_%D0%9A%D0%B0%D1%80%D0%BF%D0%B0}{Русская Википедия}
        \item \href{https://en.wikipedia.org/wiki/Rabin%E2%80%93Karp_algorithm}{Английская Википедия}
        \item Rabin M. O., Karp R. M. Efficient randomized pattern-matching algorithms -- статья Рабина и Карпа с описанием алгоритма
        \item Dietzfelbinger M., Gil J., Matias Y., Pippenger N. Polynomial hash functions are reliable -- статья Дитзфелбингера и др. о полиномиальном хеше
        \item Rabin M. O. Fingerprinting by random polynomials  -- статья про хеш Рабина
    \end{itemize}
\end{frame}
\end{document}