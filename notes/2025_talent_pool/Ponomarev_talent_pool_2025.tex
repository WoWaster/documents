\documentclass[foldmarks=false, enlargefirstpage=true,
    firstfoot=false, fromphone, fromemail, version=last]{scrlttr2}

%%% Fonts and language setup.
\usepackage{polyglossia}
% Setup fonts.
\usepackage{fontspec}
\setmainfont{CMU Serif}
\setsansfont{CMU Sans Serif}
\setmonofont{CMU Typewriter Text}

\usepackage{microtype} % Add fancy-schmancy font tricks

%%% Polyglossia setup after (nearly) everything as described in documentation.
\setdefaultlanguage{russian}
\setotherlanguage{english}

\usepackage{csquotes}

%%% HyperRef
\usepackage{hyperref}

\newcaptionname{russian}{\phonename}{Телефон}
\newcaptionname{russian}{\emailname}{Почта}

\setkomavar{fromname}{Николай Пономарев}
\setkomavar{fromaddress}{Telegram: \href{https://t.me/wowaster}{\texttt{@wowaster}}}
\setkomavar{fromphone}{+7 XXX XXX-XX-XX}
\setkomavar{fromemail}{\href{mailto:n.ponomarev@spbu.ru}{\texttt{n.ponomarev@spbu.ru}}, \href{mailto:wowasterdev@gmail.com}{\texttt{wowasterdev@gmail.com}}}

\setkomavar{subject}{Мотивационное письмо для отбора в кадровый резерв обучающихся}

\setkomavar{backaddress}{}
\setkomavar{signature}{}

\begin{document}

\begin{letter}{Комисии для отбора в кадровый резерв математико-механического факультета СПбГУ}
    \opening{Уважаемая комиссия!}
    Меня зовут Николай Алексеевич Пономарев, в данный момент я заканчиваю 4~курс бакалавриата по образовательной программе \enquote{Технологии программирования}, а также работаю лаборантом-исследователем в Лаборатории YADRO.

    В этом году я планирую поступить в магистратуру мат-меха для продолжения научных исследований под руководством своего научного руководителя Семена Вячеславовича Григорьева.
    Также я уже не один год помогаю с преподаванием своей кафедре.
    Поэтому мне бы очень хотелось попасть в кадровый резерв родного факультета.

    Заниматься образовательной деятельностью я начал ещё в школе.
    Сначала это была помощь одноклассникам с домашними заданиями или в подготовке к контрольным работам.
    В 11 классе факультативные занятия по подготовке к ЕГЭ по информатике достаточно часто проводились с моей помощью.
    В то же время меня иногда просили провести уроки по информатике у 7--8 классов.

    Поступив на матмех, я не бросил преподавание.
    На первом курсе было небольшое затишье, но уже на втором, взяв учебную практику, я попал в коллектив Лаборатории YADRO.
    С тех пор я успел
    \begin{itemize}
        \item помочь с приёмом домашних заданий по дискретной математике Кириллу Константиновичу Смирнову у первого курса ТехПрога (осень-зима 2022 г.),
        \item побыть ментором на летних проектных школах программирования 2023 и 2024 (июль 2023 г., июль 2024 г.).
        \item рецензировать перевод курса \enquote{Foundations of RISC-V Assembly Programming} и \enquote{Сквозной лабораторный практику по технологиям RISC-V}, разработанный в ЛЭТИ, по просьбе партнёра Лаборатории~--- Альянса RISC-V (осень-зима 2023 г.),
        \item разработать совместно с Ефимом Алексеевичем Кубышкиным систему автоматизированного тестирования домашних заданий для курса по формальным языкам С.~В.~Григорьева (весна 2024 г.),
        \item провести апробацию системы тестирования на магистрах ИТМО и третьекурсниках программной инженерии (весна и осень 2024 г.),
        \item разработать курс по программированию ПЛИС в рамках летней проектной школы совместно с Е.~А.~Кубышкиным и С.~В.~Григорьевым (лето 2024 г.);
        \item провести пары в формате мастер-класса по работе с одноплатными компьютерами на примере BananaPi BPI-F3 для подгрупп Григорьева~С.~В. и Литвинова~Ю.~В. по программированию первого курса ТехПрога (весна 2025 г.).
    \end{itemize}

    Поскольку я попал в Лабораторию в период её создания и становления, на втором и третьем курсе темы моих исследований регулярно менялись, но в последний год направление моих исследований стало достаточно стабильным и перспективным, в контексте его развития в течение магистратуру и, потенциально, аспирантуры.

    В данный момент мы с Ефимом Кубышкиным активно трудимся над проектом Lamagraph, целью которого является разработка параметризуемого многоядерного ускорителя на основе модели вычислений \textenglish{Interaction Nets}.
    Гипотеза, давшая старт проекту, состоит в том, что некоторые задачи в современном мире плохо распараллеливаются на классических архитектурах процессоров и видеокарт, и в таком случае разумно использовать ускорители на альтернативных архитектурах.

    В данный момент создано лишь минимальное рабочее окружение для прототипирования ускорителей и работы с ними.
    Для подтверждения или опровержения гипотезы требуется проделать большой объем работы.
    Его планируется выполнить в магистратуре.

    Дополнительно сообщаю, что являюсь сотрудником приёмной комиссии мат-меха с 2024 года, а также являюсь основным автором программы Образовательного центра системного программирования, занявшей 4 место среди победителей отбора на предоставление грантов на обеспечение обучения студентов по образовательным программам для топ-специалистов в сфере ИТ, проводимого МинЦифры РФ совместно с Аналитическим центром при Правительстве РФ.

    Спасибо за рассмотрение моей кандидатуры.
    Я очень надеюсь продолжить своё сотрудничество с мат-мехом и в лице его преподавателя-исследователя.

    \closing{С уважением, Николай Пономарев.}
\end{letter}

\end{document}
