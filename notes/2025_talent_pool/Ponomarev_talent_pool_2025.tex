\documentclass[foldmarks=false, enlargefirstpage=true,
    firstfoot=false, fromphone, fromemail, version=last]{scrlttr2}

%%% Fonts and language setup.
\usepackage{polyglossia}
% Setup fonts.
\usepackage{fontspec}
\setmainfont{CMU Serif}
\setsansfont{CMU Sans Serif}
\setmonofont{CMU Typewriter Text}

\usepackage{microtype} % Add fancy-schmancy font tricks

%%% Polyglossia setup after (nearly) everything as described in documentation.
\setdefaultlanguage{russian}
\setotherlanguage{english}

\usepackage{csquotes}

%%% HyperRef
\usepackage{hyperref}

\newcaptionname{russian}{\phonename}{Телефон}
\newcaptionname{russian}{\emailname}{Почта}

\setkomavar{fromname}{Николай Пономарев}
\setkomavar{fromaddress}{Telegram: \href{https://t.me/wowaster}{\texttt{@wowaster}}}
\setkomavar{fromphone}{+7 XXX XXX-XX-XX}
\setkomavar{fromemail}{\href{mailto:n.ponomarev@spbu.ru}{\texttt{n.ponomarev@spbu.ru}}, \href{mailto:wowasterdev@gmail.com}{\texttt{wowasterdev@gmail.com}}}

\setkomavar{subject}{Мотивационное письмо для отбора в кадровый резерв обучающихся}

\setkomavar{backaddress}{}
\setkomavar{signature}{}

\begin{document}

\begin{letter}{В комиссию для отбора в кадровый резерв математико-механического факультета СПбГУ}
    \opening{Уважаемая комиссия!}
    Меня зовут Николай Алексеевич Пономарев, в данный момент я заканчиваю 4~курс бакалавриата по образовательной программе \enquote{Технологии программирования}, а также работаю лаборантом-исследователем в Лаборатории YADRO.

    В этом году я планирую поступить в магистратуру мат-меха для продолжения научных исследований под руководством своего научного руководителя Семена Вячеславовича Григорьева.
    Также я уже не один год помогаю с преподаванием своей кафедре.
    Поэтому мне бы очень хотелось попасть в кадровый резерв родного факультета.

    Заниматься образовательной деятельностью я начал ещё в школе.
    Сначала это была помощь одноклассникам с домашними заданиями или в подготовке к контрольным работам.
    К 10--11 классу меня иногда просили провести уроки по информатике у 7--8 классов.

    Поступив на матмех, я не потерял интерес к преподаванию.
    На первом курсе было небольшое затишье, но уже на втором, взяв учебную практику, я попал в коллектив Лаборатории YADRO, который позитивно воспринял моё желание развивать свои педагогические навыки.
    С тех пор я успел поучаствовать в следующих мероприятиях.
    \begin{itemize}
        \item Приёме домашних заданий по дискретной математике К.~К.~Смирнова у первого курса Технологий Программирования (осень-зима 2022 г.).
        \item Летних проектных школах программирования 2023 и 2024 в качестве ментора (июль 2023 г., июль 2024 г.).
        \item Написании рецензий на перевод курса \enquote{Foundations of RISC-V Assembly Programming} и курса \enquote{Сквозной лабораторный практикум по технологиям RISC-V}, разработанного в ЛЭТИ, по просьбе партнёра Лаборатории~--- Альянса RISC-V (осень-зима 2023 г.).
        \item Разработке системы автоматизированного тестирования домашних заданий для курса по формальным языкам С.~В.~Григорьева совместно с Е.~Кубышкиным (весна 2024 г.).
        \item Апробации системы тестирования на магистрантах ИТМО и третьекурсниках Программной Инженерии (весна и осень 2024 г.).
        \item Разработке курса по программированию ПЛИС в рамках летней проектной школы 2024 совместно с Е.~Кубышкиным и С.~В.~Григорьевым (лето 2024 г.).
        \item Провести пары в формате мастер-класса по работе с Linux и одноплатными компьютерами на примере BananaPi BPI-F3 для подгрупп С.~В.~Григорьева и Ю.~В.~Литвинова по программированию первого курса Технологий Программирования (весна 2025 г.).
    \end{itemize}

    Поскольку я попал в Лабораторию в период её создания, на втором и третьем курсе темы моих исследований регулярно менялись, но в последний год направление моих исследований стало достаточно стабильным и перспективным, в контексте его развития в течение магистратуры и, потенциально, аспирантуры.

    В данный момент мы с Ефимом Кубышкиным активно трудимся над проектом Lamagraph, целью которого является разработка параметризуемого многоядерного ускорителя на основе модели вычислений \textenglish{Interaction Nets}.
    Гипотеза, давшая старт проекту, состоит в том, что некоторые задачи в современном мире плохо распараллеливаются на классических архитектурах процессоров и видеокарт, и в таком случае разумно использовать ускорители на альтернативных архитектурах.

    В данный момент создано лишь минимальное рабочее окружение для прототипирования ускорителей и работы с ними.
    Для подтверждения или опровержения гипотезы требуется проделать большой объем работы.
    Его планируется выполнить в магистратуре.

    Кроме того, за время, проведённое на мат-мехе, я успел стать сотрудником приёмной комиссии факультета, а также одним из основных авторов выигравшей заявки на грант МинЦифры РФ для обучения студентов по образовательным программам для топ-специалистов в сфере ИТ.

    Спасибо за рассмотрение моей кандидатуры.
    Я очень надеюсь продолжить своё сотрудничество с мат-мехом и в лице его преподавателя-исследователя.

    \closing{С уважением, Николай Пономарев.}
\end{letter}

\end{document}
