\documentclass[12pt, a4paper, oneside]{memoir}
%%% PDF settings
\pdfvariable minorversion 7 % Set PDF version to 1.7.

%%% Fonts and language setup.
\usepackage{polyglossia}
% Setup fonts.
\usepackage{fontspec}
\setmainfont{CMU Serif}
\setsansfont{CMU Sans Serif}
\setmonofont{CMU Typewriter Text}

\usepackage{microtype} % Add fancy-schmancy font tricks

\usepackage{xcolor} % Add colors support.

%% Math
\usepackage{amsmath, amsfonts, amssymb, amsthm, mathtools} % Advanced math tools.
\usepackage{thmtools}
\usepackage{unicode-math} % Allow TTF and OTF fonts in math and allow direct typing unicode math characters.
\unimathsetup{
    warnings-off={
            mathtools-colon,
            mathtools-overbracket
        }
}
\setmathfont{Latin Modern Math} % default
\setmathfont[range={\setminus,\varnothing,\smashtimes}]{Asana Math}

%%% Images
\usepackage{graphicx}
\graphicspath{{figures/}}
\usepackage{import}

%%% Polyglossia setup after (nearly) everything as described in documentation.
\setdefaultlanguage{russian}
\setotherlanguage{english}

\usepackage{csquotes}

%%% Custom commands
\newcommand{\R}{\mathbb{R}}
\newcommand{\N}{\mathbb{N}}
\newcommand{\Z}{\mathbb{Z}}
\newcommand{\Q}{\mathbb{Q}}
\newcommand{\C}{\mathbb{C}}
\newcommand{\id}{\mathrm{id}}
\AtBeginDocument{\renewcommand{\leq}{\leqslant}}
\AtBeginDocument{\renewcommand{\geq}{\geqslant}}
\AtBeginDocument{\renewcommand{\Re}{\operatorname{Re}}}
\AtBeginDocument{\renewcommand{\Im}{\operatorname{Im}}}
\AtBeginDocument{\renewcommand{\phi}{\varphi}}
\AtBeginDocument{\renewcommand{\epsilon}{\varepsilon}}

%%% theorem-like envs
\theoremstyle{definition}

\declaretheoremstyle[spaceabove=0.5\topsep,
    spacebelow=0.5\topsep,
    headfont=\bfseries\sffamily,
    bodyfont=\normalfont,
    headpunct=.,
    postheadspace=5pt plus 1pt minus 1pt]{myStyle}
\declaretheoremstyle[spacebelow=\topsep,
    headfont=\bfseries\sffamily,
    bodyfont=\normalfont,
    headpunct=.,
    postheadspace=5pt plus 1pt minus 1pt,]{myStyleWithFrame}
\declaretheoremstyle[spacebelow=\topsep,
    headfont=\bfseries\sffamily,
    bodyfont=\normalfont,
    headpunct=.,
    postheadspace=5pt plus 1pt minus 1pt,
    qed=\blacksquare]{myProofStyleWithFrame}

\usepackage[breakable]{tcolorbox}
\tcbset{sharp corners=all, colback=white}
% \tcolorboxenvironment{theorem}{}
% \tcolorboxenvironment{theorem*}{}
% \tcolorboxenvironment{axiom}{}
% \tcolorboxenvironment{assertion}{}
% \tcolorboxenvironment{lemma}{}
% \tcolorboxenvironment{proposition}{}
% \tcolorboxenvironment{corollary}{}
% \tcolorboxenvironment{definition}{}
% \tcolorboxenvironment{proofReplace}{toprule=0mm,bottomrule=0mm,rightrule=0mm, colback=white, breakable }

\declaretheorem[name=Теорема, style=myStyleWithFrame]{theorem}
\declaretheorem[name=Теорема, numbered=no, style=myStyleWithFrame]{theorem*}
\declaretheorem[name=Аксиома, sibling=theorem, style=myStyleWithFrame]{axiom}
\declaretheorem[name=Преположение, sibling=theorem, style=myStyleWithFrame]{assertion}
\declaretheorem[name=Лемма, style=myStyleWithFrame]{lemma}
\declaretheorem[name=Предложение, sibling=theorem, style=myStyleWithFrame]{proposition}
\declaretheorem[name=Следствие, numberwithin=theorem, style=myStyleWithFrame]{corollary}

\declaretheorem[name=Определение, style=myStyleWithFrame]{definition}
\declaretheorem[name=Свойство, style=myStyle]{property}
\declaretheorem[name=Свойства, numbered=no, style=myStyle]{propertylist}

\declaretheorem[name=Пример, style=myStyle]{example}
\declaretheorem[name=Замечание, numbered=no, style=myStyle]{remark}

\declaretheorem[name=Доказательство, numbered=no, style=myProofStyleWithFrame]{proofReplace}
\renewenvironment{proof}[1][\proofname]{\begin{proofReplace}}{\end{proofReplace}}
% \declaretheorem[name=Доказательство, numbered=no, style=myProofStyleWithFrame]{longProof}

%%% Memoir settings
\chapterstyle{ger}
\setlength{\headheight}{2\baselineskip}

%%% HyperRef
\usepackage{hyperref}



\title{Оптимизация библиотеки xxHash для архитектуры RISC-V}
\author{Пономарев Николай}
\date{}


\semiisopage
\begin{document}

\maketitle


RISC-V~--- молодая и активно развивающаяся архитектура. Для экосистемы любой архитектуры очень важно наличие широкоиспользуемых библиотек и оптимизации для них.
xxHash~--- современная библиотека для хеширования, скорость которой обеспечивается использование векторных инструкций процессора.
Так, например, в библиотеке уже имеется поддержка наборов инструкий SSE2, AVX512, NEON.

Однако поддержки векторного расширения RISC-V~--- RVV~--- в библиотеке пока что нет, что и стало целью работы.

Разработка процессоров~--- сложное занятие, и выпущенные процессоры обычно отстают от актуальных спецификаций архитектуры, поэтому существует две несовместимые между собой версии RVV, доступных исследователям:
\begin{itemize}
    \item RVV 1.0~--- официально принятая версия. Тем не менее на данный момент нет ни одной аппаратной платформы, доступной для покупки, где она была бы реализована;
    \item RVV 0.7.1~--- бета-версия расширения, которую можно найти в процессорах от T-Head.
\end{itemize}

В моём распоряжении была плата Lichee Pi 4A с поддержкой RVV 0.7.1, на которой и проводились эксперименты.

Основная идея работы была простой: смотря на уже существующие оптимизации, написать код для RVV для обеих версий RVV с использованием intrinsic функций.
Однако было встречено несколько трудностей:

\begin{itemize}
    \item RVV 0.7.1 поддерживает только форк компилятора GCC от компании T-Head, а современные версии GCC и Сlang поддерживают только RVV 1.0;
    \item Новые версии компиляторов (GCC $\ge$ 13, Clang $\ge$ 16) используют префикс \texttt{\_\_riscv} для intrinsic функций, что при написании кода для обеих версий стандарта требует магии с макросами;
    \item В RVV 0.7.1 отсутствуют некоторые инструкции, присутствующие в RVV 1.0, что увеличивает объем кода.
\end{itemize}

После реализации оптимизированных функций появилась необходимость измерить производительность написанного кода.
Здесь стоит сказать о том, что для выбора набора оптимизаций используются возможности макропроцессора, а также что библиотека имеет встроенный бенчмарк, позволяющий легко сравнивать различные реализации.
Сначала была произведена попытка измерить производительность с помощью QEMU, но она не увенчалась успехом в силу того, что QEMU производит все векторные операции, используя скалярные регистры.
После этого было проведено тестирование на плате.
С его результатами можно ознакомиться в пулл-реквесте.


Сам пулл-реквест можно найти по ссылке: \url{https://github.com/Cyan4973/xxHash/pull/898}.
\end{document}
