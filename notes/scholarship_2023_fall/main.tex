\documentclass[12pt, a4paper, oneside]{memoir}
%%% Обязательные пакеты
%% Beamer
\usepackage{beamerthemesplit}
\usetheme{SPbGU}
\beamertemplatenavigationsymbolsempty
\usepackage{appendixnumberbeamer}

%% Локализация
\usepackage{fontspec}
\setmainfont{CMU Serif}
\setsansfont{CMU Sans Serif}
\setmonofont{CMU Typewriter Text}
%\setmonofont{Fira Code}[Contextuals=Alternate,Scale=0.9]
%\setmonofont{Inconsolata}
% \newfontfamily\cyrillicfont{CMU Serif}

\usepackage{polyglossia}
\setdefaultlanguage{russian}
\setotherlanguage{english}
\usepackage[autostyle]{csquotes} % Правильные кавычки в зависимости от языка

%% Графика
\usepackage{wrapfig} % Позволяет вставлять графику, обтекаемую текстом
\usepackage{pdfpages} % Позволяет вставлять многостраничные pdf документы в текст

%% Математика
\usepackage{amsmath, amsfonts, amssymb, amsthm, mathtools} % "Адекватная" работа с математикой в LaTeX

% Математические окружения с русским названием
\newtheorem{rutheorem}{Теорема}
\newtheorem{ruproof}{Доказательство}
\newtheorem{rudefinition}{Определение}
\newtheorem{rulemma}{Лемма}

%%% Дополнительные пакеты. Используются в презентации, но могут быть отключены при необходимости
\usepackage{tikz} % Мощный пакет для создание рисунков, однако может очень сильно замедлять компиляцию
\usetikzlibrary{decorations.pathreplacing,calc,shapes,positioning,tikzmark}

\usepackage{multirow} % Ячейка занимающая несколько строк в таблице

%% Пакеты для оформления алгоритмов на псевдокоде
\usepackage[noend]{algpseudocode}
\usepackage{algorithm}
\usepackage{algorithmicx}

\usepackage{fancyvrb}

\NewDocumentCommand{\xxHash}{}{\textsc{xxHash}}
\NewDocumentCommand{\riscv}{}{\textsc{RISC-V}}
\NewDocumentCommand{\xxh}{m}{\textsc{XXH{#1}}}
\NewDocumentCommand{\sew}{}{\textsc{SEW}}
\NewDocumentCommand{\vl}{}{\textsc{VL}}
\NewDocumentCommand{\rvv}{}{\textsc{RVV}}
\usepackage{booktabs}
\usepackage{tabularx}
\usepackage{siunitx} % для таблиц с единицами измерений



\title{Оптимизация библиотеки xxHash для архитектуры RISC-V}
\author{Пономарев Николай}
\date{}


\semiisopage
\begin{document}

\maketitle


RISC-V~--- молодая и активно развивающаяся архитектура. Для экосистемы любой архитектуры очень важно наличие широкоиспользуемых библиотек и оптимизации для них.
xxHash~--- современная библиотека для хеширования, скорость которой обеспечивается использование векторных инструкций процессора.
Так, например, в библиотеке уже имеется поддержка наборов инструкий SSE2, AVX512, NEON.

Однако поддержки векторного расширения RISC-V~--- RVV~--- в библиотеке пока что нет, что и стало целью работы.

Разработка процессоров~--- сложное занятие, и выпущенные процессоры обычно отстают от актуальных спецификаций архитектуры, поэтому существует две несовместимые между собой версии RVV, доступных исследователям:
\begin{itemize}
    \item RVV 1.0~--- официально принятая версия. Тем не менее на данный момент нет ни одной аппаратной платформы, доступной для покупки, где она была бы реализована;
    \item RVV 0.7.1~--- бета-версия расширения, которую можно найти в процессорах от T-Head.
\end{itemize}

В моём распоряжении была плата Lichee Pi 4A с поддержкой RVV 0.7.1, на которой и проводились эксперименты.

Основная идея работы была простой: смотря на уже существующие оптимизации, написать код для RVV для обеих версий RVV с использованием intrinsic функций.
Однако было встречено несколько трудностей:

\begin{itemize}
    \item RVV 0.7.1 поддерживает только форк компилятора GCC от компании T-Head, а современные версии GCC и Сlang поддерживают только RVV 1.0;
    \item Новые версии компиляторов (GCC $\ge$ 13, Clang $\ge$ 16) используют префикс \texttt{\_\_riscv} для intrinsic функций, что при написании кода для обеих версий стандарта требует магии с макросами;
    \item В RVV 0.7.1 отсутствуют некоторые инструкции, присутствующие в RVV 1.0, что увеличивает объем кода.
\end{itemize}

После реализации оптимизированных функций появилась необходимость измерить производительность написанного кода.
Здесь стоит сказать о том, что для выбора набора оптимизаций используются возможности макропроцессора, а также что библиотека имеет встроенный бенчмарк, позволяющий легко сравнивать различные реализации.
Сначала была произведена попытка измерить производительность с помощью QEMU, но она не увенчалась успехом в силу того, что QEMU производит все векторные операции, используя скалярные регистры.
После этого было проведено тестирование на плате.
С его результатами можно ознакомиться в пулл-реквесте.


Сам пулл-реквест можно найти по ссылке: \url{https://github.com/Cyan4973/xxHash/pull/898}.
\end{document}
