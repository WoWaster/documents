\documentclass[aspectratio=169]{beamer}

% Setup fonts.
\usepackage{fontspec}
\setmainfont{Yanone Kaffeesatz}
\setsansfont{Noto Sans}
% \setmonofont{Fira Code}
\setmonofont{Iosevka}

%%% Fonts and language setup.
\usepackage{polyglossia}
%% Math
\usepackage{amsmath, amsfonts, amssymb, amsthm, mathtools} % Advanced math tools.
\usepackage{unicode-math} % Allow TTF and OTF fonts in math and allow direct typing unicode math characters.
\unimathsetup{
    warnings-off={
            mathtools-colon,
            mathtools-overbracket
        }
}
\setmathfont{Fira Math}
% \setmathfont{Asana Math}

% \usetheme{Rochester}
% \addtobeamertemplate{frametitle}{\vspace*{-0cm}}{\vspace*{-0.5cm}}
% \usetheme{Madrid}
\usecolortheme{Nord}
\usefonttheme{Nord}
\setbeamertemplate{navigation symbols}{}%remove navigation symbols


\usepackage{minted}
\usemintedstyle{nord}

\title{Сериализация}
\author{Николай Пономарев}
\date{}

%%% Polyglossia setup after (nearly) everything as described in documentation.
\setdefaultlanguage{russian}
\setotherlanguage{english}
\begin{document}
\begin{frame}[plain]
    \maketitle
\end{frame}

\setbeamertemplate{navigation symbols}{%
    \usebeamerfont{footline}%
    \usebeamercolor[fg]{footline}%
    \hspace{1em}%
    \insertframenumber/\inserttotalframenumber
}
\begin{frame}
    \frametitle{Что это и зачем нужно}
    \textbf{Сериализация} --- это преобразования объектов и структур данных в известный формат с возможностью последующего восстановления.
    Процесс обратный сериализации называется \textbf{десериализация}.

    Основные применения:
    \begin{itemize}
        \item сохранение данных в локальное хранилище или базу данных
        \item передача данных, например по сети
    \end{itemize}
\end{frame}

\begin{frame}
    \frametitle{kotlinx.serialization}
    Для Kotlin существует официальная библиотека \texttt{kotlinx.serialization}:
    \begin{itemize}
        \item официальный GitHub: \url{https://github.com/Kotlin/kotlinx.serialization}
        \item краткий гайд на \href{https://kotlinlang.org/docs/serialization.html}{\textit{официальном сайте Kotlin}}
        \item подробная документация на \href{https://github.com/Kotlin/kotlinx.serialization/blob/master/docs/serialization-guide.md}{\textit{GitHub}}
    \end{itemize}
    Кстати, буква \texttt{x} в названии пакета образована от слова \texttt{eXtension}.
    В этот пакет попадают официальные библиотеки, которые не включены в \texttt{stdlib}.
    Например:
    \begin{itemize}
        \item \texttt{kotlinx.coroutines} -- корутины
        \item \texttt{kotlinx.datetime} -- работа с временем
        \item \texttt{kotlinx.cli} -- парсер для CLI
    \end{itemize}
\end{frame}

\begin{frame}
    \frametitle{Форматы}
    Официально библиотека поддерживает:
    \begin{itemize}
        \item JSON
        \item CBOR
        \item Protocol Buffers
        \item Java .properties
        \item HOCON
    \end{itemize}
    Благодаря сообществу так же поддерживаются:
    \begin{itemize}
        \item TOML
        \item XML
        \item YAML
        \item множество других
    \end{itemize}
\end{frame}

\begin{frame}
    \frametitle{JSON}

    \begin{itemize}
        \item JSON (JavaScript Object Notation) -- один из самых популярных и человеко-читаемых форматов.
              Несмотря на то, что он произошел от JavaScript может использоваться в сочетании с любым языком программирования.
        \item В \texttt{kotlinx.serialization} только JSON считается стабильным, поддержка остальных форматов пока что экспериментальна.
        \item Примеры использования: конфигурационные файлы Windows Terminal и VS Code.
        \item Возможно использование схем (schema) -- описания кодируемых данных.
    \end{itemize}
\end{frame}

\begin{frame}
    \frametitle{JSON. Типы}
    JSON поддерживает следующие типы данных:
    \begin{itemize}
        \item \texttt{null}
        \item Number -- число, целые и вещественные числа не различимы, максимального числа так же нет
        \item String -- строка, записывается в двойных кавычках, поддерживает экранирование символов через \texttt{\textbackslash}
        \item Boolean -- принимает \texttt{true} или \texttt{false}
        \item Array -- упорядоченный список с элементами любого типа, записывается через [], разделяется запятыми
        \item Object -- набор вида ключ-значение, где ключ имеет тип String, записывается через \{\}, пары отделяются запятыми, а ключ и значение разделяются двоеточием
    \end{itemize}
\end{frame}

\begin{frame}[fragile]
    \frametitle{JSON. Пример}
    \inputminted[]{json}{serialization_examples/example.json}
\end{frame}

\begin{frame}[fragile]
    \frametitle{JSON Lines}
    Формат -- альтернатива CSV, каждая строка представляет собой валидный JSON объект или массив.
    \inputminted[]{json}{serialization_examples/example1.jsonl}
    или
    \inputminted[]{json}{serialization_examples/example2.jsonl}
\end{frame}

\begin{frame}
    \frametitle{Protocol Buffers}

    \begin{itemize}
        \item Protocol Buffers  -- бинарный формат сериализации, предложенный Google в качестве замены XML.
        \item Для упрощения жизни разработчиков, Protobuf требует наличие схемы.
        \item По умолчанию ни одно поле не может быть \texttt{null}.
        \item Продвигается Google для использования в Android.
    \end{itemize}
\end{frame}
\begin{frame}
    \frametitle{Protobuf. Пример}
    \inputminted[]{protobuf}{serialization_examples/example.proto}
\end{frame}

\begin{frame}
    \frametitle{TOML}
    TOML -- минималистичный формат, похожий на .ini файлы, но с более строгой спецификацией.
    \inputminted[]{toml}{serialization_examples/example.toml}
\end{frame}

\begin{frame}
    \frametitle{YAML}
    YAML -- минималистичный формат. Является надмножеством JSON.
    Используется, например, для настройки GitHub Actions.
    \inputminted[]{yaml}{serialization_examples/example.yaml}
\end{frame}

\begin{frame}[fragile]
    \frametitle{Простой пример. Сериализация}
    \inputminted[]{kotlin}{serialization_examples/serialization1.kt}
    Вывод:
    \inputminted[]{json}{serialization_examples/serialization1.json}
\end{frame}

\begin{frame}[fragile]
    \frametitle{Простой пример. Десериализация}
    \inputminted[]{kotlin}{serialization_examples/serialization2.kt}
    Вывод:
    \inputminted[]{text}{serialization_examples/serialization2.txt}
\end{frame}

\begin{frame}
    \frametitle{Что сериализуется}
    По-умолчанию сериализуются все свойства класса, но с некоторыми особенностями:
    \begin{itemize}
        \item Сериализуются только свойства имеющие backing field и не делегаты
        \item В первичном конструкторе класса могут быть только свойства
        \item Свойства с значением по-умолчанию не будут сериализоваться
        \item Если свойство не нужно сериализовать, его можно обозначить аннотацией \mintinline{kotlin}{@Transient}, у такого свойства должно быть значение по умолчанию
        \item Если свойство -- это ссылка, то объект по ссылке тоже должен иметь аннотацию \mintinline{kotlin}{@Serializable}
        \item Типы generic'ов выводятся во время компиляции
        \item Можно менять название свойств при сериализации с помощью \mintinline{kotlin}{@SerialName(name: String)}
    \end{itemize}
\end{frame}

\begin{frame}[fragile]
    \frametitle{Сложный пример}
    \inputminted[firstline=6, lastline=21]{kotlin}{serialization_examples/hard_example.kt}
\end{frame}
\begin{frame}[fragile]
    \frametitle{Сложный пример}
    \inputminted[firstline=23, lastline=34]{kotlin}{serialization_examples/hard_example.kt}
\end{frame}
\begin{frame}[fragile]
    \frametitle{Сложный пример. Вывод}
    \inputminted[fontsize=\small]{json}{serialization_examples/hard_example.json}
\end{frame}

\begin{frame}
    \frametitle{Что еще можно делать}
    \begin{itemize}
        \item Писать собственные сериализаторы
              \begin{itemize}
                  \item Простой пример: хотим закодировать зеленый цвет \texttt{0x00ff00} читаемо -- не числом.
                        Возможные варианты: строка \texttt{"00ff00"}, массив \texttt{[0,255,0]}, JSON объект \mintinline{json}{{"r":0,"g":255,"b":0}} -- все это возможно
              \end{itemize}
        \item Делать внешние классы сериализуемыми
        \item «Перегружать» сериализаторы с помощью \mintinline{kotlin}{@Contextual}%, чтобы кодировать свойства по разному в зависимости от контекста
        \item Работать с сложной иерархией классов
        \item Настраивать вывод: каждый формат имеет собственные настройки
    \end{itemize}
\end{frame}
\end{document}