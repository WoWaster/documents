\documentclass[aspectratio=169]{beamer}
\usepackage{fontspec}
\setmainfont{Sofia Sans}
\setsansfont{Noto Sans}
\setmonofont{Fira Code}
%%% Fonts and language setup.
\usepackage{polyglossia}
\usepackage[autostyle]{csquotes}
%% Math
\usepackage{amsmath, amsfonts, amssymb, amsthm, mathtools} % Advanced math tools.
\usepackage{unicode-math} % Allow TTF and OTF fonts in math and allow direct typing unicode math characters.
\unimathsetup{
    warnings-off={
            mathtools-colon,
            mathtools-overbracket
        }
}
% \setmathfont{STIX Two Math}
% \newfontfamily{\cyrillicfont}{Fira Math}

\usepackage{metalogox}
\usetheme{Rochester}
\usecolortheme[style=light]{Nord}
\usefonttheme{Nord}

\usepackage{minted}
\usemintedstyle{catppuccin-latte}

\title{\LaTeX}
\subtitle{для самых маленьких}
\author{Николай Пономарев}
\institute{мат-мех СПбГУ}
\date{}

\setbeamertemplate{navigation symbols}{%
    \usebeamerfont{footline}%
    \usebeamercolor[fg]{footline}%
    \hspace{1em}%
    \insertframenumber/\inserttotalframenumber
}

%%% Polyglossia setup after (nearly) everything as described in documentation.
\setdefaultlanguage{russian}
\setotherlanguage{english}

\begin{document}

\begin{frame}
    \titlepage
\end{frame}

\begin{frame}
    \frametitle{Что это за зверь?}
    \TeX~--- система компьютерной верстки, созданная Дональдом Кнутом.

    \LaTeX~--- набор макросов для \TeX, созданный Лесли Лэмпортом для облегчения набора сложных документов в \TeX.

    Основные особенности:
    \begin{itemize}
        \item Использование языка разметки текстового файла, не WYSIWYG~(What You See Is What You Get) система;
        \item Разделение содержания и оформления;
        \item Мощная система перекрёстных ссылок;
        \item \enquote{Нативная} поддержка математических формул;
        \item Переносимость и воспроизводимость.
    \end{itemize}
\end{frame}

\begin{frame}[allowframebreaks]
    \frametitle{Экскурс в историю}
    \begin{description}[3500 лет до Н.~Э.]
        \item[3500 лет до Н.~Э.] Письменность шумеров (потыкать КК)
        \item[...] На самом деле очень много чего ещё
        \item[XI век] АА
    \end{description}
    % \framebreak
    % Основные особенности:
    % \begin{itemize}
    %     \item Использование языка разметки текстового файла, не WYSIWYG~(What You See Is What You Get) система;
    %     \item Разделение содержания и оформления;
    %     \item Мощная система перекрёстных ссылок;
    %     \item \enquote{Нативная} поддержка математических формул;
    %     \item Переносимость и воспроизводимость.
    % \end{itemize}
\end{frame}


\end{document}
