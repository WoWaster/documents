\documentclass[
    aspectratio=169,
    % handout
]{beamer}
% \documentclass[a4paper]{article}
% \usepackage{beamerarticle}

% Setup fonts.
\usepackage{fontspec}
\setmainfont{Iosevka Etoile}
\setsansfont{Iosevka Aile}
\setmonofont{Iosevka}


%%% Fonts and language setup.
\usepackage{polyglossia}

%% Math
\usepackage{amsmath, amsfonts, amsthm, mathtools} % Advanced math tools.
\usepackage{amssymb}
\usepackage{unicode-math} % Allow TTF and OTF fonts in math and allow direct typing unicode math characters.
% \usepackage{euler-math}
\unimathsetup{
    warnings-off={
            mathtools-colon,
            mathtools-overbracket
        }
}
\setmathfont{Fira Math}
% \setmathfont{Asana Math}[
% Scale = MatchUppercase,
% range = {\Coloneq}
% ]
% \setmathfont{GFS Neohellenic Math}
\usepackage{emoji}
\setemojifont{Noto Color Emoji}

\usepackage{csquotes}

\usepackage{booktabs}
\usepackage{colortbl}
\usepackage{tabularx}
\arrayrulecolor{solarizedAccent}
\newcolumntype{Y}{>{\raggedright\arraybackslash}X}
\newcolumntype{Z}{>{\raggedleft\arraybackslash}X}

%%% Beamer
% \usetheme{Madrid}
\usetheme{Boadilla}
% \usetheme{Berlin}
\usecolortheme{solarized}
% \usecolortheme[style=light]{Nord}
% \usefonttheme{Nord}
\setbeamertemplate{navigation symbols}{} %remove navigation symbols
% % \setbeamertemplate{headline}{%
% %     \begin{beamercolorbox}[ht=2.25ex,dp=3.75ex]{section in head/foot}
% %         \insertnavigation{\paperwidth}
% %     \end{beamercolorbox}%
% % }%
\useinnertheme{circles}
\usefonttheme[stillsansserifsmall,stillsansseriftext]{serif}
% \usepackage{appendixnumberbeamer}
\setbeamertemplate{page number in head/foot}[appendixframenumber]
\setbeamercolor{author in head/foot}{fg=solarizedRebase03}
\setbeamercolor{title in head/foot}{fg=solarizedRebase03}
\setbeamercolor{date in head/foot}{fg=solarizedRebase03}
\setbeamercolor{page number in head/foot}{fg=solarizedRebase03}

%%% Misc

\setdefaultlanguage{russian}
\setotherlanguage{english}

\uselanguage{russian}
\languagepath{russian}
\newtranslation[to = russian]{Definition}{Определение}
\newtranslation[to = russian]{definition}{определение}


%%% Code
% \usepackage[kpsewhich,newfloat]{minted}
% \usemintedstyle{solarized-light}
% \SetupFloatingEnvironment{listing}{name=Листинг}

\title[Оптимизация xxHash для RISC-V]{Оптимизация xxHash для RISC-V с использованием различных реализаций RVV}
\author{Николай Пономарев}
\institute[Матмех СПбГУ]{Математико-механический факультет СПбГУ}
\date{20 сентября 2024 г.}

\usepackage{hyperref}

\begin{document}
\begin{frame}[plain,noframenumbering]
    \maketitle
\end{frame}


\begin{frame}
    \frametitle{xxHash}

    xxHash~--- современная библиотека для быстрого хеширования

    Особенности:
    \begin{itemize}
        \item Поддерживает вычисление 32-, 64- и 128-битных хешей
        \item Скорость работы может превышать пропускную способность RAM
        \item Алгоритм поддерживает SIMD и векторные расширения процессоров
              \begin{itemize}
                  \item SSE2, AVX2, AVX512
                  \item NEON, SVE
                  \item Но нет поддержки векторного расширения RISC-V
              \end{itemize}
        \item Использование intrinsic функций для оптимизации
    \end{itemize}

\end{frame}


\begin{frame}
    \frametitle{Векторное расширение RISC-V}

    \begin{itemize}
        \item Векторное расширение RISC-V $\equiv$ RVV
        \item В продаже встречаются три версии:
              \begin{itemize}
                  \item Урезанный RVV 0.7.1~--- реализован в ядрах Xuantie C906 (Sipeed Lichee~RV, MangoPi MQ), \textbf{НЕ} поддерживает 64-битные элементы
                  \item Полноценный RVV 0.7.1~--- в ядрах Xuantie C910 (Sipeed Lichee Pi 4A)
                  \item RVV 1.0~--- в ядрах Xuantie C908 и C920, SpacemiT X60 (Banana Pi BPI-F3)
              \end{itemize}
        \item Программные инструменты не готовы поддерживать RVV 0.7.1
              \begin{itemize}
                  \item Нужно использовать патченные производителем инструменты
              \end{itemize}
    \end{itemize}

\end{frame}

\begin{frame}
    \frametitle{Урезанный RVV 0.7.1}

    \begin{itemize}
        \item Испытуемый: Sipeed Lichee RV с урезанным RVV 0.7.1 и VLEN\,=\,128 бит
        \item \textcolor{solarizedRed}{Проблема}: Алгоритм работает над 64-битными числами
              \begin{description}
                  \item[\textcolor{solarizedGreen}{$\implies$}] будем использовать 32-битные элементы
              \end{description}
        \item \textcolor{solarizedRed}{Проблема}: Битовым операциям всё равно, но что со сложением?
              \begin{description}
                  \item[\textcolor{solarizedGreen}{$\implies$}] будем дополнительно обрабатывать перенос
              \end{description}
    \end{itemize}
\end{frame}

\begin{frame}
    \frametitle{Полноценный RVV 0.7.1}

    \begin{itemize}
        \item Испытуемый: Sipeed LicheePi 4A с полноценным RVV 0.7.1 и VLEN\,=\,128 %бит
        \item Здесь хочется объединить реализации для RVV 0.7.1 и для RVV 1.0
        \item Это возможно, \textcolor{solarizedRed}{но}
              \begin{itemize}
                  \item в апстриме компиляторов у intrinsic функций появился префикс \texttt{\_\_riscv}
                        \begin{description}
                            \item[$\implies$] придётся повозиться с макросами
                        \end{description}
                  \item можно попасться на несуществующую в одной из версий инструкцию
              \end{itemize}
        \item А ещё в GCC 14 есть поддержка XTheadVector, но со сломанной кодогенерацией \emoji{confused}
    \end{itemize}

\end{frame}

\begin{frame}
    \frametitle{RVV 1.0!}

    \begin{itemize}
        \item BPI-F3 с RVV 1.0 и VLEN\,=\,256 бит
        \item Кажется, что счастье!
              \begin{itemize}
                  \item Полная поддержка в апстриме компиляторов
                  \item Автовекторизация (иногда некорректная)
              \end{itemize}
    \end{itemize}

\end{frame}

\begin{frame}
    \frametitle{Бенчмарки}

    \begin{center}
        \begin{tabularx}{\textwidth}{YZZZZ}
            \toprule
                                  & Размер регистра, бит & Скалярная версия, Мб/с & Векторная версия, Мб/с & Ускорение, раз \\
            \midrule
            Урезанный RVV~0.7.1   & 128                  & 169.3                  & 116.0                  & 0.69           \\
            \midrule
            Полноценный RVV~0.7.1 & 128                  & 645.3                  & 472.6                  & 0.73           \\
            \midrule
            RVV 1.0               & 256                  & 516.3                  & 2036.0                 & 3.94           \\
            \midrule
            AVX2                  & 256                  & 21374.8                & 46864.3                & 2.19           \\
            \bottomrule
        \end{tabularx}
    \end{center}

\end{frame}

\begin{frame}
    \frametitle{Выводы}

    \begin{itemize}
        \item Урезанный RVV 0.7.1 применим далеко не для всех алгоритмов
              % \begin{itemize}
              %     \item
              % \end{itemize}
        \item Полноценный RVV 0.7.1 и RVV 1.0 позволяют портировать любые алгоритмы
              \begin{itemize}
                  \item И даже получать настоящее ускорение
              \end{itemize}
        \item Pull Request в xxHash
              \begin{itemize}
                  \item \url{https://github.com/Cyan4973/xxHash/pull/898}
              \end{itemize}
    \end{itemize}

    \ %

    \begin{center}
        \Huge  Спасибо!
    \end{center}
\end{frame}

\end{document}
