\documentclass[
    aspectratio=169,
    % handout
]{beamer}
% \documentclass[a4paper]{article}
% \usepackage{beamerarticle}

% Setup fonts.
\usepackage{fontspec}
\setmainfont{Iosevka Etoile}
\setsansfont{Iosevka Aile}
\setmonofont{Iosevka}


%%% Fonts and language setup.
\usepackage{polyglossia}

%% Math
\usepackage{amsmath, amsfonts, amsthm, mathtools} % Advanced math tools.
\usepackage{amssymb}
\usepackage{unicode-math} % Allow TTF and OTF fonts in math and allow direct typing unicode math characters.
% \usepackage{euler-math}
\unimathsetup{
    warnings-off={
            mathtools-colon,
            mathtools-overbracket
        }
}
\setmathfont{Fira Math}
% \setmathfont{Asana Math}[
% Scale = MatchUppercase,
% range = {\Coloneq}
% ]
% \setmathfont{GFS Neohellenic Math}
\usepackage{csquotes}

\usepackage{booktabs}
\usepackage{colortbl}
\usepackage{tabularx}
\arrayrulecolor{solarizedAccent}
\newcolumntype{Y}{>{\raggedright\arraybackslash}X}
\newcolumntype{Z}{>{\raggedleft\arraybackslash}X}

%%% Beamer
% \usetheme{Madrid}
\usetheme{Boadilla}
% \usetheme{Berlin}
\usecolortheme{solarized}
% \usecolortheme[style=light]{Nord}
% \usefonttheme{Nord}
\setbeamertemplate{navigation symbols}{} %remove navigation symbols
% % \setbeamertemplate{headline}{%
% %     \begin{beamercolorbox}[ht=2.25ex,dp=3.75ex]{section in head/foot}
% %         \insertnavigation{\paperwidth}
% %     \end{beamercolorbox}%
% % }%
\useinnertheme{circles}
\usefonttheme[stillsansserifsmall,stillsansseriftext]{serif}
% \usepackage{appendixnumberbeamer}
\setbeamertemplate{page number in head/foot}[appendixframenumber]

%%% Misc

\setdefaultlanguage{russian}
\setotherlanguage{english}

\uselanguage{russian}
\languagepath{russian}
\newtranslation[to = russian]{Definition}{Определение}
\newtranslation[to = russian]{definition}{определение}


%%% Code
% \usepackage[kpsewhich,newfloat]{minted}
% \usemintedstyle{solarized-light}
% \SetupFloatingEnvironment{listing}{name=Листинг}

\title[Оптимизация xxHash для RISC-V]{Оптимизация xxHash для RISC-V с использованием различных реализаций RVV}
\subtitle{Длинная дорога к ускорению маленькой библиотеки для RISC-V}
\author{Николай Пономарев}
\institute[Матмех СПбГУ]{Математико-механический факультет СПбГУ}
\date{20 сентября 2024 г.}

\usepackage{hyperref}

\begin{document}
\begin{frame}[plain,noframenumbering]
    \maketitle
\end{frame}

% \begin{frame}
%     \frametitle{Обо мне}

%     Надо???

% \end{frame}

\begin{frame}
    \frametitle{С чего всё начиналось (весна 2023)}

    \begin{itemize}[<+->]
        \item Большой интерес лаборатории к векторному расширению RISC-V
        \item И наивное ожидание SBC с поддержкой RVV 1.0
        \item В качестве подопытного~--- библиотека xxHash
              \begin{itemize}
                  \item Алгоритм хеширования, поддерживающий векторизацию
                  \item Готовые реализации для SSE2, AVX, NEON, SVE
                  \item Использование intrinsic функций для оптимизации
              \end{itemize}
    \end{itemize}

\end{frame}


\begin{frame}
    \frametitle{Что умеет RISC-V}

    \begin{itemize}
        \item Векторное расширение RISC-V $\equiv$ RVV
        \item В природе встречается две версии:
              \begin{itemize}
                  \item RVV 0.7.1~--- в ядрах Xuantie C906 (Sipeed Lichee RV, MangoPi MQ) и C910 (Beagle V, Sipeed Lichee Pi 4A)
                  \item RVV 1.0~--- в ядрах Xuantie C908 и C920, SpacemiT X60 (Banana Pi BPI-F3)
              \end{itemize}
        \item<+-> Программные инструменты не готовы поддерживать RVV 0.7.1
    \end{itemize}

    \ %

    \uncover<+->{$\implies$ для работы с RVV 0.7.1 требуются инструменты напрямую от вендора}

\end{frame}

\begin{frame}
    \frametitle{Первые эксперименты~--- QEMU (весна 2023)}

    \begin{itemize}[<+->]
        \item Зачем использовать плату, если есть эмулятор?
        \item Поддержка только RVV 1.0
        \item Векторные операции target архитектуры исполняются на скалярных регистрах host устройства
    \end{itemize}

    \ %

    \uncover<+->{$\implies$ QEMU~--- инструмент тестирования \textbf{корректности}, не быстродействия}

\end{frame}

\begin{frame}
    \frametitle{Первые эксперименты~--- Lichee RV (весна 2023)}

    \begin{itemize}[<+->]
        \item Sipeed Lichee RV с ядрами Xuantie C906
        \item RVV 0.7.1 с поддержкой элементов размером 32 бита и меньше
        \item \textcolor{solarizedRed}{Проблема 1:} чем компилировать?
        \item \textcolor{solarizedGreen}{Решение 1:} будем использовать форк GCC 10 от Xuantie
        \item \textcolor{solarizedRed}{Проблема 2:} xxHash использует элементы по 64 бита
        \item \textcolor{solarizedGreen}{Решение 2:} используем 32-битные элементы
        \item \textcolor{solarizedRed}{Проблема 3:} внутри алгоритма используется сложение, нужно помнить про перенос
        \item \textcolor{solarizedGreen}{Решение 3:} будем таскать за собой перенос, но это потребует масок
    \end{itemize}

    \ %

    \uncover<+->{$\implies$ получим слишком много лишних действий $\implies$ получить нормальную скорость \textbf{невозможно}}

\end{frame}

\begin{frame}
    \frametitle{Полноценный RVV 0.7.1 (осень 2023)}

    \begin{itemize}[<+->]
        \item Sipeed LicheePi 4A с Xuantie C910 и полноценным RVV 0.7.1
        \item Код стал проще и более похож на уже существующий
        \item К этому времени в апстриме компиляторов поменялись названия intrinsic функций
        \item Пришлось использовать макросы для тестирования в QEMU
        \item Однако ускорения не получилось
        \item Возможная проблема~--- дороговизна инструкции перестановки элементов вектора
    \end{itemize}

    \ %

    \uncover<+->{$\implies$ дальнейшие эксперименты были отложены в дальний ящик}

\end{frame}

\begin{frame}
    \frametitle{RVV 1.0! (лето 2024)}

    \begin{itemize}[<+->]
        \item BPI-F3 с SpacemiT X60 и RVV 1.0
        \item Апстримовые компиляторы!
        \item И наконец-то ускорение!
    \end{itemize}

    \ %

    \uncover<+->{$\implies$ потребовалось почти 3 года с момента принятия RVV 1.0, чтобы суметь провести оптимизации для него}
\end{frame}

\begin{frame}
    \frametitle{Бенчмарки}

    \begin{center}
        \begin{tabularx}{\textwidth}{YZZZ}
            \toprule
                                  & Скалярная версия, Мб/с & Векторная версия, Мб/с & Ускорение, раз \\
            \midrule
            Урезанный RVV~0.7.1   & 169.3                  & 116.0                  & 0.69           \\
            \midrule
            Полноценный RVV~0.7.1 & 645.3                  & 472.6                  & 0.73           \\
            \midrule
            RVV 1.0               & 516.3                  & 2036.0                 & 3.94           \\
            \bottomrule
        \end{tabularx}
    \end{center}

\end{frame}

\begin{frame}
    \frametitle{Выводы}

    \begin{itemize}[<+->]
        \item Появление бета версий расширений в доступном \enquote{железе} ведёт к усложнению поддержки кода
        \item Между возможностью скомпилировать код и измерить ускорение может пройти достаточно большое количество времени
        \item Нельзя утверждать, что на другом ядре получится добиться ускорения
    \end{itemize}

    \ %

    \uncover<+->{\centering \Huge  Спасибо!}
\end{frame}

% \appendix

% \begin{frame}
%     \frametitle{Немного про векторные расширения}

%     SIMD и векторные расширения~--- наборы команд, которые позволяют обрабатывать данные "пачками" одновременно. (потом исправлю)

%     Тут надо описать разницу между SIMD и векторами

% \end{frame}

% \begin{frame}
%     \frametitle{Как использовать SIMD/вектора I/III}

%     Рукописный код на ASM

% \end{frame}

% \begin{frame}
%     \frametitle{Как использовать SIMD/вектора II/III}

%     Intrinsics

% \end{frame}

% \begin{frame}
%     \frametitle{Как использовать SIMD/вектора III/III}

%     Автовекторизация

% \end{frame}


\end{document}
