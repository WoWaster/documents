% !BIB program = bibtex
\documentclass[
    aspectratio=169,
    % handout
]{beamer}
% \documentclass[a4paper]{article}
% \usepackage{beamerarticle}

% Setup fonts.
\usepackage{fontspec}
\setmainfont{Iosevka Etoile}
\setsansfont{Iosevka Aile}
\setmonofont{Iosevka}


%%% Fonts and language setup.
\usepackage{polyglossia}

%% Math
\usepackage{amsmath, amsfonts, amsthm, mathtools} % Advanced math tools.
\usepackage{amssymb}

\usepackage{unicode-math} % Allow TTF and OTF fonts in math and allow direct typing unicode math characters.
\unimathsetup{
    warnings-off={
        mathtools-colon,
        mathtools-overbracket
        }
    }
% \setmathfont{Fira Math}
% \setmathfont{Noto Sans Math}[range = {\Coloneq, cal}]
% \setmathfont{Asana Math}
% [
%         Scale = MatchUppercase,
%         range = {\Coloneq, cal}
% ]
% \setmathfont{STIX Two Math}[range = {\Coloneq, cal}]
% \usepackage{euler-math}
\setmathfont{Lete Sans Math}[CharacterVariant={3,6},StylisticSet={4}]

% \usepackage{emoji}
% \setemojifont{Noto Color Emoji}



\usepackage{booktabs}
\usepackage{colortbl}
\usepackage{tabularx}
\arrayrulecolor{solarizedAccent}
\newcolumntype{Y}{>{\raggedright\arraybackslash}X}
\newcolumntype{Z}{>{\raggedleft\arraybackslash}X}

%%% Beamer
% \usetheme{Madrid}
\usetheme{Boadilla}
% \usetheme{Berlin}
\usecolortheme{solarized}
% \usecolortheme[style=light]{Nord}
% \usefonttheme{Nord}
\setbeamertemplate{navigation symbols}{} %remove navigation symbols
\setbeamertemplate{headline}{%
    \begin{beamercolorbox}[ht=2.25ex,dp=3.75ex]{section in head/foot}
        \insertnavigation{\paperwidth}
    \end{beamercolorbox}%
}
\useinnertheme{circles}
\usefonttheme[stillsansserifsmall,stillsansseriftext]{serif}
% \usepackage{appendixnumberbeamer}
\setbeamertemplate{page number in head/foot}[appendixframenumber]
\setbeamercolor{author in head/foot}{fg=solarizedRebase03}
\setbeamercolor{title in head/foot}{fg=solarizedRebase03}
\setbeamercolor{date in head/foot}{fg=solarizedRebase03}
\setbeamercolor{page number in head/foot}{fg=solarizedRebase03}

%%% Misc

\setdefaultlanguage{russian}
\setotherlanguage{english}

\uselanguage{russian}
\languagepath{russian}
\newtranslation[to = russian]{Definition}{Определение}
\newtranslation[to = russian]{definition}{определение}

%% Tikz
\usepackage{tikz}
\usetikzlibrary{arrows.meta}
\usetikzlibrary{external}
\usetikzlibrary{positioning}
\usetikzlibrary{shapes.geometric}
\usetikzlibrary{automata}
\usetikzlibrary{decorations.pathmorphing}
\usetikzlibrary{calc}
\tikzsetexternalprefix{figures/}
\tikzexternalize

%%% Code
\usepackage[kpsewhich,newfloat]{minted}
\usemintedstyle{solarized-light}
% \SetupFloatingEnvironment{listing}{name=Листинг}

\usepackage[backend=biber, style=alphabetic]{biblatex}
\renewcommand*{\bibfont}{\scriptsize}
\addbibresource{rpq-cfpq.bib}

\usepackage{csquotes}

\title[Запросы в терминах формальных языков]{Запросы к графовым базам данных в терминах формальных языков}
\subtitle{В основном про RPQ и немного больше}
\author{Николай Пономарев}
\institute[Матмех СПбГУ]{Математико-механический факультет СПбГУ}
\date{8 октября 2024 г.}

\usepackage{hyperref}

\begin{document}
\begin{frame}[plain,noframenumbering]
    \maketitle
\end{frame}

\section{(Не)много  математики}

\begin{frame}
    \frametitle{Иерархия Хомского}

    Небольшое напоминание:
    \begin{columns}
        \begin{column}{0.5\textwidth}
            \begin{itemize}
                \item Рекурсивно перечислимые языки.
                      Продукции вида: $\gamma \to \alpha$
                \item Контекстно-зависимые языки.
                      Продукции вида: $\alpha A \beta \to \alpha \gamma \beta$
                \item Контекстно-свободные языки.
                      Продукции вида: $A \to \alpha \gamma \beta$
                \item Регулярные языки.
                      Продукции вида: $A \to a$, $A \to aB$
            \end{itemize}
        \end{column}
        \begin{column}{0.5\textwidth}
            % Taken from https://tex.stackexchange.com/a/484557
            \begin{tikzpicture}[font=\sffamily\small,breathe dist/.initial=0.3ex]
                \foreach \X [count=\Y,remember=\Y as \LastY] in
                    {regular,context free,context sensitive,recursively enumerable}
                    {\ifnum\Y=1
                            \node[ellipse,draw,outer sep=0pt] (F-\Y) {\X};
                        \else
                            \node[anchor=south] (T-\Y) at (F-\LastY.north) {\X};
                            \path let \p1=($([yshift=\pgfkeysvalueof{/tikz/breathe dist}]T-\Y.north)-(F-\LastY.south)$),
                            \p2=($(F-1.east)-(F-1.west)$),\p3=($(F-1.north)-(F-1.south)$)
                            in ($([yshift=\pgfkeysvalueof{/tikz/breathe dist}]T-\Y.north)!0.5!(F-\LastY.south)$)
                            node[minimum height=\y1,minimum width={\y1*\x2/\y3},
                                    draw,ellipse,inner sep=0pt] (F-\Y){};
                        \fi}
            \end{tikzpicture}
        \end{column}
    \end{columns}
    \vspace{1em}
    Обозначения: $a$~--- терминал, $A, B$~--- нетерминалы, $\alpha, \beta, \gamma$~--- цепочки терминалов и/или нетерминалов
\end{frame}

\begin{frame}
    \frametitle{Наш главный объект изучения~--- граф}

    \begin{definition}[из \cite{shemetova_one_2021}]
        Реберно-меченный ориентированный мультиграф~--- это $\mathcal{G} = (V, E, \Sigma)$, где
        \begin{itemize}
            \item $V$~--- конечное множество вершин, обычно от 0 до $|V| - 1$
            \item $E \subseteq V \times \Sigma \times V $~--- конечное множество ребер
            \item $\Sigma$~--- конечный алфавит меток
        \end{itemize}
    \end{definition}

\end{frame}

\begin{frame}
    \frametitle{Знакомьтесь, реберно-меченный мультиорграф}

    \begin{center}
        \begin{tikzpicture}[
                shorten >=1pt,
                auto]
            \node[state] (q0) {$0$};
            \node[state] (q1) [right = of q0] {$1$};
            \node[state] (q2) [right = of q1] {$2$};

            \path[->] (q0) edge node {$a$} (q1)
            (q1) edge node {$c$} (q2)
            (q1) edge[out=120, in=60, looseness=10] node {$b$} (q1)
            (q1) edge[bend right] node[below] {$d$} (q2);
        \end{tikzpicture}
    \end{center}

    \[\mathcal{G} = (V = \{0, 1, 2\}, E = \{(0, a, 1), (1, b, 1), (1, c, 2), (1, d, 2)\},\Sigma = \{a, b, c, d\})\]
\end{frame}

\begin{frame}
    \frametitle{Ещё немного определений про графы}

    \begin{definition}
        Путём $\pi$ в графе $\mathcal{G} = (V, E, \Sigma)$ называется последовательность $e_0, e_1, \dots, e_{n-1}$, где $e_i = (v_i, \sigma_i, u_i) \in E$, и для всех $e_i, e_{i+1}$ выполнено $u_i = v_{i+1}$.
        Будем обозначать путь из $v$ в $u$ как $v \pi u$.
    \end{definition}

    \begin{definition}
        Словом, образованным путём $\pi = (v_0, \sigma_0, u_0), (v_1, \sigma_1, u_1), \dots, (v_{n-1}, \sigma_{n-1}, u_{n-1})$, будем называть конкатенацию всех меток этого пути:
        \[\lambda(\pi) = \sigma_0 \sigma_1 \dots \sigma_{n-1}\]
    \end{definition}

\end{frame}

\begin{frame}
    \frametitle{А теперь про формальные языки}

    \begin{definition}[из \cite{mendelzon_finding_1995}] % ИЗ FINDING REGULAR SIMPLE PATHS IN GRAPH DATABASES
        Пусть $\Sigma$~--- конечный алфавит, не содержащий $\{\epsilon, \varnothing, (, )\}$.
        Регулярное выражение $R$ над $\Sigma$ определено следующим образом:
        \begin{itemize}
            \item Пустая строка $\epsilon$, пустое множество $\varnothing$, и все $a \in \Sigma$ являются регулярными выражениями
            \item Если $A$ и $B$~--- регулярные выражения, то $(A + B)$ (альтернатива), $AB$ (конкатенация), $A^*$ (звезда Клини) тоже регулярные выражения
            \item Ничего другого регулярным выражением не является
        \end{itemize}
    \end{definition}

\end{frame}

\begin{frame}
    \frametitle{И ещё про формальные языки}

    \begin{definition}
        Язык $\mathcal{L}(R)$, описываемый $R$, определяется следующим образом:
        \begin{enumerate}
            \item $\mathcal{L}(\epsilon) = \{\epsilon\}$
            \item $\mathcal{L}(\varnothing) = \varnothing$
            \item $\mathcal{L}(a) = \{a\}$ $\forall a \in \Sigma$
            \item $\mathcal{L}(A + B) = \mathcal{L}(A) \cup \mathcal{L}(B) = \{ w : w \in \mathcal{L}(A) \lor w \in \mathcal{L}(B)\}$
            \item $\mathcal{L}(AB) = \mathcal{L}(A) \mathcal{L}(B) = \{ w_1 w_2 : w_1 \in \mathcal{L}(A) \land w_2 \in \mathcal{L}(B)\}$
            \item $\mathcal{L}(A^*) = \bigcup_{k=0}^\infty \mathcal{L}(A)^k$, где $\mathcal{L}(A)^0 = \{\epsilon\}$ и $\mathcal{L}(A)^k = \mathcal{L}(A)^{k-1} \mathcal{L}(A)$
        \end{enumerate}
    \end{definition}

\end{frame}

\section{RPQ}

\begin{frame}
    \frametitle{Постановка задачи RPQ}

    \begin{definition}
        Пусть $\mathcal{G} = (V, E, \Sigma)$~--- реберно-меченный мультиорграф, $R$~--- регулярное выражение над $\Sigma$.
        Тогда решением \textit{задачи достижимости с ограничениями в терминах регулярных языков} для $\mathcal{G}$ и $R$ является множество
        \[ \Bigl\{(u, v) \in V \times V : \exists \text{ путь } \pi \text{ в } \mathcal{G} \text{ из } u \text{ в } v: \lambda(\pi) \in \mathcal{L}(R)\Bigr\} \]
    \end{definition}
    \begin{itemize}
        \item На английском, Regular Path Query (RPQ)
        \item Часто назначают стартовые и конечные вершины
    \end{itemize}
\end{frame}

\begin{frame}
    \frametitle{Пример RPQ (из \cite{nole_regular_2016})} % ИЗ Regular Path Queries on Massive Graphs

    \begin{columns}
        \begin{column}{0.5\textwidth}
            \begin{center}
                \begin{tikzpicture}[
                        shorten >=1pt,
                        auto]
                    \node[state] (q0) {$0$};
                    \node[state] (q1) [below left = of q0] {$1$};
                    \node[state] (q2) [below right = of q0] {$2$};
                    \node[state] (q3) [below right = of q1] {$3$};
                    \node[state] (q4) [below right = of q2] {$4$};
                    \node[state] (q5) [below left = of q3] {$5$};
                    \node[state] (q6) [below right = of q3] {$6$};

                    \node (lab) [left = of q0] {$\mathcal{G}$:};

                    \path[->] (q0) edge node[above left] {$a$} (q1)
                    (q2) edge node[above right] {$a$} (q0)
                    (q1) edge node[below left] {$a$} (q3)
                    (q3) edge node[below right] {$a$} (q2)
                    (q3) edge node[above left] {$b$} (q5)
                    (q3) edge node[above right] {$c$} (q6)
                    (q2) edge node[above right] {$d$} (q4);
                \end{tikzpicture}
            \end{center}
        \end{column}
        \begin{column}{0.5\textwidth}
            \begin{gather*}
                R_1 = (a + aa)(b + d) \\
                \mathcal{L}\left(R_1\right) = \{ab, aab, ad, aad\} \\
                \mathrm{RPQ}\left(\mathcal{G}, R_1\right) = \bigl\{(0, 5), (1, 5), (1, 4), (3, 4)\bigr\}
            \end{gather*}

            \begin{gather*}
                R_2 = (aaaa)^* (b + c + d) \\
                \begin{aligned}
                    \mathcal{L}(R_2) = \Biggl\{ w_1 w_2 & : w_1 \in \bigcup_{k=0}^\infty (aaaa)^k \\
                                                        & \land  w_2 \in \{ b, c ,d \} \Biggr\}
                \end{aligned} \\
                \mathrm{RPQ}\left(\mathcal{G}, R_2\right) = \bigl\{(3, 5), (3, 6), (2, 4)\bigr\}
            \end{gather*}
        \end{column}
    \end{columns}

\end{frame}

\begin{frame}
    \frametitle{Как это вычислять?}

    \begin{enumerate}
        \item Пересечение конечных автоматов
              \begin{itemize}
                  \item Обычно решает задачу для всех вершин
                  \item Реализация почти очевидна из названия, немного есть в \cite{shemetova_one_2021}
              \end{itemize}
        \item Синхронный обход в ширину конечного автомата и графа
              \begin{itemize}
                  \item Часто решает задачу для конкретных стартовых, конечных вершин
                  \item Способ умнее, почитать можно в \cite{elekes_graphblas_2020}
              \end{itemize}
    \end{enumerate}

\end{frame}

\begin{frame}
    \frametitle{Кто это поддерживает? I}

    Графовые базы данных
    \begin{itemize}
        \item Начало развития теории~--- конец 80-х (см. \cite[35]{bonifati_querying_2018})
        \item Ответственный за популярность~--- Neo4j (2007)
        \item Сейчас главное~--- язык запросов
              \begin{itemize}
                  \item openCypher \cite{neo4j_opencypher_nodate}: Neo4j, Amazon Neptune, ArcadeDB, Redis Graph, VK Tarantool
                  \item SPARQL \cite{w3c_sparql_nodate}: Amazon Neptune, Eclipse RDF4J, Apache Jena
                        \begin{itemize}
                            \item Стандарт W3C
                        \end{itemize}
                  \item Apache TinkerPop's Gremlin \cite{apache_tinkerpop_apache_nodate}: Neo4j, JanusGraph, Amazon Neptune, ArcadeDB
                  \item GQL \cite{isoiec_39075_information_nodate}: на данный момент $\varnothing$
                        \begin{itemize}
                            \item Стандарт ISO 2024 года!
                            \item Унификация всего, что выше и не только
                        \end{itemize}
              \end{itemize}
    \end{itemize}


\end{frame}

\begin{frame}
    \frametitle{Кто это поддерживает? II}
    Реляционные базы данных
    \begin{itemize}
        \item Стандарт SQL/PGQ (property graph queries)
        \item Разработан Oracle, принят в ISO SQL:2023 \parencite{melton_information_2023}
        \item Посмотреть как это выглядит можно в \cite{deutsch_graph_2021}
    \end{itemize}

\end{frame}

\begin{frame}
    \frametitle{Области применения RPQ (по \cite{garcia_path_2024,bonifati_querying_2018})}

    \begin{itemize}
        \item Графы топологии сетей
        \item Социальные сети
        \item Биология
    \end{itemize}

\end{frame}

\section{CFPQ и Datalog}

\begin{frame}
    \frametitle{А можно круче?}

    Если хочется ещё более мощной системы, то есть
    \begin{enumerate}
        \item CFPQ
        \item Datalog
    \end{enumerate}

\end{frame}

\begin{frame}
    \frametitle{CFPQ}

    \begin{columns}
        \begin{column}{0.5\textwidth}
            \begin{itemize}
                \item Используем вместо регулярных языков КС-языки
                \item Внезапно, оно довольно быстро работает (\cite{_optimisation_2024})
                \item Но нет спроса у пользователей
                \item Известное применение~--- статический анализ (\cite{yamaguchi_modeling_2014})
            \end{itemize}
        \end{column}
        \begin{column}{0.5\textwidth}

            \begin{center}
                \resizebox{0.95\textwidth}{!}{%
                    \begin{tikzpicture}[shorten >=1pt,
                            auto]
                        \node[] (1) {$\&O$};
                        \node[] (2) [right = of 1] {$v4$};
                        \node[] (3) [right = of 2] {$*v2$};
                        \node[] (4) [below = of 3] {$v2$};
                        \node[] (5) [below = of 4] {$\&v2$};
                        \node[] (6) [right = of 5] {$v1$};
                        \node[] (7) [below = of 6] {$\&v1$};
                        \node[] (8) [right = of 7] {$v3$};
                        \node[] (9) [above = of 8] {$*v3$};
                        \node[] (10) [right = of 9] {$v5$};
                        \node[] (11) [above = of 10] {$*v5$};
                        \node[] (12) [left = of 11] {$v6$};
                        \node[] (13) [above = of 12] {$*v6$};
                        \draw[->] (1) -- node[midway, above, sloped] {a} (2);
                        \draw[->] (2) -- node[midway, above, sloped] {a} (3);
                        \draw[dashed,->] (4) -- node[midway, left] {d} (3);
                        \draw[dashed,->] (5) -- node[midway, left] {d} (4);
                        \draw[->] (5) -- node[midway, above] {a} (6);
                        \draw[dashed,->] (7) -- node[midway, left] {d} (6);
                        \draw[->] (7) -- node[midway, above] {a} (8);
                        \draw[dashed,->] (8) -- node[midway, right] {d} (9);
                        \draw[->] (9) -- node[midway, above] {a} (10);
                        \draw[dashed,->] (10) -- node[midway, right] {d} (11);
                        \draw[->] (11) -- node[midway, above] {a} (12);
                        \draw[dashed,->] (12) -- node[midway, left] {d} (13);

                        \draw[dotted,<->] (1) to [out=45, in=135, looseness=1] node[midway, above] {VA} (2);
                        \draw[dotted,<->] (2) to [out=45, in=135, looseness=1] node[midway, above] {VA} (3);
                        \draw[dotted,<->] (1) to [out=-45, in=-135, looseness=1] node[midway, below] {VA} (3);
                        \draw[dotted,<->,red] (3) to [out=45, in=135, looseness=1] node[midway, above,red] {MA} (13);
                        \draw[dotted,<->] (4) to [out=45, in=135, looseness=1] node[midway, above] {VA} (12);
                        \draw[dotted,<->,red] (4) to [out=-30, in=-150, looseness=0.3] node[midway, above,red] {MA} (11);
                        \draw[dotted,<->] (5) to [out=30, in=150, looseness=0.6] node[midway, above] {VA} (10);
                        \draw[dotted,<->] (5) to [out=-30, in=-150, looseness=0.5] node[midway, below] {VA} (6);
                        \draw[dotted,<->,red] (6) to [out=30, in=150, looseness=0.7] node[midway, below,red] {MA} (9);
                        \draw[dotted,<->] (7) to [out=45, in=135, looseness=1] node[midway, above] {VA} (8);
                        \draw[dotted,<->] (9) to [out=-30, in=-150, looseness=0.5] node[midway, below] {VA} (10);
                        \draw[dotted,<->] (12) to [out=30, in=150, looseness=1] node[midway, above] {VA} (11);
                    \end{tikzpicture}
                }
            \end{center}
            \begin{flushright}
                \scriptsize
                Рисунок из \cite{_experimental_2023}
            \end{flushright}
        \end{column}
    \end{columns}
\end{frame}

\begin{frame}[fragile]
    \frametitle{Datalog}

    \begin{columns}
        \begin{column}{0.5\textwidth}
            \begin{itemize}
                \item Формально, это уже про дедуктивные БД
                \item Datalog~--- де факто стандарт
                \item В нём выразим RPQ (\cite{green_datalog_2013})
                \item И даже больше
            \end{itemize}
        \end{column}
        \begin{column}{0.5\textwidth}
            \begin{minted}[fontsize=\scriptsize]{prolog}
r1 reachable(X,Y) :- link(X,Y).
r2 reachable(X,Y) :- link(X,Z), reachable(Z,Y).
query(X,Y) :- reachable(X,Y).
            \end{minted}
        \end{column}
    \end{columns}
\end{frame}

\section{Вопросы к зачёту}
\begin{frame}
    \frametitle{Вопросы к зачёту}

    \begin{enumerate}
        \item Постановка задачи RPQ.
              Пример решения задачи для некоторого графа и регулярного выражения.
        \item Примеры языков запросов и ПО, использующего их.
              Области использования RPQ.
        \item CFPQ и Datalog, их отношение к RPQ.
    \end{enumerate}

\end{frame}

\appendix
\section{Список литературы}
\setbeamertemplate{headline}{}
\begin{frame}[allowframebreaks, plain]
    \frametitle{Источники}
    \printbibliography[heading=none]
\end{frame}


\end{document}
