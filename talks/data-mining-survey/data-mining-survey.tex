% \documentclass[aspectratio=169]{beamer}
% \documentclass[a4paper]{article}
% \usepackage{beamerarticle}

% Setup fonts.
\usepackage{fontspec}
\mode<beamer>{
\setmainfont{Yanone Kaffeesatz}
\setsansfont{Roboto}
\setmonofont{Fira Code}
}
\mode<article>{
\setmainfont{CMU Serif}
\setsansfont{CMU Sans Serif}
\setmonofont{CMU Typewriter Text}
}

%%% Fonts and language setup.
\usepackage{polyglossia}

%% Math
\usepackage{amsmath, amsfonts, amsthm, mathtools} % Advanced math tools.
\usepackage{amssymb}
\usepackage{unicode-math} % Allow TTF and OTF fonts in math and allow direct typing unicode math characters.
% \usepackage{euler-math}
\unimathsetup{
    warnings-off={
            mathtools-colon,
            mathtools-overbracket
        }
}
\mode<beamer>{
\setmathfont{Fira Math}
\setmathfont{Asana Math}[
Scale = MatchUppercase,
range = {\Coloneq}
]
}
\mode<article>{
\setmathfont{Latin Modern Math} % default
\setmathfont[range={\setminus,\varnothing,\smashtimes}]{Asana Math}
}
\usepackage{csquotes}

%%% Beamer
\usetheme{Madrid}
\usecolortheme[style=light]{Nord}
\usefonttheme{Nord}
\setbeamertemplate{navigation symbols}{} %remove navigation symbols
% \setbeamertemplate{headline}{%
%     \begin{beamercolorbox}[ht=2.25ex,dp=3.75ex]{section in head/foot}
%         \insertnavigation{\paperwidth}
%     \end{beamercolorbox}%
% }%
\useinnertheme{circles}



%%% Misc

\setdefaultlanguage{russian}
\setotherlanguage{english}

\uselanguage{russian}
\languagepath{russian}
\newtranslation[to = russian]{Definition}{Определение}
\newtranslation[to = russian]{definition}{определение}


%%% Code
% \usepackage[kpsewhich,newfloat]{minted}
% \usemintedstyle{catppuccin-latte}
% \SetupFloatingEnvironment{listing}{name=Листинг}

\title{Data mining: задачи, техники и приложения}
\author{Николай Пономарев}
\institute[Матмех СПбГУ]{Математико-механический факультет СПбГУ}
\date{23 мая 2024 г.}

\usepackage{hyperref}

\begin{document}
\begin{frame}[plain,noframenumbering]
    \maketitle
\end{frame}

\textbf{Дисклеймер:} данный текст просто пересобранная презентация, саму статью можно найти в конце

\section{Введение}

\begin{frame}
    \frametitle{Немного о терминологии}

    \uncover<+->{Две близких области: Information Retrieval~(IR) и Data Mining~(DM)}

    \begin{definition}<+->
        IR~--- наука об организации информации и алгоритмах её \textit{быстрого} получения
    \end{definition}

    \begin{definition}<+->
        DM~--- наука о методах обнаружения в данных ранее \textit{неизвестных} и практически полезных знаний
    \end{definition}

    \uncover<+->{Формально, наш курс про IR}

\end{frame}

\section{Задачи}

\begin{frame}[allowframebreaks]
    \frametitle{Задачи}
    \begin{definition}
        Классификация~(classification)~--- процесс разделения новых наблюдений~(observation) на предопределенные классы
    \end{definition}

    \begin{definition}
        Кластеризация~(clustering)~--- процесс разбиения данных (или наблюдений о них) на группы
    \end{definition}

    \mode<beamer>{\pagebreak}

    \begin{definition}
        Анализ выбросов~(outlier analysis)~--- способ извлечения полезных знаний из выбросов
    \end{definition}

    \begin{definition}
        Ассоциативный анализ~(association analysis)~--- процесс поиска ассоциаций среди данных, которые удовлетворяют определенным статистическим требованиям
    \end{definition}

\end{frame}

\section{Техники}

\begin{frame}
    \frametitle{Статистические подходы}

    Строится статистическая модель, а затем анализируется следующими способами:

    \begin{description}[Байесовские сети]
        \item[Байесовские сети] способ выяснения зависимостей между переменными \enquote{с помощью} формулы Байеса
        \item[Корреляция] используется для установления зависимости между фактами
        \item[Регрессия] установление соответствия между случайными переменными отражающими связь между зависимой переменной и независимыми переменными х
        \item[Факторный анализ] используется для поиска основных источников корреляции
    \end{description}

\end{frame}

\begin{frame}
    \frametitle{Машинное обучение и нейронные сети}

    \begin{itemize}
        \item Развивает идеи статистических методов
        \item В каком-то смысле автоматизирует анализ
        \item Часто результаты лучше
        \item Модно
    \end{itemize}

\end{frame}

\begin{frame}
    \frametitle{Ещё техники}

    \begin{itemize}
        \item СУБД
        \item Генетические алгоритмы
        \item Нечёткие множества
        \item Визуализация
    \end{itemize}

\end{frame}

\section{Приложения}
\begin{frame}
    \frametitle{Телекоммуникации}

    Телеком и мобильные операторы используют data mining для
    \begin{itemize}
        \item Маркетинга
              \begin{itemize}
                  \item Классификация и кластеризация $\Rightarrow$ таргетированная реклама
              \end{itemize}
        \item Удержания клиентов
              \begin{itemize}
                  \item Классификация и кластеризация $\Rightarrow$ обнаружение недовольных клиентов
              \end{itemize}
        \item Создание оптимальных тарифов
        \item Оптимизация использования инфраструктуры
    \end{itemize}

\end{frame}

\begin{frame}
    \frametitle{Продажи}

    Торговым предприятиям нужен DM для исследования
    \begin{itemize}
        \item Поведения покупателей
              \begin{itemize}
                  \item Ассоциативный анализ
              \end{itemize}
        \item Корзины
        \item Выбора товаров
        \item Расстановки продуктов на полках
              \begin{itemize}
                  \item Кластеризация
              \end{itemize}
        \item Влияния акций
    \end{itemize}

\end{frame}

\begin{frame}
    \frametitle{Медицина}

    В медицине DM используется для
    \begin{itemize}
        \item Обнаружения и анализа хронических заболеваний
        \item Поиск эффективных лекарств
        \item Отслеживать вероятность эпидемий
    \end{itemize}

\end{frame}

\begin{frame}
    \frametitle{Зачем ещё это надо?}

    \begin{itemize}
        \item Финансы
        \item Предотвращение преступлений
        \item Рекомендательные системы
        \item Реклама
    \end{itemize}

\end{frame}

\appendix
\begin{frame}
    \frametitle{Источник}
    \begin{thebibliography}{}
        \bibitem{main}
        M. K. Gupta \& P. Chandra \newblock
        A comprehensive survey of data mining \newblock
        International Journal of Information Technology, 12 (2020) 1243–1257 \newblock
        \url{https://doi.org/10.1007/s41870-020-00427-7}
    \end{thebibliography}
\end{frame}

\end{document}
