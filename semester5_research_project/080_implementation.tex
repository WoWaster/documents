% !TeX spellcheck = ru_RU
% !TEX root = vkr.tex

\section{Реализация}

В данном разделе описывается процесс компиляции приложений, с различными системами сборки, на примерах проектов с открытым исходным кодом.

\remark{Считаем, что в системе есть тулчейн из репозиторией, qemu с binfmt.}
% TODO: ?
% \subsection{! (Make)}

\subsection{Libusb (Autotools)}

\libusb{}\footnote{\url{https://github.com/libusb/libusb}}~--- небольшая библиотека, предоставляющая универсальный API для доступа к USB устройствам на различных операционных системах.
В качестве системы сборки использует \autotools{}.

Для сборки потребуется создать sysroot c необходимыми зависимостями.
Для его создания будем использовать утилиту \debootstrap{}, которая умеет создавать sysroot \debian{}-подобных дистрибутивов.

Для удобства будем считать, что путь к sysroot хранится в переменной среды \sysrootPath{}, тогда sysroot для сборки \libusb{} можно создать командой
\begin{minted}[breaklines]{console}
# debootstrap --arch=riscv64 --include=libudev-dev,pkg-config,umockdev,libumockdev-dev unstable $SYSROOT_PATH http://ftp.debian.org/debian
\end{minted}
Флаг \texttt{--arch} указывает на архитектуру создаваемого sysroot, флаг \texttt{--include} позволяет указать необходимые пакеты, которые необходимо установить, помимо базовой системы.
После этого передается кодовое название версии дистрибутива, в данном случае используется \debian{} \textsc{Unstable}, так как текущая стабильная версия~--- \textsc{Bookworm}~--- не поддерживает архитектуру \riscv{}.
Последними двумя аргументами являются путь, где будет создан sysroot, и ссылка на зеркало, с которого будут скачиваться пакеты.

Далее для сборки достаточно выполнить:
\begin{minted}[breaklines]{console}
$ ./bootstrap.sh
$ mkdir build && cd build
$ CC="riscv64-linux-gnu-gcc --sysroot=$SYSROOT_PATH" ../configure --enable-examples-build --enable-tests-build --enable-udev --host=riscv64-linux-gnu
$ make
\end{minted}

Флаг \texttt{--host} указывает triple для кросс-компиляции, а для передачи sysroot переопределяется переменная окружения \texttt{CC}.
В данном проекте файл \texttt{configure} поддерживает флаг в \texttt{--with-sysroot}, но, к сожалению, он ничего не делает.

Другим способом передать путь к sysroot будет добавление его непосредственно в переменную \texttt{CFLAGS}:
\begin{minted}[breaklines]{console}
$ CFLAGS="--sysroot=$SYSROOT_PATH" ../configure --enable-examples-build --enable-tests-build --enable-udev --host=riscv64-linux-gnu
\end{minted}

\subsection{SingleApplication (CMake)}

\singleapp{}\footnote{\url{https://github.com/itay-grudev/SingleApplication/}}~--- компонент для \qt{}, позволяющий создавать приложения, у которых одновременно может запущен только один экземпляр.

Проект поддерживает сборку как с помощью \qmake{}, и с помощью \cmake{}.
В данном разделе рассмотрим сборку с помощью второго.

Для кросс-компиляции с помощью \cmake{} необходимо создать toolchain файл\footnote{\url{https://cmake.org/cmake/help/book/mastering-cmake/chapter/Cross\%20Compiling\%20With\%20CMake.html}} примерно такого содержания

\remark{И тут всё сломалось...}

\remark{Пример тулчейн файла + фикс moc}

\subsection{Trik Runtime (QMake)}

\remark{Docker + chroot}

\subsection{Umockdev (Meson)}

\remark{Пример тулчейн файла для Meson + personality для pkgconf}

\subsection{Trik Runtime (Jenkins)}

\remark{Jenkins + конфиг ядра для VF2 для докера }

% \subsection{GitLab}

% Make
% При необходимости передачи sysroot, он может быть дописан к пути компилятора, например
% \begin{minted}[breaklines]{console}
% CC="riscv64-linux-gnu-gcc --sysroot=/sysroot" make ...
% \end{minted}
% либо к аргументам компилятора, например
% \begin{minted}[breaklines]{console}
% CFLAGS="--sysroot=/sysroot" make ...
% \end{minted}

% Meson
% Выяснить эти проще всего путем выполнения chroot в sysroot и исполнения команды
% \begin{minted}[breaklines]{console}
% pkgconf --dump-personality
% \end{minted}

% GitLab
% Раннер гитлаба написан на Go, и существует собранный под RISC-V, но пока его можно найти только в их артефактах сборки.
% На LPi4A не проверял, т.к. наша версия имеет мало места на eMMC и докеру будет неприятно, а внешний SSD периодически отваливается.
% На VisionFive2 потребовалась пересборка ядра, спасибо вики Gentoo, что рассказали что нужно включить в конфиге ядра.
% Даже их официальная инструкция говорит о том, что требуется пересборка ядра.
% \remark{(TODO: аккуратнее прокликать menuconfig, потому что я делал это поздно ночью, мог включить что-то ненужное)}
% Кроме того, с гитлабом есть проблема: синхронизация из \GitHub{} в GitLab требует GitLab Premium, наверное можно взять self-hosted, но это большая боль как по мне.
% Есть ещё Action, который синкает репо ручками, тоже кажется костылём.

% Jenkins
% Java, которая умеет в риск, всё супер. В дебиановском репе пакеты под архитектуру All. На VisionFive2 по умолчанию старые репозитории, и он не стартует. Если вписать http://ftp.ru.debian.org/debian/ и установить более новую Java, то всё становится хорошо
% \begin{minted}{console}
% # update-alternatives --set iptables /usr/sbin/iptables-legacy
% # update-alternatives --set ip6tables /usr/sbin/ip6tables-legacy
% \end{minted}

% docker-pipelines~--- плагин чтобы использовать docker
% \begin{minted}{console}
% # usermod -aG docker jenkins
% \end{minted}

% И серверу и агенту на одной платке тесно.
% Надо разносить, но наверное для компании, если у них уже Jenkins, норм.

% Надо ещё проверить как поллить ГХ на тему коммитов
