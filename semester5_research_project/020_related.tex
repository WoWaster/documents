% !TeX spellcheck = ru_RU
% !TEX root = vkr.tex

\section{Обзор}
\label{sec:relatedworks}

Всегда будет рассматривать компиляцию только для ОС Linux.

Для начала введём некоторые термины, использующиеся при кросс-компиляции (в соответствии с \cite{CrossCompilationAutomake}):
\begin{itemize}
	\item Build~--- система, на которой производится сборка;
	\item Host~--- система, под которую производится сборка;
	\item Target~--- в случае компиляторов, система, для которой будет генерироваться код.
\end{itemize}

\remark{(Примеры?)}

Для небольших приложений \remark{(библиотека --- приложение?)} обычно достаточно минимального набора библиотек, предоставляемого компилятором.

В случае же если требуются дополнительные зависимости, то можно либо собрать их самому, либо создать sysroot какого-нибудь дистрибутива.

Первый способ обычно достаточно трудозатратен и не стоит большого внимания.
Более того, допустим, что все подобные зависимости \enquote{подтягиваются}, например с помощью Git Submodules, вместе с исходным кодом и собираются системой сборки проекта.

Sysroot~--- минимальный образ системы для host платформы, который включает в себя необходимые зависимости для сборки и работы ПО. \remark{(хреновое определение, да)}
Для его создания используются такие утилиты как debootstrap, buildroot, yocto, dnf. \remark{(Первый точно да, вторые два надо подумать, третий должен уметь в федору -- не пробовал)}

\subsection{Виды сборки}

\subsubsection{Нативная сборка}

Самый простой вид сборки~--- от разработчика требуется минимум усилий для сборки своего проекта на другой платформе.
Кроме того, сам процесс компиляции происходит довольно быстро, а зависимости можно установить из репозиториев.
Минусом данного подхода в контексте данной работы является необходимость в покупке устройств на платформе \riscv{}.

\subsubsection{Кросс-компиляция}

\remark{Здесь и далее я пока сдался писать нормальный текст}

Быстро, требует зависимости в сисруте, которые не каждая система сборки нормально переваривает.
Кроме того, некоторые проекты (\OCaml) вообще не умеют в кросс.

\subsubsection{Нативная сборка в эмуляторе}

В Linux есть прекрасная штука~--- binfmt\_misc.
Она распознает заголовок исполняемого файла и в случае необходимости позволяет запускать его через программу обертку, например Wine или QEMU (в usermode). \remark{(подробности про QEMU?)}
В плане поддержки софтом работает как обычная нативная сборка.
Делаем chroot в sysroot или проворачиваем это дело в docker контейнере для riscv.
Из минусов: скорость, QEMU очень медленно работает.
Настолько, что создание большого сисрута может занимать около часа, а потом ещё что-то компилировать.

\subsection{Поддержка в CI}

Такие хостинги для проектов как GitHub и GitLab, пусть и предоставляют собственные сервисы для настройки CI, тем не менее не имеют собственных runner'ов на архитектуре \riscv{}.

\remark{
	(Вообще щас нагуглил: \begin{itemize}
		\item \url{https://github.blog/changelog/2023-10-30-accelerate-your-ci-cd-with-arm-based-hosted-runners-in-github-actions/}
		\item \url{https://circleci.com/blog/managing-ci-cd-pipelines-with-arm-compute-resource-classes/}
		\item \url{https://cloud-v.co/}
	\end{itemize}
	первые два только про арм, но уже хоть что-то.
	Третий по сути тоже self-hosted Jenkins)
}

Но есть возможность сделать self-hosted runner.

\subsubsection{GitHub}

Приложение от GitHub написано в основном на .NET.
На данный момент .NET поддерживает \riscv{} базово.
Кроме того, GH использует скрипты для MSBuild для сборки переносимого приложения.
Есть draft PR\footnote{
	\url{https://github.com/actions/runner/pull/2386}
} для ppc64-le, но выглядит страшно.
Кроме того, даже сейчас, в альфе .NET9, у них нет нормального GC.
\remark{(Наверное если извратиться, то можно попробовать тот же QEMU или Box64, но привет тормоза и привет разъезжающиеся докер образы)}

\subsubsection{GitLab}

Раннер гитлаба написан на Go, и существует собранный под RISC-V, но пока его можно найти только в их артефактах сборки.
На LPi4A не проверял, т.к. наша версия имеет мало места на eMMC и докеру будет неприятно, а внешний SSD периодически отваливается.
На VisionFive2 потребовалась пересборка ядра, спасибо вики Gentoo, что рассказали что нужно включить в конфиге ядра.
Даже их официальная инструкция говорит о том, что требуется пересборка ядра.
\remark{(TODO: аккуратнее прокликать menuconfig, потому что я делал это поздно ночью, мог включить что-то ненужное)}
Кроме того, с гитлабом есть проблема: синхронизация из GitHub в GitLab требует GitLab Premium, наверное можно взять self-hosted, но это большая боль как по мне.
Есть ещё Action, который синкает репо ручками, тоже кажется костылём.

\subsubsection{Jenkins}
Java, которая умеет в риск, всё супер. В дебиановском репе пакеты под архитектуру All. На VisionFive2 по умолчанию старые репозитории, и он не стартует. Если вписать http://ftp.ru.debian.org/debian/ и установить более новую Java, то всё становится хорошо
\begin{minted}{shell}
# update-alternatives --set iptables /usr/sbin/iptables-legacy
# update-alternatives --set ip6tables /usr/sbin/ip6tables-legacy
\end{minted}

docker-pipelines~--- плагин чтобы использовать docker
\begin{minted}{shell}
# usermod -aG docker jenkins
\end{minted}

И серверу и агенту на одной платке тесно.
Надо разносить, но наверное для компании, если у них уже Jenkins, норм.

Надо ещё проверить как поллить ГХ на тему коммитов

\subsection{Системы сборки}

\subsubsection{Make}
Тут скорее всего всё плохо~--- очень легко облажаться и где-нибудь с чем-нибудь накосячить.

\subsubsection{Autotools}
В целом должно быть норм.
Флаг \texttt{--host} есть, могут быть проблемы с сисрутом, но куда-нибудь (например в CC или CFLAGS) его уж можно подсунуть.

\subsubsection{CMake}
Если не извращаться, то должно легко заработать.
Но надо проверить на MROB.

\subsubsection{QMake}
Вот здесь ничего хорошего.

QMake (и весь QT) надо собирать ручками для кросса.
Сделать сисрут будет просто, а вот чтобы потом при копировании оно завелось~--- сложно.
Если точнее нужно брать флаги в билд скриптах нужного дистрибутива.

Если не хочется этой боли, то можно под QEMU или совсем нативно.
В первом случае даже рантайм собирается почти час :).

\subsubsection{Meson}
Не самая приятная система.
Плохая документация для файлов кросс тулчейнов, хотя жить можно.
Ищет зависимости через pkg-config, поэтому для него (а точнее для более новой и адекватной версии --- pkfconf) надо сделать .personality файл, в котором правильно указать пути.
(А пути все абсолютные, а так же надо делать chroot, чтобы не гадать)
