% \documentclass{beamer}
%Для защит онлайн лучше использовать разрешение 16x9
\documentclass[aspectratio=169]{beamer}

%%% Обязательные пакеты
%% Beamer
\usepackage{beamerthemesplit}
\usetheme{SPbGU}
\beamertemplatenavigationsymbolsempty
\usepackage{appendixnumberbeamer}

%% Локализация
\usepackage{fontspec}
\setmainfont{CMU Serif}
\setsansfont{CMU Sans Serif}
\setmonofont{CMU Typewriter Text}
%\setmonofont{Fira Code}[Contextuals=Alternate,Scale=0.9]
%\setmonofont{Inconsolata}
% \newfontfamily\cyrillicfont{CMU Serif}

\usepackage{polyglossia}
\setdefaultlanguage{russian}
\setotherlanguage{english}
\usepackage[autostyle]{csquotes} % Правильные кавычки в зависимости от языка

%% Графика
\usepackage{wrapfig} % Позволяет вставлять графику, обтекаемую текстом
\usepackage{pdfpages} % Позволяет вставлять многостраничные pdf документы в текст

%% Математика
\usepackage{amsmath, amsfonts, amssymb, amsthm, mathtools} % "Адекватная" работа с математикой в LaTeX

% Математические окружения с русским названием
\newtheorem{rutheorem}{Теорема}
\newtheorem{ruproof}{Доказательство}
\newtheorem{rudefinition}{Определение}
\newtheorem{rulemma}{Лемма}

%%% Дополнительные пакеты. Используются в презентации, но могут быть отключены при необходимости
\usepackage{tikz} % Мощный пакет для создание рисунков, однако может очень сильно замедлять компиляцию
\usetikzlibrary{decorations.pathreplacing,calc,shapes,positioning,tikzmark}

\usepackage{multirow} % Ячейка занимающая несколько строк в таблице

%% Пакеты для оформления алгоритмов на псевдокоде
\usepackage[noend]{algpseudocode}
\usepackage{algorithm}
\usepackage{algorithmicx}

\usepackage{fancyvrb}

\NewDocumentCommand{\xxHash}{}{\textsc{xxHash}}
\NewDocumentCommand{\riscv}{}{\textsc{RISC-V}}
\NewDocumentCommand{\xxh}{m}{\textsc{XXH{#1}}}
\NewDocumentCommand{\sew}{}{\textsc{SEW}}
\NewDocumentCommand{\vl}{}{\textsc{VL}}
\NewDocumentCommand{\rvv}{}{\textsc{RVV}}
\usepackage{booktabs}
\usepackage{tabularx}
\usepackage{siunitx} % для таблиц с единицами измерений


% То, что в квадратных скобках, отображается внизу по центру каждого слайда.
\title[Обзор инфраструктуры сборки]{Обзор инфраструктуры сборки проектов с открытым исходным кодом}

% То, что в квадратных скобках, отображается в левом нижнем углу.
\institute[СПбГУ]{}

% То, что в квадратных скобках, отображается в левом нижнем углу.
\author[Николай Пономарев]{Пономарев Николай Алексеевич, группа 21.Б10-мм}

\begin{document}
{
\setbeamertemplate{footline}{}
% Лого университета или организации, отображается в шапке титульного листа
\begin{frame}
	\includegraphics[width=1.4cm]{pictures/SPbGU_Logo.png}
	\vspace{-35pt}
	\hspace{-10pt}
	\begin{center}
		\begin{tabular}{c}
			\scriptsize{Санкт-Петербургский государственный университет} \\
			\scriptsize{Кафедра системного программирования}
		\end{tabular}
		\titlepage
	\end{center}

	\btVFill

	{\scriptsize
		% У научного руководителя должна быть указана научная степень
		\textbf{Научный руководитель:} Я. А. Кириленко, старший преподаватель кафедры системного программирования
	}
	\begin{center}
		\vspace{5pt}
		\scriptsize{Санкт-Петербург\\
			2024}
	\end{center}

\end{frame}
}

\begin{frame}[fragile]
	\frametitle{Введение}
	\begin{itemize}
		\item На рынке всё больше устройств на архитектурах отличных от \amd{}, например \riscv{} или \arm{}
		\item Нужно переносить используемые приложения, но сначала надо научиться собирать приложения на другой архитектуре и настраивать \ci{} для них
		\item В основном изучаем проекты на \textsc{C/C++}
	\end{itemize}
\end{frame}

\begin{frame}
	\frametitle{Постановка задачи}
	\textbf{Целью} данной работы является изучение возможностей по созданию
	инфраструктуры для сборки проектов с открытым исходным кодом.

	\textbf{Задачи}:
	\begin{enumerate}
		\item Провести обзор видов сборки проектов на языках \textsc{C}/\textsc{C++}
		\item Провести обзор поддержки нативной и кросс-сборки в системах сборки
		\item Исследовать возможность настройки \ci{} на архитектурах, отличных от \amd{}
		\item Протестировать изученные возможности систем сборки на реальных проектах
		\item Описать процесс компиляции и настройки \ci{} на примере проектов с различными системами сборки
	\end{enumerate}

\end{frame}

\begin{frame}
	\frametitle{Виды сборки}

	\begin{itemize}
		\item Нативная сборка
		      \begin{itemize}
			      \item Сборка производится на целевой платформе
			      \item Обычный способ сборки
			      \item Требует оборудования на целевой архитектуре
		      \end{itemize}
		\item Кросс-сборка
		      \begin{itemize}
			      \item Целевая платформа отличается от платформы, на которой производится сборка
			      \item Могут быть сложности с удовлетворением зависимостей
			      \item Не во всех проектах есть поддержка
		      \end{itemize}
		\item Нативная сборка в эмуляторе
		      \begin{itemize}
			      \item Приложения считают, что они запущены на своей целевой платформе
			      \item \qemu{} + \textsc{binfmt\_misc} позволяют не запускать виртуальную машину целиком
			      \item Малая скорость исполнения
		      \end{itemize}
	\end{itemize}
\end{frame}

\begin{frame}
	\frametitle{Поддержка кросс-компиляции в системах сборки}

	\begin{itemize}
		\item Make
		      \begin{itemize}
			      \item При аккуратно написанном Makefile поддерживает кросс-компиляцию
		      \end{itemize}
		\item Autotools
		      \begin{itemize}
			      \item Поддерживает кросс-компиляцию из коробки
			      \item Могут быть сложности с указанием зависимостей
		      \end{itemize}
		\item CMake
		      \begin{itemize}
			      \item Хорошая поддержка кросс-сборки
			      \item Иногда требует правки сборочных скриптов
		      \end{itemize}
		\item QMake
		      \begin{itemize}
			      \item Требует сборки в специальном кросс режиме
			      \item Можно использовать нативную сборку в эмуляторе
		      \end{itemize}
		\item Meson
		      \begin{itemize}
			      \item Большие возможности кросс-компиляции
			      \item Плохая документация
			      \item Сложности с поиском зависимостей с помощью pkg-config
		      \end{itemize}
	\end{itemize}
\end{frame}

\begin{frame}
	\frametitle{Поддержка в \ci{} системах}

	\begin{itemize}
		\item \GitHub{}
		      \begin{itemize}
			      \item В своём облаке поддерживает только \amd{}
			      \item Self-hosted runner официально поддерживает \amd{} и \arm{}
			      \item Self-hosted runner написан на \dotnet{}, проверить работу на \riscv{} невозможно
		      \end{itemize}
		\item \gitlab{}
		      \begin{itemize}
			      \item В своём облаке поддерживает только \amd{}
			      \item Self-hosted runner поддерживает множество архитектур, в том числе \riscv{}
		      \end{itemize}
		\item \jenkins{}
		      \begin{itemize}
			      \item Написан на \java{}, которая поддерживает множество архитектур
		      \end{itemize}
	\end{itemize}
\end{frame}

\begin{frame}
	\frametitle{Тестирование}

	\begin{itemize}
		% \item Libusb (Autotools)
		\item SingleApplication (CMake)
		      \begin{itemize}
			      \item \cmake{} пытается использовать утилиты \qt{} для целевой платформы
		      \end{itemize}
		      % \item TRIK Runtime (QMake)
		      % \item Umockdev (Meson)
		      % \begin{itemize}
		      % 	\item
		      % \end{itemize}
		\item TRIK Runtime (Jenkins)
		      \begin{itemize}
			      \item Попытка запустить на StarFive VisionFive V2
			      \item Потребовалась пересборка ядра
			      \item На одной плате серверу и агенту \enquote{тесно}
		      \end{itemize}
	\end{itemize}

\end{frame}

\begin{frame}
	\frametitle{Результаты}
	Полностью достигнуты следующие результаты:
	\begin{enumerate}
		\item Проведен обзор видов сборки проектов на языках \textsc{C}/\textsc{C++}
		\item Проведен обзор поддержки нативной и кросс-сборки в системах сборки
		\item Исследована возможность настройки \ci{} на архитектурах, отличных от \amd{}
		\item Протестированы изученные возможности систем сборки на реальных проектах
	\end{enumerate}
	Частично достигнуты следующие результаты:
	\begin{enumerate}
		\item Описан процесс компиляции и настройки \ci{} на примере проектов с различными системами сборки
	\end{enumerate}

\end{frame}
\end{document}
