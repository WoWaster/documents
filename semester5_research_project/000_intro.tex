% !TeX spellcheck = ru_RU
% !TEX root = vkr.tex

\section*{Введение}
\thispagestyle{withCompileDate}

В последние годы наблюдается увеличение количества устройств на архитектурах, отличных от \amd{}, например \arm{} или \riscv{}.
Если первая имеет достаточно долгую историю и популярна, как в мобильных устройствах, так и в серверах~\cite{aleksandarkChinaHosts402023}, то \riscv{} только набирает популярность~\cite{sperlingRISCVPushesMainstream2022}.
Тем не менее \riscv{} привлекает к себе всё больше внимания~--- это открытая архитектура, позволяющая создавать процессоры под свои нужды, выбирая необходимые расширения процессора или даже создавая собственные.
Поэтому всё больше сил уделяется переносу приложений на неё~\cite{ducassePortingJITCompiler2022,newsHowAlibabaPorting,mcgrewPortingNetBSDRISCV}.

В процессе переноса одной из первых целей является обеспечение возможности сборки проекта.
В общем случае эта задача не решается~--- не любая система сборки поддерживает кросс-компиляцию из коробки, скрипты сборки позволяют \enquote{испортить} встроенную поддержку кросс-сборки, а получить доступ к оборудованию на нужной архитектуре не всегда возможно.
Кроме того, хорошей практикой считается наличие \ci{}~(Continuous Integration), выполняющего сборку проекта и запуск тестов на поддерживаемых архитектурах.

Компилируемые в машинный код языки, такие как \textsc{C} или \textsc{C++}, применяются во множестве различных областей, в том числе для эффективных вычислений~\cite{davisAlgorithm1000SuiteSparse2019}, создания системных компонент\footnote{\url{https://github.com/torvalds/linux}} и утилит, написания прошивок для встраиваемых устройств\footnote{\url{https://github.com/trikset/trikRuntime}}, а также разработки сред исполнения других языков программирования, таких как \dotnet{} и \java{}.
Поэтому основные сложности возникают именно при переносе приложений, написанных на этих языках.

Целью данной работы является систематизация знаний о способах сборки проектов, написанных на языках \textsc{C}/\textsc{C++}, а также возможностей по настройке \ci{}.
