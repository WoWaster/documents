% !TeX spellcheck = ru_RU
% !TEX root = vkr.tex

\section*{Введение}
\thispagestyle{withCompileDate}

Компилируемые языки, такие как \textsc{C} или \textsc{C++}, применяются в множестве различных областей, например для эффективных вычислений~\cite{davisAlgorithm1000SuiteSparse2019}, создания системных компонент\footnote{\url{https://github.com/torvalds/linux}} и утилит, написания прошивок для встраиваемых устройств\footnote{\url{https://github.com/trikset/trikRuntime}}, а также разработки рантаймов других языков, таких как .NET и Java.

Архитектура \riscv{} привлекает всё большее внимание как исследователей, так и разработчиков ПО. \remark{oh really?}

В общем случае задача сборки проекта не решается~--- не любая система сборки поддерживает кросс-компиляцию из коробки, скрипты сборки позволяют "испортить" встроенную поддержку кросса...

Целью данной работы является систематизация знаний о способах сборки проекта для архитектуры \riscv{}, а также настройки \ci{}.

\remark{Версия 1}

Компилируемые языки типа C/C++ активно применяются для <<быстрых>> вычислений, а также для написания рантаймов других языков, например .NET и Java.
\remark{(А ещё есть embedded)}
В силу особенностей \remark{(каких?)} с ними стоит использовать системы сборки, такие как Make, CMake, Autotools, Meson, QMake и др.

Архитектура \riscv{} привлекает всё большее внимание как исследователей, так и разработчиков ПО.
Тем не менее, выбор устройств на \riscv{} пока что ограничен, доступные IP ядра отстают от спецификаций \remark{(надо?)}, а производительности эмуляторов, таких как \qemu{}, недостаточно для повседневных задач.

Перед разработчиками ПО встает задача сборки и тестирования своего ПО на платформе \riscv{}.
Для этого необходим \ci{}, поддержка системы сборки...

Целью данной работы является систематизация знаний о способах сборки проекта для архитектуры \riscv{}, а также настройки \ci{}.
