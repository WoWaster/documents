% !TeX spellcheck = ru_RU
% !TEX root = vkr.tex

\section*{Введение}
\thispagestyle{withCompileDate}

\remark{Текст введения~--- большой набросок, который необходимо расширять и править!}

\remark{Версия 2}

Компилируемые языки, такие как \textsc{C} или \textsc{C++}, применяются во многих областях \remark{(чего?)}, например для ускорения вычислений, создания системных компонент и утилит, написания прошивок \remark{(типа триковского рантайма)} для embedded \remark{(embedded~--- встраеваемый? или всё-таки нет?)} устройств, а также разработки рантаймов других языков, например .NET и Java.
...

\remark{Версия 1}

Компилируемые языки типа C/C++ активно применяются для <<быстрых>> вычислений, а также для написания рантаймов других языков, например .NET и Java.
\remark{(А ещё есть embedded)}
В силу особенностей \remark{(каких?)} с ними стоит использовать системы сборки, такие как Make, CMake, Autotools, Meson, QMake и др.

Архитектура \riscv{} привлекает всё большее внимание как исследователей, так и разработчиков ПО.
Тем не менее, выбор устройств на \riscv{} пока что ограничен, доступные IP ядра отстают от спецификаций \remark{(надо?)}, а производительности эмуляторов, таких как \qemu{}, недостаточно для повседневных задач.

Перед разработчиками ПО встает задача сборки и тестирования своего ПО на платформе \riscv{}.
Для этого необходим CI, поддержка системы сборки...

Целью данной работы является систематизация знаний о способах сборки проекта для архитектуры \riscv{}, а также настройки CI.
