% !TeX spellcheck = ru_RU
% !TEX root = vkr.tex

\section*{Введение}
\thispagestyle{withCompileDate}

\remark{Текст введения~--- большой набросок, который необходимо расширять и править!}

Компилируемые языки типа C/C++ активно применяются для <<быстрых>> вычислений, а также для написания рантаймов других языков, например .NET и Java.
\remark{(А ещё есть embedded)}
В силу особенностей \remark{(каких?)} с ними стоит использовать системы сборки, такие как Make, CMake, Autotools, Meson, QMake и др.

Архитектура \riscv{} привлекает всё большее внимание как исследователей, так и разработчиков ПО.
Тем не менее, выбор устройств на \riscv{} пока что ограничен, доступные IP ядра отстают от спецификаций \remark{(надо?)}, а производительности эмуляторов, таких как QEMU, недостаточно для повседневных задач.

Перед разработчиками ПО встает задача сборки и тестирования своего ПО на платформе \riscv{}.
Для этого необходим CI, поддержка системы сборки...

Целью данной работы является систематизация знаний о способах сборки проекта для архитектуры \riscv{}, а также настройки CI.
