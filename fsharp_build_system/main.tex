\documentclass[aspectratio=169]{beamer}
\usepackage{fontspec}
\setmainfont{Yanone Kaffeesatz}
\setsansfont{Noto Sans}
\setmonofont{Fira Code}
%%% Fonts and language setup.
\usepackage{polyglossia}
\usepackage[autostyle]{csquotes}
%% Math
\usepackage{amsmath, amsfonts, amssymb, amsthm, mathtools} % Advanced math tools.
\usepackage{unicode-math} % Allow TTF and OTF fonts in math and allow direct typing unicode math characters.
\unimathsetup{
    warnings-off={
            mathtools-colon,
            mathtools-overbracket
        }
}
\setmathfont{Fira Math}
\newfontfamily{\cyrillicfont}{Fira Math}


\usetheme{Rochester}
\usecolortheme[style=light]{Nord}
\usefonttheme{Nord}
\setbeamertemplate{navigation symbols}{}%remove navigation symbols


\usepackage{minted}
\usemintedstyle{catppuccin-latte}

\title{Инфраструктура сборки F\#}
\author{Николай Пономарев}
\date{}


%%% Polyglossia setup after (nearly) everything as described in documentation.
\setdefaultlanguage{russian}
\setotherlanguage{english}

\begin{document}

\begin{frame}
    \titlepage
\end{frame}

\begin{frame}
    \frametitle{FAKE}

    FAKE\footnote{\href{https://fake.build/index.html}{Официальный сайт}}~%
    \uncover<2->{
        (F\# Make)~--- система автоматизации сборки и релиза приложений.

        Целью проекта является замена \enquote{устаревших} скриптов на bash или powershell, скриптами на F\#.

        Способы использования FAKE:
        \begin{itemize}
            \item Уменьшение количества зависимостей в CI/CD
            \item Создание воспроизводимых и тестируемых скриптов
            \item Замена скриптов написанных на bash/powershell
            \item Автоматизация ручных действий
        \end{itemize}
    }
\end{frame}

\begin{frame}
    \frametitle{Использование FAKE}
    FAKE можно запускать несколькими способами:
    \begin{itemize}
        \item FAKE runner
              \begin{itemize}
                  \item По сути заменяет MSBuild и устанавливается как отдельное приложение
              \end{itemize}
        \item F\# interactive (FSI)
              \begin{itemize}
                  \item Выполняется как обычный F\# скрипт
              \end{itemize}
        \item Использование готового проекта
              \begin{itemize}
                  \item MiniScaffold
                  \item SAFE-Dojo
              \end{itemize}
    \end{itemize}
\end{frame}

\begin{frame}[fragile]
    \frametitle{Пример использования FAKE}
    \begin{minted}[breaklines,fontsize=\small]{fsharp}
#r "paket:
nuget Fake.Core.Target //"
open Fake.Core
// *** Define Targets ***
Target.create "Clean" (fun _ -> Trace.log " --- Cleaning stuff --- ")
Target.create "Build" (fun _ -> Trace.log " --- Building the app --- ")
Target.create "Deploy" (fun _ -> Trace.log " --- Deploying app --- ")
open Fake.Core.TargetOperators
// *** Define Dependencies ***
"Clean" ==> "Build" ==> "Deploy"
// *** Start Build ***
Target.runOrDefault "Deploy"
    \end{minted}
\end{frame}

\begin{frame}
    \frametitle{Paket}
    Paket\footnote{\href{https://fsprojects.github.io/Paket/index.html}{Официальный сайт}}~--- замена NuGet из мира F\#

    Зачем оно надо:
    \begin{itemize}
        \item Правильная работа с транзитивными зависимостями
              \begin{itemize}
                  \item NuGet не различает транзитивные и прямые зависимости
                  \item NuGet выбирает версии не самым адекватным образом
              \end{itemize}
    \end{itemize}
\end{frame}

\begin{frame}
    \frametitle{MiniScaffold}
    MiniScaffold\footnote{\href{https://github.com/TheAngryByrd/MiniScaffold}{Репозиторий}}~--- готовый шаблон проекта для F\#.

    \begin{minipage}{0.45\linewidth}
        Зачем это надо:
        \begin{itemize}
            \item Уже готовая структура проекта
            \item Тщательно подобранные библиотеки
            \item Готовые настройки репозитория
            \item Инфраструктура, заточенная под F\#
        \end{itemize}
    \end{minipage}
    \begin{minipage}{0.45\linewidth}
        Что включено:
        \begin{itemize}
            \item Автоматизация сборки через FAKE
            \item Управление пакетами через Paket
            \item Модульное тестирование и оценка покрытия с помощью Expecto/Altcover
            \item Форматирование кода с помощью Fantomas
        \end{itemize}
    \end{minipage}


\end{frame}
\end{document}
