% !TeX spellcheck = ru_RU
% !TEX root = vkr.tex

% Note: под закрытостью подразумевается закрытость микроархитектуры и необходимость платы лицензионных отчислений

Большинство современных устройств работает на процессорах от крупных корпораций, таких как \textenglish{Intel, ARM, IBM}.
Закрытость архитектуры и сложность реализации таких процессоров не позволяют использовать их для обучения инженеров.
Для этих целей университет Беркли с 1982 года~\cite{Séquin:CSD-82-106} разрабатывает собственные \textenglish{RISC} архитектуры.
Последняя из них, \riscv{}, была разработана в 2010 году с целью не только быть удобной для обучения студентов, но и быть применимой в реальных системах.
Она заинтересовала не только исследователей, но и множество компаний, в том числе \textenglish{Seagate, Alibaba} и \textenglish{Nvidia}~\cite{RISCVpopularity}.
Для массового применения архитектура должна обладать развитой экосистемой, в том числе иметь оптимизации для часто используемых алгоритмов.

Одним из таких алгоритмов является алгоритм проверки целостности данных \textenglish{CRC~(Cyclic Redundancy Check)}, который имеет широкий спектр применения.
В том числе применяется в ядре \textenglish{Linux}, где его используют файловые системы, такие как \textenglish{ext4} и \textenglish{Btrfs}, а так же алгоритмы сжатия ядра \textenglish{gzip} и \textenglish{bzip2}.

\textenglish{CRC}~--- параметризуемый алгоритм, который имеет множество вариантов.
В данной работе речь пойдет об оптимизации 32-битного \textenglish{CRC}, чаще называемого \crctt{}.

Ускорению данного алгоритма посвящены работы~\cite{kadatch2010everything, fastestCRC32}.
Одной из самых быстрых оптимизаций является оптимизация от компании \textenglish{Intel}~\cite{gopal2009fast}, использующая специальную инструкцию процессора для умножения многочленов.
Кроме того, во многих процессорах (в том числе, процессорах \textenglish{Intel} и \textenglish{ARM}) существует инструкция для вычисления \crctt{}\footnote{Однако обычно эти инструкции считают \crctt{} по другому многочлену, имеющему более узкое применение}.

Базовый набор инструкций \riscv{} не содержит в себе необходимых инструкций, но новые расширения для \riscv{} позволяют реализовать оптимизацию \textenglish{Intel} на данной платформе. \blfootnote{\raggedleft Дата сборки: \today}
