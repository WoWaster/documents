% !TeX spellcheck = ru_RU
% !TEX root = vkr.tex

Эксперименты проводились на симуляторе gem5 со следующими характеристиками:
\begin{itemize}
    \item MinorCPU с частотой 1~ГГц
    \item 512~Мб ОЗУ DDR3 с частотой 1600 МГц
\end{itemize}

Исходные файлы компилировались с флагами \texttt{-O3 -static}, а затем запускались на симуляторе.

\begin{table}[h]
    \begin{center}
        \begin{tabularx}{0.95\linewidth}{*3{>{\raggedleft\arraybackslash}X}}
            \toprule
            Объем данных, байт & Стандартный алгоритм, тиков & Оптимизированный алгоритм, тиков \\ \midrule
            128                & $121.4 \cdot 10^6$          & $109.8 \cdot 10^6$               \\ \midrule
            1024               & $219.4 \cdot 10^6$          & $132.0 \cdot 10^6$               \\ \midrule
            8192               & $1116.1 \cdot 10^6$         & $224.2 \cdot 10^6$               \\ \midrule
            65536              & $7444.9 \cdot 10^6$         & $1077.9 \cdot 10^6$              \\
            \bottomrule
        \end{tabularx}
    \end{center}
    \caption{Результаты измерений}
\end{table}

По результатам эксперимента, можно сделать вывод о том, что оптимизация действительно работает и дает ощутимый прирост производительности на больших объемах данных.
