% !TeX spellcheck = ru_RU
% !TEX root = vkr.tex

Эксперименты проводились на симуляторе gem5 со следующими характеристиками:
\begin{itemize}
    \item MinorCPU с частотой 1~ГГц и размером кэша L1 в 64~Кб
    \item 512~Мб ОЗУ DDR4 с частотой 2400 МГц
\end{itemize}

Исходные файлы компилировались с флагами \texttt{-O3 -static}, а затем запускались на симуляторе.
В качестве стандартного алгоритма использовался вариант \crctt{} с таблицей размером 16~Кб.

\begin{table}[h]
    \begin{center}
        \begin{tabularx}{0.95\linewidth}{*3{>{\raggedleft\arraybackslash}X}}
            \toprule
            Объем данных, байт & Стандартный алгоритм, тиков & Оптимизированный алгоритм, тиков \\ \midrule
            128                & $308.5 \cdot 10^3$          & $83\cdot 10^3$                   \\ \midrule
            1024               & $2277.5 \cdot 10^3$         & $424 \cdot 10^3$                 \\ \midrule
            8192               & $18 \cdot 10^6$             & $3.6 \cdot 10^6$                 \\ \midrule
            65536              & $191 \cdot 10^6$            & $30.5 \cdot 10^6$                \\
            \bottomrule
        \end{tabularx}
    \end{center}
    \caption{Результаты измерений}
\end{table}

При малом объеме, данные полностью помещаются в кэш и время исполнения стабильно.
В ином случае возникают задержки, которые влияют на время работы.

По результатам эксперимента, можно сделать вывод о том, что оптимизация действительно работает и дает ощутимый прирост производительности на больших объемах данных.
