\documentclass[a4paper, 12pt]{moderncv}
%%% PDF settings
\pdfvariable minorversion 7 % Set PDF version to 1.7.

%%% Fonts and language setup.
\usepackage{polyglossia}
% Setup fonts.
\usepackage{fontspec}
% \setmainfont{CMU Serif}
% \setsansfont{CMU Sans Serif}
\setmainfont{Roboto} % Здесь поменял местами, потому что локализация походу ломает дефолтный санс шрифт
\setsansfont{CMU Serif}
\setmonofont{CMU Typewriter Text}

\usepackage{microtype} % Add fancy-schmancy font tricks

\usepackage{xcolor} % Add colors support.

%% Math
\usepackage{amsmath, amsfonts, amssymb, amsthm, mathtools} % Advanced math tools.
\usepackage{thmtools}
\usepackage{unicode-math} % Allow TTF and OTF fonts in math and allow direct typing unicode math characters.
\unimathsetup{
    warnings-off={
            mathtools-colon,
            mathtools-overbracket
        }
}
\setmathfont{Latin Modern Math} % default
\setmathfont[range={\setminus,\varnothing,\smashtimes}]{Asana Math}

%%% Images
\usepackage{graphicx}
\graphicspath{{figures/}}
\usepackage{import}

%%% Polyglossia setup after (nearly) everything as described in documentation.
\setdefaultlanguage{russian}
\setotherlanguage{english}

\usepackage{csquotes}


\moderncvtheme[orange]{classic}

\photo[48pt]{photo}
\firstname{Николай}
\familyname{Пономарев}
\email{wowasterdev@gmail.com}
\address{г. Санкт-Петербург}{Россия}
\extrainfo{\faGithub{} \href{https://github.com/WoWaster}{WoWaster}}

\sethintscolumnlength{0.3\textwidth} % Русский слишком широкий :(

\begin{document}

\maketitle

\section{Интересы}

\cvline{}{Системное программирование:}
\cvlistitem{Программирование embedded устройств}
\cvlistitem{Компиляторы}
\cvlistitem{RTL проектирование}
\cvlistitem{Функциональные языки}
\cvlistitem{Оптимизация ПО}

\section{Образование}

\cventry{2021~-- н.~в.}{Бакалавр}{Санкт-Петербургский государственный университет}{}{}{Направление: Технологии программирования (Математическое обеспечение и администрирование
    информационных систем)}

\section{Опыт работы}

\cventry{июль 2023~-- н.~в.}{Лаборант-исследователь}{Лаборатория YADRO}{СПбГУ}{}{}

\section{Технические навыки}

\cvline{ЯП}{Haskell, F\#, C, Python, C\#, RISC-V ASM}
\cvline{Технологии}{GNU/Linux, Git, QEMU}

\section{Владение языками}

\cvlanguage{Русский}{Родной}{}
\cvlanguage{Английский}{B2}{}



\end{document}
